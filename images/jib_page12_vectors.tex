\documentclass{standalone}
\usepackage{tikz}
\usepackage{amsmath}
\usetikzlibrary{angles, quotes}

\begin{document}

\begin{tikzpicture}
  % Define angles in degrees
  \def\A{102} % Angle at A
  \def\B{42}  % Angle at B
  \def\C{36}  % Angle at C

  % Define side AB (arbitrary length)
  \def\AB{5}

  % Compute side lengths using Law of Sines
  \pgfmathsetmacro{\BC}{\AB*sin(\A)/sin(\C)} % Side BC, opposite angle A
  \pgfmathsetmacro{\CA}{\AB*sin(\B)/sin(\C)} % Side CA, opposite angle B

  % Define coordinates for vertical AB on the left
  \coordinate (B) at (0,0);           % Vertex B at origin
  \coordinate (A) at (0,\AB);         % Vertex A above B to make AB vertical
  \coordinate (C) at ({\CA*sin(\A)}, {\AB - \CA*cos(\A)}); % Ensures angle at A = 102°, C to the right

  % Draw triangle
  \draw (A) -- (B) node[midway, left, xshift=-2mm] {$W = 37.4938\,\text{kN}$};
  \draw (A) -- (C);
  \draw (C) -- (B);

  % Label vertices
  \node[above left] at (A) {$A$};
  \node[below left] at (B) {$B$};
  \node[right] at (C) {$C$};

  % Draw and label angles
  \pic [draw, "$102^\circ$", angle radius=1cm] {angle = B--A--C};
  \pic [draw, "$42^\circ$", angle radius=1cm] {angle = C--B--A};
  \pic [draw, "$36^\circ$", angle radius=1cm] {angle = A--C--B};
\end{tikzpicture}

\end{document}