% Options for packages loaded elsewhere
% Options for packages loaded elsewhere
\PassOptionsToPackage{unicode}{hyperref}
\PassOptionsToPackage{hyphens}{url}
\PassOptionsToPackage{dvipsnames,svgnames,x11names}{xcolor}
%
\documentclass[
  letterpaper,
  DIV=11,
  numbers=noendperiod]{scrartcl}
\usepackage{xcolor}
\usepackage{amsmath,amssymb}
\setcounter{secnumdepth}{-\maxdimen} % remove section numbering
\usepackage{iftex}
\ifPDFTeX
  \usepackage[T1]{fontenc}
  \usepackage[utf8]{inputenc}
  \usepackage{textcomp} % provide euro and other symbols
\else % if luatex or xetex
  \usepackage{unicode-math} % this also loads fontspec
  \defaultfontfeatures{Scale=MatchLowercase}
  \defaultfontfeatures[\rmfamily]{Ligatures=TeX,Scale=1}
\fi
\usepackage{lmodern}
\ifPDFTeX\else
  % xetex/luatex font selection
\fi
% Use upquote if available, for straight quotes in verbatim environments
\IfFileExists{upquote.sty}{\usepackage{upquote}}{}
\IfFileExists{microtype.sty}{% use microtype if available
  \usepackage[]{microtype}
  \UseMicrotypeSet[protrusion]{basicmath} % disable protrusion for tt fonts
}{}
\makeatletter
\@ifundefined{KOMAClassName}{% if non-KOMA class
  \IfFileExists{parskip.sty}{%
    \usepackage{parskip}
  }{% else
    \setlength{\parindent}{0pt}
    \setlength{\parskip}{6pt plus 2pt minus 1pt}}
}{% if KOMA class
  \KOMAoptions{parskip=half}}
\makeatother
% Make \paragraph and \subparagraph free-standing
\makeatletter
\ifx\paragraph\undefined\else
  \let\oldparagraph\paragraph
  \renewcommand{\paragraph}{
    \@ifstar
      \xxxParagraphStar
      \xxxParagraphNoStar
  }
  \newcommand{\xxxParagraphStar}[1]{\oldparagraph*{#1}\mbox{}}
  \newcommand{\xxxParagraphNoStar}[1]{\oldparagraph{#1}\mbox{}}
\fi
\ifx\subparagraph\undefined\else
  \let\oldsubparagraph\subparagraph
  \renewcommand{\subparagraph}{
    \@ifstar
      \xxxSubParagraphStar
      \xxxSubParagraphNoStar
  }
  \newcommand{\xxxSubParagraphStar}[1]{\oldsubparagraph*{#1}\mbox{}}
  \newcommand{\xxxSubParagraphNoStar}[1]{\oldsubparagraph{#1}\mbox{}}
\fi
\makeatother


\usepackage{longtable,booktabs,array}
\usepackage{calc} % for calculating minipage widths
% Correct order of tables after \paragraph or \subparagraph
\usepackage{etoolbox}
\makeatletter
\patchcmd\longtable{\par}{\if@noskipsec\mbox{}\fi\par}{}{}
\makeatother
% Allow footnotes in longtable head/foot
\IfFileExists{footnotehyper.sty}{\usepackage{footnotehyper}}{\usepackage{footnote}}
\makesavenoteenv{longtable}
\usepackage{graphicx}
\makeatletter
\newsavebox\pandoc@box
\newcommand*\pandocbounded[1]{% scales image to fit in text height/width
  \sbox\pandoc@box{#1}%
  \Gscale@div\@tempa{\textheight}{\dimexpr\ht\pandoc@box+\dp\pandoc@box\relax}%
  \Gscale@div\@tempb{\linewidth}{\wd\pandoc@box}%
  \ifdim\@tempb\p@<\@tempa\p@\let\@tempa\@tempb\fi% select the smaller of both
  \ifdim\@tempa\p@<\p@\scalebox{\@tempa}{\usebox\pandoc@box}%
  \else\usebox{\pandoc@box}%
  \fi%
}
% Set default figure placement to htbp
\def\fps@figure{htbp}
\makeatother





\setlength{\emergencystretch}{3em} % prevent overfull lines

\providecommand{\tightlist}{%
  \setlength{\itemsep}{0pt}\setlength{\parskip}{0pt}}



 


\KOMAoption{captions}{tableheading}
\makeatletter
\@ifpackageloaded{caption}{}{\usepackage{caption}}
\AtBeginDocument{%
\ifdefined\contentsname
  \renewcommand*\contentsname{Table of contents}
\else
  \newcommand\contentsname{Table of contents}
\fi
\ifdefined\listfigurename
  \renewcommand*\listfigurename{List of Figures}
\else
  \newcommand\listfigurename{List of Figures}
\fi
\ifdefined\listtablename
  \renewcommand*\listtablename{List of Tables}
\else
  \newcommand\listtablename{List of Tables}
\fi
\ifdefined\figurename
  \renewcommand*\figurename{Figure}
\else
  \newcommand\figurename{Figure}
\fi
\ifdefined\tablename
  \renewcommand*\tablename{Table}
\else
  \newcommand\tablename{Table}
\fi
}
\@ifpackageloaded{float}{}{\usepackage{float}}
\floatstyle{ruled}
\@ifundefined{c@chapter}{\newfloat{codelisting}{h}{lop}}{\newfloat{codelisting}{h}{lop}[chapter]}
\floatname{codelisting}{Listing}
\newcommand*\listoflistings{\listof{codelisting}{List of Listings}}
\makeatother
\makeatletter
\makeatother
\makeatletter
\@ifpackageloaded{caption}{}{\usepackage{caption}}
\@ifpackageloaded{subcaption}{}{\usepackage{subcaption}}
\makeatother
\usepackage{bookmark}
\IfFileExists{xurl.sty}{\usepackage{xurl}}{} % add URL line breaks if available
\urlstyle{same}
\hypersetup{
  pdftitle={Modes of Heat Transfer},
  colorlinks=true,
  linkcolor={blue},
  filecolor={Maroon},
  citecolor={Blue},
  urlcolor={Blue},
  pdfcreator={LaTeX via pandoc}}


\title{Modes of Heat Transfer}
\author{}
\date{}
\begin{document}
\maketitle


\subsection{Temperature Difference Across a Heat Exchanger
Endplate}\label{temperature-difference-across-a-heat-exchanger-endplate}

The shell diameter of a large heat exchanger is 1.8 m. The flat endplate
(cover) at one end is made of carbon steel, 45 mm thick, and is
uninsulated. If the maximum allowable heat loss through the cover, to
avoid insulation is 150 MJ/h, determine the temperature difference
permitted across the endplate. (For steel, k = 55 W/m°C)

\textbf{Given}

\begin{itemize}
\tightlist
\item
  Shell diameter: (D = 1.8~\(\mathrm{m}\))\\
\item
  Endplate thickness: (s = 45~\(\mathrm{mm}\) = 0.045~\(\mathrm{m}\))\\
\item
  Material: carbon steel, thermal conductivity: (k =
  55~\(\mathrm{W/m\cdot K}\))\\
\item
  Maximum allowable heat loss:
  (\(Q_\mathrm{max} = 150\)~\(\mathrm{MJ/h}\))\\
\item
  Endplate is \textbf{uninsulated}
\end{itemize}

Calculate the \textbf{temperature difference} \(\Delta T\) across the
endplate to limit heat loss.

\begin{enumerate}
\def\labelenumi{\arabic{enumi}.}
\tightlist
\item
  \textbf{Area of the circular endplate}
\end{enumerate}

\[
A = \pi \left( \frac{D}{2} \right)^2 = \pi \left( \frac{1.8}{2} \right)^2
\]

\[
A = \pi \times 0.9^2 = 2.5447\ \mathrm{m^2}
\]

\begin{enumerate}
\def\labelenumi{\arabic{enumi}.}
\setcounter{enumi}{1}
\tightlist
\item
  \textbf{Convert heat loss to Watts}
\end{enumerate}

\[
Q_\mathrm{max} = 150\ \mathrm{MJ/h} = 150 \times 10^6\ \mathrm{J/h}
\]

\[
1\ \mathrm{h} = 3600\ \mathrm{s} \quad \Rightarrow \quad Q = \frac{150 \times 10^6}{3600} = 41666.7\ \mathrm{W}
\]

\begin{enumerate}
\def\labelenumi{\arabic{enumi}.}
\setcounter{enumi}{2}
\tightlist
\item
  \textbf{Use law of conduction for a flat plate}
\end{enumerate}

\[
Q = \frac{k A \Delta T}{s} \quad \Rightarrow \quad \Delta T = \frac{Q s}{k A}
\]

Substitute values:

\[
\Delta T = \frac{41666.7 \times 0.045}{55 \times 2.5447}
\]

\[
\Delta T = \frac{1875.0}{139.96} = 13.39\ \mathrm{K}
\]

\textbf{Result}

\[
\boxed{\Delta T = 13.4^\circ \mathrm{C}}
\]

\textbf{Notes:}

\begin{itemize}
\item
  The maximum allowable \textbf{temperature difference across the
  endplate} is 13.4°C to keep heat loss below 150 MJ/h.
\item
  If the temperature difference exceeds this, insulation would be
  required.
\end{itemize}

\subsection{Hot Metal Convection Heat
Transfer}\label{hot-metal-convection-heat-transfer}

A hot metal plate measuring \textbf{1.2 m × 0.8 m} is exposed to air at
\textbf{25°C}. The surface temperature of the plate is maintained at
\textbf{85°C}. If the \textbf{convective heat transfer coefficient}
(surface heat transfer coefficient) between the plate and air is
\(h_A = 18 \, \text{W/m}^2\text{·°C}\), calculate the \textbf{rate of
heat loss by convection} from the entire plate surface.

\textbf{Given:}

\begin{itemize}
\item
  Plate dimensions: (\(1.2\  \text{m} \times 0.8\  \text{m}\))
\item
  Plate surface area: \(A = 1.2 \times 0.8 = 0.96 \, \text{m}^2\)
\item
  Surface temperature: \(T_s = 85^\circ \text{C}\)
\item
  Air temperature: \(T_f = 25^\circ \text{C}\)
\item
  Convection coefficient: \(h_A = 18 \, \text{W/m}^2\text{·°C}\)
\end{itemize}

\textbf{Solution}

The rate of convective heat loss is given by:

\[
Q = h_A \, A \, (T_s - T_f)
\]

Substituting the values:

\[
Q = 18 \times 0.96 \times (85 - 25)
\]

\[
Q = 18 \times 0.96 \times 60
\]

\[
Q = 1036.8 \, \text{W}
\]

\textbf{Answer}

\[
\boxed{Q = 1036.8 \, \text{W}}
\]

\subsection{Radiant Heat from a Flat Circular
Plate}\label{radiant-heat-from-a-flat-circular-plate}

A flat circular plate is \textbf{400 mm} in diameter. Calculate the
theoretical quantity of heat radiated per hour when its temperature is
\textbf{227 °C} and the temperature of its surrounds is \textbf{27 °C}.

\textbf{Given}:

\begin{itemize}
\item
  Stefan--Boltzmann constant:
  \(\sigma = 5.6703\times10^{-11}\ \mathrm{kW\,m^{-2}\,K^{-4}}\)
\item
  Emissivity: \(\varepsilon = 1\) (ideal black body)
\end{itemize}

\begin{enumerate}
\def\labelenumi{\arabic{enumi}.}
\tightlist
\item
  \textbf{Convert temperatures to Kelvin}
\end{enumerate}

\[
T_1 = 227 + 273 = 500\ \mathrm{K}, \quad
T_2 = 27 + 273 = 300\ \mathrm{K}
\]

\begin{enumerate}
\def\labelenumi{\arabic{enumi}.}
\setcounter{enumi}{1}
\tightlist
\item
  \textbf{Calculate the area of the circular plate}
\end{enumerate}

\[
A = \pi \left( \frac{D}{2} \right)^2 = \pi \left( \frac{0.4}{2} \right)^2 \approx 0.125664\ \mathrm{m^2}
\]

\begin{enumerate}
\def\labelenumi{\arabic{enumi}.}
\setcounter{enumi}{2}
\tightlist
\item
  \textbf{Apply the formula}
\end{enumerate}

\[
Q = \sigma \times \varepsilon \times A \times t \times \left( T_1^4 - T_2^4 \right)
\]

Substituting the values:

\[
Q = 5.6703 \times 10^{-11} \times 1 \times 0.125664 \times 3600 \times \left(500^4 - 300^4\right)
\] \[
Q = 1395.46\ \mathrm{kWh}
\]

\textbf{Result}

\[
\text{Plate area: } 0.125664\ \mathrm{m^2}
\] \[
\text{Emissivity: } \varepsilon = 1
\] \[
\boxed{\text{Heat radiated (per hour): } 1395.46\ \mathrm{kWh}}
\]

\textbf{Note:} This represents the \textbf{theoretical maximum
radiation} for a perfect black body. Real materials would radiate less
depending on their emissivity \(\varepsilon < 1\).




\end{document}
