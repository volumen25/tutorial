% Options for packages loaded elsewhere
% Options for packages loaded elsewhere
\PassOptionsToPackage{unicode}{hyperref}
\PassOptionsToPackage{hyphens}{url}
%
\documentclass[
  letterpaper,
]{book}
\usepackage{xcolor}
\usepackage{amsmath,amssymb}
\setcounter{secnumdepth}{5}
\usepackage{iftex}
\ifPDFTeX
  \usepackage[T1]{fontenc}
  \usepackage[utf8]{inputenc}
  \usepackage{textcomp} % provide euro and other symbols
\else % if luatex or xetex
  \usepackage{unicode-math} % this also loads fontspec
  \defaultfontfeatures{Scale=MatchLowercase}
  \defaultfontfeatures[\rmfamily]{Ligatures=TeX,Scale=1}
\fi
\usepackage{lmodern}
\ifPDFTeX\else
  % xetex/luatex font selection
\fi
% Use upquote if available, for straight quotes in verbatim environments
\IfFileExists{upquote.sty}{\usepackage{upquote}}{}
\IfFileExists{microtype.sty}{% use microtype if available
  \usepackage[]{microtype}
  \UseMicrotypeSet[protrusion]{basicmath} % disable protrusion for tt fonts
}{}
\makeatletter
\@ifundefined{KOMAClassName}{% if non-KOMA class
  \IfFileExists{parskip.sty}{%
    \usepackage{parskip}
  }{% else
    \setlength{\parindent}{0pt}
    \setlength{\parskip}{6pt plus 2pt minus 1pt}}
}{% if KOMA class
  \KOMAoptions{parskip=half}}
\makeatother
% Make \paragraph and \subparagraph free-standing
\makeatletter
\ifx\paragraph\undefined\else
  \let\oldparagraph\paragraph
  \renewcommand{\paragraph}{
    \@ifstar
      \xxxParagraphStar
      \xxxParagraphNoStar
  }
  \newcommand{\xxxParagraphStar}[1]{\oldparagraph*{#1}\mbox{}}
  \newcommand{\xxxParagraphNoStar}[1]{\oldparagraph{#1}\mbox{}}
\fi
\ifx\subparagraph\undefined\else
  \let\oldsubparagraph\subparagraph
  \renewcommand{\subparagraph}{
    \@ifstar
      \xxxSubParagraphStar
      \xxxSubParagraphNoStar
  }
  \newcommand{\xxxSubParagraphStar}[1]{\oldsubparagraph*{#1}\mbox{}}
  \newcommand{\xxxSubParagraphNoStar}[1]{\oldsubparagraph{#1}\mbox{}}
\fi
\makeatother

\usepackage{color}
\usepackage{fancyvrb}
\newcommand{\VerbBar}{|}
\newcommand{\VERB}{\Verb[commandchars=\\\{\}]}
\DefineVerbatimEnvironment{Highlighting}{Verbatim}{commandchars=\\\{\}}
% Add ',fontsize=\small' for more characters per line
\usepackage{framed}
\definecolor{shadecolor}{RGB}{241,243,245}
\newenvironment{Shaded}{\begin{snugshade}}{\end{snugshade}}
\newcommand{\AlertTok}[1]{\textcolor[rgb]{0.68,0.00,0.00}{#1}}
\newcommand{\AnnotationTok}[1]{\textcolor[rgb]{0.37,0.37,0.37}{#1}}
\newcommand{\AttributeTok}[1]{\textcolor[rgb]{0.40,0.45,0.13}{#1}}
\newcommand{\BaseNTok}[1]{\textcolor[rgb]{0.68,0.00,0.00}{#1}}
\newcommand{\BuiltInTok}[1]{\textcolor[rgb]{0.00,0.23,0.31}{#1}}
\newcommand{\CharTok}[1]{\textcolor[rgb]{0.13,0.47,0.30}{#1}}
\newcommand{\CommentTok}[1]{\textcolor[rgb]{0.37,0.37,0.37}{#1}}
\newcommand{\CommentVarTok}[1]{\textcolor[rgb]{0.37,0.37,0.37}{\textit{#1}}}
\newcommand{\ConstantTok}[1]{\textcolor[rgb]{0.56,0.35,0.01}{#1}}
\newcommand{\ControlFlowTok}[1]{\textcolor[rgb]{0.00,0.23,0.31}{\textbf{#1}}}
\newcommand{\DataTypeTok}[1]{\textcolor[rgb]{0.68,0.00,0.00}{#1}}
\newcommand{\DecValTok}[1]{\textcolor[rgb]{0.68,0.00,0.00}{#1}}
\newcommand{\DocumentationTok}[1]{\textcolor[rgb]{0.37,0.37,0.37}{\textit{#1}}}
\newcommand{\ErrorTok}[1]{\textcolor[rgb]{0.68,0.00,0.00}{#1}}
\newcommand{\ExtensionTok}[1]{\textcolor[rgb]{0.00,0.23,0.31}{#1}}
\newcommand{\FloatTok}[1]{\textcolor[rgb]{0.68,0.00,0.00}{#1}}
\newcommand{\FunctionTok}[1]{\textcolor[rgb]{0.28,0.35,0.67}{#1}}
\newcommand{\ImportTok}[1]{\textcolor[rgb]{0.00,0.46,0.62}{#1}}
\newcommand{\InformationTok}[1]{\textcolor[rgb]{0.37,0.37,0.37}{#1}}
\newcommand{\KeywordTok}[1]{\textcolor[rgb]{0.00,0.23,0.31}{\textbf{#1}}}
\newcommand{\NormalTok}[1]{\textcolor[rgb]{0.00,0.23,0.31}{#1}}
\newcommand{\OperatorTok}[1]{\textcolor[rgb]{0.37,0.37,0.37}{#1}}
\newcommand{\OtherTok}[1]{\textcolor[rgb]{0.00,0.23,0.31}{#1}}
\newcommand{\PreprocessorTok}[1]{\textcolor[rgb]{0.68,0.00,0.00}{#1}}
\newcommand{\RegionMarkerTok}[1]{\textcolor[rgb]{0.00,0.23,0.31}{#1}}
\newcommand{\SpecialCharTok}[1]{\textcolor[rgb]{0.37,0.37,0.37}{#1}}
\newcommand{\SpecialStringTok}[1]{\textcolor[rgb]{0.13,0.47,0.30}{#1}}
\newcommand{\StringTok}[1]{\textcolor[rgb]{0.13,0.47,0.30}{#1}}
\newcommand{\VariableTok}[1]{\textcolor[rgb]{0.07,0.07,0.07}{#1}}
\newcommand{\VerbatimStringTok}[1]{\textcolor[rgb]{0.13,0.47,0.30}{#1}}
\newcommand{\WarningTok}[1]{\textcolor[rgb]{0.37,0.37,0.37}{\textit{#1}}}

\usepackage{longtable,booktabs,array}
\usepackage{calc} % for calculating minipage widths
% Correct order of tables after \paragraph or \subparagraph
\usepackage{etoolbox}
\makeatletter
\patchcmd\longtable{\par}{\if@noskipsec\mbox{}\fi\par}{}{}
\makeatother
% Allow footnotes in longtable head/foot
\IfFileExists{footnotehyper.sty}{\usepackage{footnotehyper}}{\usepackage{footnote}}
\makesavenoteenv{longtable}
\usepackage{graphicx}
\makeatletter
\newsavebox\pandoc@box
\newcommand*\pandocbounded[1]{% scales image to fit in text height/width
  \sbox\pandoc@box{#1}%
  \Gscale@div\@tempa{\textheight}{\dimexpr\ht\pandoc@box+\dp\pandoc@box\relax}%
  \Gscale@div\@tempb{\linewidth}{\wd\pandoc@box}%
  \ifdim\@tempb\p@<\@tempa\p@\let\@tempa\@tempb\fi% select the smaller of both
  \ifdim\@tempa\p@<\p@\scalebox{\@tempa}{\usebox\pandoc@box}%
  \else\usebox{\pandoc@box}%
  \fi%
}
% Set default figure placement to htbp
\def\fps@figure{htbp}
\makeatother





\setlength{\emergencystretch}{3em} % prevent overfull lines

\providecommand{\tightlist}{%
  \setlength{\itemsep}{0pt}\setlength{\parskip}{0pt}}



 


\usepackage{makeidx}
\makeindex
\makeatletter
\@ifpackageloaded{tcolorbox}{}{\usepackage[skins,breakable]{tcolorbox}}
\@ifpackageloaded{fontawesome5}{}{\usepackage{fontawesome5}}
\definecolor{quarto-callout-color}{HTML}{909090}
\definecolor{quarto-callout-note-color}{HTML}{0758E5}
\definecolor{quarto-callout-important-color}{HTML}{CC1914}
\definecolor{quarto-callout-warning-color}{HTML}{EB9113}
\definecolor{quarto-callout-tip-color}{HTML}{00A047}
\definecolor{quarto-callout-caution-color}{HTML}{FC5300}
\definecolor{quarto-callout-color-frame}{HTML}{acacac}
\definecolor{quarto-callout-note-color-frame}{HTML}{4582ec}
\definecolor{quarto-callout-important-color-frame}{HTML}{d9534f}
\definecolor{quarto-callout-warning-color-frame}{HTML}{f0ad4e}
\definecolor{quarto-callout-tip-color-frame}{HTML}{02b875}
\definecolor{quarto-callout-caution-color-frame}{HTML}{fd7e14}
\makeatother
\makeatletter
\@ifpackageloaded{bookmark}{}{\usepackage{bookmark}}
\makeatother
\makeatletter
\@ifpackageloaded{caption}{}{\usepackage{caption}}
\AtBeginDocument{%
\ifdefined\contentsname
  \renewcommand*\contentsname{Table of contents}
\else
  \newcommand\contentsname{Table of contents}
\fi
\ifdefined\listfigurename
  \renewcommand*\listfigurename{List of Figures}
\else
  \newcommand\listfigurename{List of Figures}
\fi
\ifdefined\listtablename
  \renewcommand*\listtablename{List of Tables}
\else
  \newcommand\listtablename{List of Tables}
\fi
\ifdefined\figurename
  \renewcommand*\figurename{Figure}
\else
  \newcommand\figurename{Figure}
\fi
\ifdefined\tablename
  \renewcommand*\tablename{Table}
\else
  \newcommand\tablename{Table}
\fi
}
\@ifpackageloaded{float}{}{\usepackage{float}}
\floatstyle{ruled}
\@ifundefined{c@chapter}{\newfloat{codelisting}{h}{lop}}{\newfloat{codelisting}{h}{lop}[chapter]}
\floatname{codelisting}{Listing}
\newcommand*\listoflistings{\listof{codelisting}{List of Listings}}
\usepackage{amsthm}
\theoremstyle{definition}
\newtheorem{example}{Example}[chapter]
\theoremstyle{remark}
\AtBeginDocument{\renewcommand*{\proofname}{Proof}}
\newtheorem*{remark}{Remark}
\newtheorem*{solution}{Solution}
\newtheorem{refremark}{Remark}[chapter]
\newtheorem{refsolution}{Solution}[chapter]
\makeatother
\makeatletter
\makeatother
\makeatletter
\@ifpackageloaded{caption}{}{\usepackage{caption}}
\@ifpackageloaded{subcaption}{}{\usepackage{subcaption}}
\makeatother
\usepackage{bookmark}
\IfFileExists{xurl.sty}{\usepackage{xurl}}{} % add URL line breaks if available
\urlstyle{same}
\hypersetup{
  pdftitle={Tutorial},
  hidelinks,
  pdfcreator={LaTeX via pandoc}}


\title{Tutorial}
\usepackage{etoolbox}
\makeatletter
\providecommand{\subtitle}[1]{% add subtitle to \maketitle
  \apptocmd{\@title}{\par {\large #1 \par}}{}{}
}
\makeatother
\subtitle{A Concise Introduction to Computational Tools}
\author{}
\date{}
\begin{document}
\frontmatter
\maketitle

\clearpage
\vfill
\begin{center}
{\small
This work is licensed under a Creative Commons Attribution-ShareAlike 4.0 International License (CC BY-SA 4.0).\\[0.5em]
Full text available at \href{https://creativecommons.org/licenses/by-sa/4.0/}{https://creativecommons.org/licenses/by-sa/4.0/}.
}
\end{center}
\vfill
\clearpage

\renewcommand*\contentsname{Contents}
{
\setcounter{tocdepth}{2}
\tableofcontents
}
\listoffigures
\listoftables

\mainmatter
\bookmarksetup{startatroot}

\chapter*{Preface}\label{preface}
\addcontentsline{toc}{chapter}{Preface}

\markboth{Preface}{Preface}

A Concise Introduction to Computational Tools provides a series of
tutorials derived from lecture notes, offering clear and focused
guidance on essential topics. No prior programming experience is
required.

Students are expected to be comfortable with college-level mathematics
and science and are strongly encouraged to consult reliable reference
texts. The tutorials also assume working familiarity with either macOS
or Microsoft Windows.

For best results, these materials should be used on a laptop or desktop
computer rather than a tablet or smartphone.

\bookmarksetup{startatroot}

\chapter*{Study Guide}\label{study-guide}
\addcontentsline{toc}{chapter}{Study Guide}

\markboth{Study Guide}{Study Guide}

\begin{itemize}
\item
  First and foremost, solve the problem sets using only pen, paper, and
  a calculator. After completing a problem by hand, slightly modify the
  conditions or variables and solve it again. Then, input the new values
  into the Python script and compare the results with your
  hand-calculated solution.
\item
  Buddy system: Study with a classmate. Teaching and helping one another
  significantly deepens understanding of the material. Students are
  strongly encouraged to work through problem sets collaboratively. For
  example, one student solves the odd-numbered problems while the other
  solves the even-numbered ones; then each explains their solutions to
  the other.
\item
  Practice, practice, practice: As the saying goes, ``practice makes
  perfect.'' More accurately in this context, ``good practice makes
  perfect.'' That means pen-and-paper-first practice. Use Python as a
  verification tool, not a crutch. Always derive the solution
  analytically on paper first. Only after obtaining a complete
  hand-calculated result should you run the corresponding Python script.
  This approach echoes a proverb commonly attributed to Confucius:

  \begin{quote}
  I hear and I forget

  I see and I remember

  I do and I understand
  \end{quote}
\item
  Maintain a solution notebook: For each major problem, keep a single
  document (physical or digital) containing:

  \begin{itemize}
  \tightlist
  \item
    The full hand derivation\\
  \item
    Key numerical results with units\\
  \item
    A brief paragraph summarizing insights or reconciling any
    intermediate discrepancies
  \end{itemize}

  This record becomes an invaluable review resource before exams.
\end{itemize}

\bookmarksetup{startatroot}

\chapter*{Python Tutorial}\label{python-tutorial}
\addcontentsline{toc}{chapter}{Python Tutorial}

\markboth{Python Tutorial}{Python Tutorial}

This tutorial introduces Python programming, covering basic concepts
with examples to illustrate key points. We will start by using Python as
a calculator, then explore variables, functions, and control flow.

\section*{Requirements}\label{requirements}
\addcontentsline{toc}{section}{Requirements}

\markright{Requirements}

To follow this tutorial, the easiest way to get started is by using a
web-based Python environment. This lets you write and run Python code
right in your browser; no downloads or setup needed. I recommend
\href{https://python-fiddle.com/}{python-fiddle.com}, an easy-to-use
online editor that lets you experiment with Python instantly and solve
your problem sets effortlessly.

If you prefer working on your own computer, make sure you have
\href{https://www.python.org/downloads/}{Python} (version 3.10 or later)
installed. Python works on Windows, macOS, and Linux. You'll also need a
text editor or an Integrated Development Environment (IDE) to write your
code. I recommend \href{https://positron.posit.co/}{Positron}, a
beginner-friendly IDE with a built-in terminal, though other editors
like VS Code or PyCharm are also good options.

\section*{Basic Syntax}\label{basic-syntax}
\addcontentsline{toc}{section}{Basic Syntax}

\markright{Basic Syntax}

Python uses indentation\index{indentation} (typically four spaces) to
define code blocks. A colon\index{colon} (\texttt{:}) introduces a
block, and statements within the block must be indented consistently.
Python is case-sensitive, so \texttt{Variable} and \texttt{variable} are
distinct identifiers. \emph{Statements typically end with a newline, but
you can use a backslash (\texttt{\textbackslash{}}) to continue a
statement across multiple lines.}

\begin{Shaded}
\begin{Highlighting}[]
\NormalTok{total }\OperatorTok{=} \DecValTok{1} \OperatorTok{+} \DecValTok{2} \OperatorTok{+} \DecValTok{3} \OperatorTok{+} \DecValTok{4} \OperatorTok{+} \DecValTok{5}
\BuiltInTok{print}\NormalTok{(total)  }\CommentTok{\# Output: 15}
\end{Highlighting}
\end{Shaded}

Basic syntax rules:

\begin{itemize}
\tightlist
\item
  Comments start with \texttt{\#} and extend to the end of the line.
\item
  Strings can be enclosed in single quotes
  (\texttt{\textquotesingle{}}), double quotes (\texttt{"}), or triple
  quotes
  (\texttt{\textquotesingle{}\textquotesingle{}\textquotesingle{}} or
  \texttt{"""}) for multi-line strings.
\item
  Python is case-sensitive, so \texttt{Variable} and \texttt{variable}
  are different identifiers.
\end{itemize}

\section*{\texorpdfstring{The \texttt{print()}
Function}{The print() Function}}\label{the-print-function}
\addcontentsline{toc}{section}{The \texttt{print()} Function}

\markright{The \texttt{print()} Function}

The \texttt{print()} function\index{print} displays output in Python.

\begin{Shaded}
\begin{Highlighting}[]
\NormalTok{name }\OperatorTok{=} \StringTok{"Rudolf Diesel"}
\NormalTok{year }\OperatorTok{=} \DecValTok{1858}
\BuiltInTok{print}\NormalTok{(}\SpecialStringTok{f"}\SpecialCharTok{\{}\NormalTok{name}\SpecialCharTok{\}}\SpecialStringTok{ was born in }\SpecialCharTok{\{}\NormalTok{year}\SpecialCharTok{\}}\SpecialStringTok{."}\NormalTok{)}
\end{Highlighting}
\end{Shaded}

Output: \texttt{Rudolf\ Diesel\ was\ born\ in\ 1858.}

\section*{\texorpdfstring{Formatting in
\texttt{print()}}{Formatting in print()}}\label{formatting-in-print}
\addcontentsline{toc}{section}{Formatting in \texttt{print()}}

\markright{Formatting in \texttt{print()}}

The following table illustrates common f-string formatting options for
the \texttt{print()} function:

\begin{longtable}[]{@{}
  >{\raggedright\arraybackslash}p{(\linewidth - 6\tabcolsep) * \real{0.2466}}
  >{\raggedright\arraybackslash}p{(\linewidth - 6\tabcolsep) * \real{0.2466}}
  >{\raggedright\arraybackslash}p{(\linewidth - 6\tabcolsep) * \real{0.2603}}
  >{\raggedright\arraybackslash}p{(\linewidth - 6\tabcolsep) * \real{0.2466}}@{}}
\toprule\noalign{}
\begin{minipage}[b]{\linewidth}\raggedright
Format
\end{minipage} & \begin{minipage}[b]{\linewidth}\raggedright
Code
\end{minipage} & \begin{minipage}[b]{\linewidth}\raggedright
Example
\end{minipage} & \begin{minipage}[b]{\linewidth}\raggedright
Output
\end{minipage} \\
\midrule\noalign{}
\endhead
\bottomrule\noalign{}
\endlastfoot
\textbf{Round to 2 decimals} & \texttt{f"\{x:.2f\}"} &
\texttt{print(f"\{3.14159:.2f\}")} & \texttt{3.14} \\
\textbf{Round to whole number} & \texttt{f"\{x:.0f\}"} &
\texttt{print(f"\{3.9:.0f\}")} & \texttt{4} \\
\textbf{Thousands separator} & \texttt{f"\{x:,.2f\}"} &
\texttt{print(f"\{1234567.89:,.2f\}")} & \texttt{1,234,567.89} \\
\textbf{Percentage} & \texttt{f"\{x:.1\%\}"} &
\texttt{print(f"\{0.756:.1\%\}")} & \texttt{75.6\%} \\
\textbf{Currency (CDN)} & \texttt{f"\$\{x:,.2f\}"} &
\texttt{print(f"\$\{1234.5:,.2f\}")} & \texttt{\$1,234.50} \\
\textbf{Currency (EUR)} & \texttt{f"€\{x:,.2f\}"} &
\texttt{print(f"€\{1234.5:,.2f\}")} & \texttt{€1,234.50} \\
\textbf{Currency (JPY)} & \texttt{f"¥\{x:,.0f\}"} &
\texttt{print(f"¥\{1234567:,.0f\}")} & \texttt{¥1,234,567} \\
\end{longtable}

\section*{Variables and Data Types}\label{variables-and-data-types}
\addcontentsline{toc}{section}{Variables and Data Types}

\markright{Variables and Data Types}

Variables\index{variables} store data and are assigned values using the
\texttt{=} operator.

\begin{Shaded}
\begin{Highlighting}[]
\NormalTok{x }\OperatorTok{=} \DecValTok{10}
\NormalTok{y }\OperatorTok{=} \FloatTok{3.14}
\NormalTok{name }\OperatorTok{=} \StringTok{"Rudolph"}
\end{Highlighting}
\end{Shaded}

Python has several built-in data types, including:

\begin{itemize}
\tightlist
\item
  Integers (\texttt{int}): Whole numbers, e.g., \texttt{10}, \texttt{-5}
\item
  Floating-point numbers (\texttt{float}): Decimal numbers, e.g.,
  \texttt{3.14}, \texttt{-0.001}
\item
  Strings (\texttt{str}): Text, e.g., \texttt{"Hello"},
  \texttt{\textquotesingle{}World\textquotesingle{}}
\item
  Booleans (\texttt{bool}): \texttt{True} or \texttt{False}
\end{itemize}

\subsection*{Arithmetic Operations}\label{arithmetic-operations}
\addcontentsline{toc}{subsection}{Arithmetic Operations}

\begin{Shaded}
\begin{Highlighting}[]
\NormalTok{a }\OperatorTok{=} \DecValTok{10}
\NormalTok{b }\OperatorTok{=} \DecValTok{3}
\BuiltInTok{print}\NormalTok{(a }\OperatorTok{+}\NormalTok{ b)  }\CommentTok{\# Addition: 13}
\BuiltInTok{print}\NormalTok{(a }\OperatorTok{{-}}\NormalTok{ b)  }\CommentTok{\# Subtraction: 7}
\BuiltInTok{print}\NormalTok{(a }\OperatorTok{*}\NormalTok{ b)  }\CommentTok{\# Multiplication: 30}
\BuiltInTok{print}\NormalTok{(a }\OperatorTok{/}\NormalTok{ b)  }\CommentTok{\# Division: 3.3333...}
\BuiltInTok{print}\NormalTok{(a }\OperatorTok{//}\NormalTok{ b) }\CommentTok{\# Integer Division: 3}
\BuiltInTok{print}\NormalTok{(a }\OperatorTok{**}\NormalTok{ b) }\CommentTok{\# Exponentiation: 1000}
\end{Highlighting}
\end{Shaded}

\subsection*{String Operations}\label{string-operations}
\addcontentsline{toc}{subsection}{String Operations}

\begin{Shaded}
\begin{Highlighting}[]
\NormalTok{first\_name }\OperatorTok{=} \StringTok{"Rudolph"}
\NormalTok{last\_name }\OperatorTok{=} \StringTok{"Diesel"}
\NormalTok{full\_name }\OperatorTok{=}\NormalTok{ first\_name }\OperatorTok{+} \StringTok{" "} \OperatorTok{+}\NormalTok{ last\_name  }\CommentTok{\# Concatenation using +}
\BuiltInTok{print}\NormalTok{(full\_name)  }\CommentTok{\# Output: Rudolph Diesel}
\BuiltInTok{print}\NormalTok{(}\SpecialStringTok{f"}\SpecialCharTok{\{}\NormalTok{first\_name}\SpecialCharTok{\}}\SpecialStringTok{ }\SpecialCharTok{\{}\NormalTok{last\_name}\SpecialCharTok{\}}\SpecialStringTok{"}\NormalTok{)  }\CommentTok{\# Concatenation using f{-}string}
\BuiltInTok{print}\NormalTok{(full\_name }\OperatorTok{*} \DecValTok{2}\NormalTok{)  }\CommentTok{\# Repetition: Rudolph DieselRudolph Diesel}
\BuiltInTok{print}\NormalTok{(full\_name.upper())  }\CommentTok{\# Uppercase: RUDOLPH DIESEL}
\end{Highlighting}
\end{Shaded}

Note: String repetition (\texttt{*}) concatenates the string multiple
times without spaces. For example, \texttt{full\_name\ *\ 2} produces
\texttt{Rudolph\ DieselRudolph\ Diesel}.

\section*{Python as a Calculator in Interactive
Mode}\label{python-as-a-calculator-in-interactive-mode}
\addcontentsline{toc}{section}{Python as a Calculator in Interactive
Mode}

\markright{Python as a Calculator in Interactive Mode}

Python's interactive mode allows you to enter commands and see results
immediately, ideal for quick calculations. To start, open a terminal (on
macOS, Linux, or Windows) and type:

\begin{Shaded}
\begin{Highlighting}[]
\ExtensionTok{python3}  \CommentTok{\# Use \textquotesingle{}python\textquotesingle{} on Windows if \textquotesingle{}python3\textquotesingle{} is not recognized}
\end{Highlighting}
\end{Shaded}

You should see the Python prompt:

\begin{Shaded}
\begin{Highlighting}[]
\OperatorTok{\textgreater{}\textgreater{}\textgreater{}}
\end{Highlighting}
\end{Shaded}

Enter expressions and press \textbf{Enter} to see results:

\begin{Shaded}
\begin{Highlighting}[]
\DecValTok{2} \OperatorTok{+} \DecValTok{3}  \CommentTok{\# Output: 5}
\DecValTok{7} \OperatorTok{{-}} \DecValTok{4}  \CommentTok{\# Output: 3}
\DecValTok{6} \OperatorTok{*} \DecValTok{9}  \CommentTok{\# Output: 54}
\DecValTok{8} \OperatorTok{/} \DecValTok{2}  \CommentTok{\# Output: 4.0}
\DecValTok{8} \OperatorTok{//} \DecValTok{2} \CommentTok{\# Output: 4}
\DecValTok{2} \OperatorTok{**} \DecValTok{3} \CommentTok{\# Output: 8}
\end{Highlighting}
\end{Shaded}

\subsection*{Parentheses for Grouping}\label{parentheses-for-grouping}
\addcontentsline{toc}{subsection}{Parentheses for Grouping}

\begin{Shaded}
\begin{Highlighting}[]
\NormalTok{(}\DecValTok{2} \OperatorTok{+} \DecValTok{3}\NormalTok{) }\OperatorTok{*} \DecValTok{4}  \CommentTok{\# Output: 20}
\DecValTok{2} \OperatorTok{+}\NormalTok{ (}\DecValTok{3} \OperatorTok{*} \DecValTok{4}\NormalTok{)  }\CommentTok{\# Output: 14}
\end{Highlighting}
\end{Shaded}

\subsection*{Variables}\label{variables}
\addcontentsline{toc}{subsection}{Variables}

\begin{Shaded}
\begin{Highlighting}[]
\NormalTok{x }\OperatorTok{=} \DecValTok{10}
\NormalTok{y }\OperatorTok{=} \DecValTok{3}
\NormalTok{x }\OperatorTok{/}\NormalTok{ y  }\CommentTok{\# Output: 3.3333333333333335}
\end{Highlighting}
\end{Shaded}

\subsection*{Exiting Interactive Mode}\label{exiting-interactive-mode}
\addcontentsline{toc}{subsection}{Exiting Interactive Mode}

To exit, type:

\begin{Shaded}
\begin{Highlighting}[]
\NormalTok{exit()}
\end{Highlighting}
\end{Shaded}

Alternatively, use: - \textbf{Ctrl+D} (macOS/Linux) - \textbf{Ctrl+Z}
then Enter (Windows)

\section*{Control Flow}\label{control-flow}
\addcontentsline{toc}{section}{Control Flow}

\markright{Control Flow}

Control flow\index{control flow} statements direct the execution of code
based on conditions.

\subsection*{Conditional Statements}\label{conditional-statements}
\addcontentsline{toc}{subsection}{Conditional Statements}

Conditional statements\index{conditional statements} allow you to
execute different code blocks based on specific conditions. Python
provides three keywords for this purpose:

\begin{itemize}
\tightlist
\item
  \textbf{\texttt{if}}: Evaluates a condition and executes its code
  block if the condition is \texttt{True}.
\item
  \textbf{\texttt{elif}}: Short for ``else if,'' it checks an additional
  condition if the preceding \texttt{if} or \texttt{elif} conditions are
  \texttt{False}. You can use multiple \texttt{elif} statements to test
  multiple conditions sequentially, and Python will execute the first
  \texttt{True} condition's block, skipping the rest.
\item
  \textbf{\texttt{else}}: Executes a code block if none of the preceding
  \texttt{if} or \texttt{elif} conditions are \texttt{True}. It serves
  as a fallback and does not require a condition.
\end{itemize}

The following example uses age to categorize a person as a Minor, Adult,
or Senior, demonstrating how \texttt{if}, \texttt{elif}, and
\texttt{else} work together.

\begin{Shaded}
\begin{Highlighting}[]
\CommentTok{\# Categorize a person based on their age}
\NormalTok{age }\OperatorTok{=} \DecValTok{19}
\ControlFlowTok{if}\NormalTok{ age }\OperatorTok{\textless{}} \DecValTok{18}\NormalTok{:}
    \BuiltInTok{print}\NormalTok{(}\StringTok{"Minor"}\NormalTok{)}
\ControlFlowTok{elif}\NormalTok{ age }\OperatorTok{\textless{}=} \DecValTok{64}\NormalTok{:}
    \BuiltInTok{print}\NormalTok{(}\StringTok{"Adult"}\NormalTok{)}
\ControlFlowTok{else}\NormalTok{:}
    \BuiltInTok{print}\NormalTok{(}\StringTok{"Senior"}\NormalTok{)}
\end{Highlighting}
\end{Shaded}

Output: \texttt{Adult}

\subsection*{For Loop}\label{for-loop}
\addcontentsline{toc}{subsection}{For Loop}

A \texttt{for} loop\index{for loop} iterates over a sequence (e.g., list
or string).

\begin{Shaded}
\begin{Highlighting}[]
\NormalTok{components }\OperatorTok{=}\NormalTok{ [}\StringTok{"piston"}\NormalTok{, }\StringTok{"liner"}\NormalTok{, }\StringTok{"connecting rod"}\NormalTok{]}
\ControlFlowTok{for}\NormalTok{ component }\KeywordTok{in}\NormalTok{ components:}
    \BuiltInTok{print}\NormalTok{(component)}
\end{Highlighting}
\end{Shaded}

Output:

\begin{verbatim}
piston
liner
connecting rod
\end{verbatim}

\subsection*{While Loop}\label{while-loop}
\addcontentsline{toc}{subsection}{While Loop}

A \texttt{while} loop\index{while loop} executes as long as a condition
is true. Ensure the condition eventually becomes false to avoid infinite
loops.

\begin{Shaded}
\begin{Highlighting}[]
\NormalTok{count }\OperatorTok{=} \DecValTok{0}
\ControlFlowTok{while}\NormalTok{ count }\OperatorTok{\textless{}=} \DecValTok{5}\NormalTok{:}
    \BuiltInTok{print}\NormalTok{(count)}
\NormalTok{    count }\OperatorTok{+=} \DecValTok{1}
\end{Highlighting}
\end{Shaded}

Output:

\begin{verbatim}
0
1
2
3
4
5
\end{verbatim}

\section*{Functions}\label{functions}
\addcontentsline{toc}{section}{Functions}

\markright{Functions}

\subsection*{\texorpdfstring{The \texttt{def}
Keyword}{The def Keyword}}\label{the-def-keyword}
\addcontentsline{toc}{subsection}{The \texttt{def} Keyword}

Functions are reusable code blocks defined using the \texttt{def}
keyword\index{def keyword}. They can include default parameters for
optional arguments.

\begin{Shaded}
\begin{Highlighting}[]
\KeywordTok{def}\NormalTok{ add(a, b}\OperatorTok{=}\DecValTok{0}\NormalTok{):}
    \ControlFlowTok{return}\NormalTok{ a }\OperatorTok{+}\NormalTok{ b}
\BuiltInTok{print}\NormalTok{(add(}\DecValTok{5}\NormalTok{))      }\CommentTok{\# Output: 5}
\BuiltInTok{print}\NormalTok{(add(}\DecValTok{5}\NormalTok{, }\DecValTok{3}\NormalTok{))   }\CommentTok{\# Output: 8}

\KeywordTok{def}\NormalTok{ multiply(}\OperatorTok{*}\NormalTok{args):}
\NormalTok{    result }\OperatorTok{=} \DecValTok{1}
    \ControlFlowTok{for}\NormalTok{ num }\KeywordTok{in}\NormalTok{ args:}
\NormalTok{        result }\OperatorTok{*=}\NormalTok{ num}
    \ControlFlowTok{return}\NormalTok{ result}
\BuiltInTok{print}\NormalTok{(multiply(}\DecValTok{2}\NormalTok{, }\DecValTok{3}\NormalTok{, }\DecValTok{4}\NormalTok{))  }\CommentTok{\# Output: 24}
\end{Highlighting}
\end{Shaded}

\subsection*{\texorpdfstring{The \texttt{lambda}
Keyword}{The lambda Keyword}}\label{the-lambda-keyword}
\addcontentsline{toc}{subsection}{The \texttt{lambda} Keyword}

The \texttt{lambda} keyword\index{lambda keyword} creates anonymous
functions for short, one-off operations, often used in functional
programming.

\begin{Shaded}
\begin{Highlighting}[]
\NormalTok{celsius\_to\_fahrenheit }\OperatorTok{=} \KeywordTok{lambda}\NormalTok{ c: (c }\OperatorTok{*} \DecValTok{9} \OperatorTok{/} \DecValTok{5}\NormalTok{) }\OperatorTok{+} \DecValTok{32}
\BuiltInTok{print}\NormalTok{(celsius\_to\_fahrenheit(}\DecValTok{25}\NormalTok{))  }\CommentTok{\# Output: 77.0}
\end{Highlighting}
\end{Shaded}

\section*{\texorpdfstring{The \texttt{math}
Module}{The math Module}}\label{the-math-module}
\addcontentsline{toc}{section}{The \texttt{math} Module}

\markright{The \texttt{math} Module}

The \texttt{math} module\index{math module} provides mathematical
functions and constants.

\begin{Shaded}
\begin{Highlighting}[]
\ImportTok{import}\NormalTok{ math}
\BuiltInTok{print}\NormalTok{(math.sqrt(}\DecValTok{16}\NormalTok{))  }\CommentTok{\# Output: 4.0}
\BuiltInTok{print}\NormalTok{(math.pi)        }\CommentTok{\# Output: 3.141592653589793}
\end{Highlighting}
\end{Shaded}

\begin{Shaded}
\begin{Highlighting}[]
\ImportTok{import}\NormalTok{ math}
\NormalTok{angle }\OperatorTok{=}\NormalTok{ math.pi }\OperatorTok{/} \DecValTok{4}  \CommentTok{\# 45 degrees in radians}
\BuiltInTok{print}\NormalTok{(math.sin(angle))  }\CommentTok{\# Output: 0.7071067811865475 (approximately √2/2)}
\BuiltInTok{print}\NormalTok{(math.cos(angle))  }\CommentTok{\# Output: 0.7071067811865476 (approximately √2/2)}
\BuiltInTok{print}\NormalTok{(math.tan(angle))  }\CommentTok{\# Output: 1.0}
\end{Highlighting}
\end{Shaded}

Note: Floating-point arithmetic may result in small precision
differences, as seen in the \texttt{sin} and \texttt{cos} outputs.

\begin{Shaded}
\begin{Highlighting}[]
\ImportTok{import}\NormalTok{ math}
\BuiltInTok{print}\NormalTok{(math.log(}\DecValTok{10}\NormalTok{))       }\CommentTok{\# Natural logarithm of 10: 2.302585092994046}
\BuiltInTok{print}\NormalTok{(math.log(}\DecValTok{100}\NormalTok{, }\DecValTok{10}\NormalTok{))  }\CommentTok{\# Logarithm of 100 with base 10: 2.0}
\end{Highlighting}
\end{Shaded}

\subsection*{Converting Between Radians and
Degrees}\label{converting-between-radians-and-degrees}
\addcontentsline{toc}{subsection}{Converting Between Radians and
Degrees}

The \texttt{math} module provides \texttt{math.radians()} to convert
degrees to radians and \texttt{math.degrees()} to convert radians to
degrees, which is useful for trigonometric calculations.

\begin{tcolorbox}[enhanced jigsaw, toprule=.15mm, leftrule=.75mm, colbacktitle=quarto-callout-warning-color!10!white, bottomrule=.15mm, bottomtitle=1mm, toptitle=1mm, colback=white, colframe=quarto-callout-warning-color-frame, opacitybacktitle=0.6, coltitle=black, opacityback=0, breakable, titlerule=0mm, title=\textcolor{quarto-callout-warning-color}{\faExclamationTriangle}\hspace{0.5em}{Warning}, left=2mm, rightrule=.15mm, arc=.35mm]

In Python, the math module's trigonometric functions (sin, cos, tan,
etc.) expect angles in radians, not degrees. Always convert degrees to
radians before using sin, cos, or tan in Python's math module.

\end{tcolorbox}

\begin{Shaded}
\begin{Highlighting}[]
\ImportTok{import}\NormalTok{ math}
\NormalTok{degrees }\OperatorTok{=} \DecValTok{180}
\NormalTok{radians }\OperatorTok{=}\NormalTok{ math.radians(degrees)}
\BuiltInTok{print}\NormalTok{(}\SpecialStringTok{f"}\SpecialCharTok{\{}\NormalTok{degrees}\SpecialCharTok{\}}\SpecialStringTok{ degrees is }\SpecialCharTok{\{}\NormalTok{radians}\SpecialCharTok{:.3f\}}\SpecialStringTok{ radians"}\NormalTok{)  }\CommentTok{\# Output: 180 degrees is 3.142 radians}

\NormalTok{radians }\OperatorTok{=}\NormalTok{ math.pi }\OperatorTok{/} \DecValTok{2}
\NormalTok{degrees }\OperatorTok{=}\NormalTok{ math.degrees(radians)}
\BuiltInTok{print}\NormalTok{(}\SpecialStringTok{f"}\SpecialCharTok{\{}\NormalTok{radians}\SpecialCharTok{:.3f\}}\SpecialStringTok{ radians is }\SpecialCharTok{\{}\NormalTok{degrees}\SpecialCharTok{:.1f\}}\SpecialStringTok{ degrees"}\NormalTok{)  }\CommentTok{\# Output: 1.571 radians is 90.0 degrees}
\end{Highlighting}
\end{Shaded}

\section*{Writing Python Scripts}\label{writing-python-scripts}
\addcontentsline{toc}{section}{Writing Python Scripts}

\markright{Writing Python Scripts}

Write Python code in a \texttt{.py} file and run it as a script. Create
a file named \texttt{script.py}:

\begin{Shaded}
\begin{Highlighting}[]
\CommentTok{\# script.py}
\ImportTok{import}\NormalTok{ math}
\BuiltInTok{print}\NormalTok{(}\StringTok{"Square root of 16 is:"}\NormalTok{, math.sqrt(}\DecValTok{16}\NormalTok{))}
\BuiltInTok{print}\NormalTok{(}\StringTok{"Value of pi is:"}\NormalTok{, math.pi)}
\BuiltInTok{print}\NormalTok{(}\StringTok{"Sine of 90 degrees is:"}\NormalTok{, math.sin(math.pi }\OperatorTok{/} \DecValTok{2}\NormalTok{))}
\BuiltInTok{print}\NormalTok{(}\StringTok{"Natural logarithm of 10 is:"}\NormalTok{, math.log(}\DecValTok{10}\NormalTok{))}
\BuiltInTok{print}\NormalTok{(}\StringTok{"Logarithm of 100 with base 10 is:"}\NormalTok{, math.log(}\DecValTok{100}\NormalTok{, }\DecValTok{10}\NormalTok{))}
\end{Highlighting}
\end{Shaded}

To run the script, open a terminal, navigate to the directory containing
\texttt{script.py} using the \texttt{cd} command (e.g.,
\texttt{cd\ /path/to/directory}), and type:

\begin{Shaded}
\begin{Highlighting}[]
\ExtensionTok{python3}\NormalTok{ script.py  }\CommentTok{\# or python script.py on Windows}
\end{Highlighting}
\end{Shaded}

Output:

\begin{verbatim}
Square root of 16 is: 4.0
Value of pi is: 3.141592653589793
Sine of 90 degrees is: 1.0
Natural logarithm of 10 is: 2.302585092994046
Logarithm of 100 with base 10 is: 2.0
\end{verbatim}

\section*{Summary}\label{summary}
\addcontentsline{toc}{section}{Summary}

\markright{Summary}

You've now learned Python basics: syntax, variables, control flow,
functions, and the math module. Practice all examples instantly at
\href{https://python-fiddle.com/}{python-fiddle.com}.

\bookmarksetup{startatroot}

\chapter{Statics}\label{statics}

Statics is the branch of mechanics that analyzes bodies in equilibrium
under the action of forces. This chapter examines cases where forces are
in balance and presents methods for calculating resultant forces and
their associated moments.

\section{Space Diagrams}\label{space-diagrams}

A space diagram illustrates the geometry of the system and the applied
forces.

\begin{figure}

\centering{

\includegraphics[width=0.75\linewidth,height=\textheight,keepaspectratio]{images/spacedia.png}

}

\caption{\label{fig-spacedia}Space Diagram}

\end{figure}%

\section{Vector Diagrams}\label{vector-diagrams}

A vector diagram (or force polygon) is a \textbf{scaled} graphical
construction in which force vectors are placed head-to-tail in
succession. It is used to determine the resultant of a system of forces
through vector addition, as well as to resolve equilibrium problems when
the polygon closes.

\begin{figure}

\centering{

\pandocbounded{\includegraphics[keepaspectratio]{images/vectordia.png}}

}

\caption{\label{fig-vectordia}Vector Diagram}

\end{figure}%

\subsection{Resultant}\label{resultant}

The resultant is a single force that has exactly the same effect on an
object as all the original forces acting together. It is found by adding
the forces vectorially taking into account both their magnitude and
direction. The resultant gives the overall direction and magnitude of
the combined forces as shown in Figure~\ref{fig-vectordia}.

\subsection{Equilibrium}\label{equilibrium}

Equilibrium of an object occurs when all the forces acting on it are
balanced, so the object remains at rest or moves at a constant speed in
a straight line. In equilibrium, as shown in
Figure~\ref{fig-equilibrant}, there is no net force or acceleration,
meaning the object is in a stable state without any change in its
motion.

\begin{figure}

\centering{

\pandocbounded{\includegraphics[keepaspectratio]{images/vectordia-equilibrant.png}}

}

\caption{\label{fig-equilibrant}Equilibrant}

\end{figure}%

\section{Conditions of Equilibrium}\label{conditions-of-equilibrium}

\begin{enumerate}
\def\labelenumi{\arabic{enumi}.}
\tightlist
\item
  Net force must be zero:
\end{enumerate}

\begin{equation}\phantomsection\label{eq-force}{
\sum_{k} \vec{F}_{k} = \vec{0}
}\end{equation}

\begin{enumerate}
\def\labelenumi{\arabic{enumi}.}
\setcounter{enumi}{1}
\tightlist
\item
  Net torque must be zero:
\end{enumerate}

\begin{equation}\phantomsection\label{eq-torque}{
\sum_{k} \vec{\tau}_{k} = \vec{0}
}\end{equation}

\section{Cosine Rule}\label{cosine-rule}

\begin{figure}

\centering{

\pandocbounded{\includegraphics[keepaspectratio]{images/sinecosrule.png}}

}

\caption{\label{fig-law_cos_sine}Rules of Cosine and Sine}

\end{figure}%

The Cosine Rule\index{Cosine rule} is used to relate the lengths of the
sides of a triangle as shown in Figure~\ref{fig-law_cos_sine} to the
cosine of one of its angles:

\begin{equation}\phantomsection\label{eq-cos}{
c^2 = a^2 + b^2 - 2ab\cos \gamma
}\end{equation}

Where:

\begin{itemize}
\item
  a, b, c are the sides of the triangle.
\item
  \(\gamma\) is the angle opposite side c.
\end{itemize}

\section{Sine Rule}\label{sine-rule}

The Sine Rule\index{Sine rule} relates the sides and angles of a
triangle in Figure~\ref{fig-law_cos_sine}:

\begin{equation}\phantomsection\label{eq-sine}{
\frac{a}{\sin \alpha} = \frac{b}{\sin \beta} = \frac{c}{\sin \gamma}
}\end{equation}

Where:

\begin{itemize}
\item
  \(\alpha\), \(\beta\), \(\gamma\) are the angles of the triangle.
\item
  a, b, c are the sides of the triangle opposite to angles \(\alpha\),
  \(\beta\), \(\gamma\) respectively.
\end{itemize}

\section{Problem Set}\label{problem-set}

\begin{example}[]\protect\hypertarget{exm-1_6}{}\label{exm-1_6}

Two slings of equal length are slung from a horizontal beam and
connected to a ring at their lower ends, the slings and beam forming an
equilateral triangle. Find the force in each sling when a load of 30 kN
hangs from the ring.

\begin{tcolorbox}[enhanced jigsaw, toprule=.15mm, leftrule=.75mm, colbacktitle=quarto-callout-note-color!10!white, bottomrule=.15mm, bottomtitle=1mm, toptitle=1mm, colback=white, colframe=quarto-callout-note-color-frame, opacitybacktitle=0.6, coltitle=black, opacityback=0, breakable, titlerule=0mm, title=\textcolor{quarto-callout-note-color}{\faInfo}\hspace{0.5em}{Note}, left=2mm, rightrule=.15mm, arc=.35mm]

In Python, the math module's trigonometric functions (sin, cos, tan,
etc.) expect angles in radians, not degrees. Always convert degrees to
radians before using sin, cos, or tan in Python's math module.

\end{tcolorbox}

\begin{Shaded}
\begin{Highlighting}[]
\ImportTok{import}\NormalTok{ math}

\CommentTok{\# Given}
\NormalTok{load }\OperatorTok{=} \DecValTok{30}  \CommentTok{\# kN}
\NormalTok{angle\_deg }\OperatorTok{=} \DecValTok{60}  \CommentTok{\# each sling makes 60° with the horizontal}

\CommentTok{\# Convert angle to radians}
\NormalTok{angle\_rad }\OperatorTok{=}\NormalTok{ math.radians(angle\_deg)}

\CommentTok{\# Vertical equilibrium: 2 * T * sin(angle) = load}
\NormalTok{T }\OperatorTok{=}\NormalTok{ load }\OperatorTok{/}\NormalTok{ (}\DecValTok{2} \OperatorTok{*}\NormalTok{ math.sin(angle\_rad))}

\BuiltInTok{print}\NormalTok{(}\SpecialStringTok{f"Tension in each sling: }\SpecialCharTok{\{}\NormalTok{T}\SpecialCharTok{:.4f\}}\SpecialStringTok{ kN"}\NormalTok{)}
\end{Highlighting}
\end{Shaded}

\end{example}

\begin{example}[]\protect\hypertarget{exm-isosceles}{}\label{exm-isosceles}

Two identical slings of equal length are attached to a horizontal beam
and meet at a ring from which a 45 kN load is suspended. The slings and
the beam form an isosceles triangle in which each sling makes an angle
of 50° with the horizontal. Find the force (tension) in each sling.

\begin{Shaded}
\begin{Highlighting}[]
\ImportTok{import}\NormalTok{ math}

\CommentTok{\# Given}
\NormalTok{load }\OperatorTok{=} \DecValTok{45}      \CommentTok{\# kN}
\NormalTok{angle\_deg }\OperatorTok{=} \DecValTok{50} \CommentTok{\# degrees}

\CommentTok{\# Convert angle to radians}
\NormalTok{angle\_rad }\OperatorTok{=}\NormalTok{ math.radians(angle\_deg)}

\CommentTok{\# Compute tension}
\NormalTok{T }\OperatorTok{=}\NormalTok{ load }\OperatorTok{/}\NormalTok{ (}\DecValTok{2} \OperatorTok{*}\NormalTok{ math.sin(angle\_rad))}

\BuiltInTok{print}\NormalTok{(}\SpecialStringTok{f"Tension in each sling: }\SpecialCharTok{\{}\NormalTok{T}\SpecialCharTok{:.4f\}}\SpecialStringTok{ kN"}\NormalTok{)}
\end{Highlighting}
\end{Shaded}

\end{example}

\begin{example}[]\protect\hypertarget{exm-1_7}{}\label{exm-1_7}

Two lifting ropes are connected at their lower ends to a common shackle
from which a load of 25 kN hangs. If the ropes make angles of 32° and
42° respectively to the vertical, find the tension in each rope.

\begin{Shaded}
\begin{Highlighting}[]
\ImportTok{import}\NormalTok{ math}

\CommentTok{\# Given}
\NormalTok{load }\OperatorTok{=} \DecValTok{25}        \CommentTok{\# kN}
\NormalTok{theta1 }\OperatorTok{=} \DecValTok{32}      \CommentTok{\# angle opposite T2}
\NormalTok{theta2 }\OperatorTok{=} \DecValTok{42}      \CommentTok{\# angle opposite T1}
\NormalTok{theta3 }\OperatorTok{=} \DecValTok{180} \OperatorTok{{-}}\NormalTok{ (theta1 }\OperatorTok{+}\NormalTok{ theta2)  }\CommentTok{\# angle opposite load}

\CommentTok{\# Convert to radians}

\NormalTok{t1 }\OperatorTok{=}\NormalTok{ math.radians(theta1)}
\NormalTok{t2 }\OperatorTok{=}\NormalTok{ math.radians(theta2)}
\NormalTok{t3 }\OperatorTok{=}\NormalTok{ math.radians(theta3)}

\CommentTok{\# Law of sines}

\NormalTok{T1 }\OperatorTok{=}\NormalTok{ load }\OperatorTok{*}\NormalTok{ math.sin(t2) }\OperatorTok{/}\NormalTok{ math.sin(t3)  }\CommentTok{\# tension in rope 1}
\NormalTok{T2 }\OperatorTok{=}\NormalTok{ load }\OperatorTok{*}\NormalTok{ math.sin(t1) }\OperatorTok{/}\NormalTok{ math.sin(t3)  }\CommentTok{\# tension in rope 2}

\BuiltInTok{print}\NormalTok{(}\SpecialStringTok{f"Tension in rope 1: }\SpecialCharTok{\{}\NormalTok{T1}\SpecialCharTok{:.4f\}}\SpecialStringTok{ kN"}\NormalTok{)}
\BuiltInTok{print}\NormalTok{(}\SpecialStringTok{f"Tension in rope 2: }\SpecialCharTok{\{}\NormalTok{T2}\SpecialCharTok{:.4f\}}\SpecialStringTok{ kN"}\NormalTok{)}
\end{Highlighting}
\end{Shaded}

\end{example}

\begin{example}[]\protect\hypertarget{exm-1_10}{}\label{exm-1_10}

The angle between the jib and vertical post of a jib crane is 40°, and
the angle between the jib and tie is 45°. Find the force in the jib and
tie when a load of 15 kN hangs from the crane head.

\begin{Shaded}
\begin{Highlighting}[]
\ImportTok{import}\NormalTok{ math}

\CommentTok{\# Given}
\NormalTok{load }\OperatorTok{=} \DecValTok{15}  \CommentTok{\# kN}
\NormalTok{angle\_jib\_vertical }\OperatorTok{=} \DecValTok{40}  \CommentTok{\# degrees}
\NormalTok{angle\_jib\_tie }\OperatorTok{=} \DecValTok{45}       \CommentTok{\# degrees}

\CommentTok{\# Third angle in the force triangle}

\NormalTok{angle\_tie\_load }\OperatorTok{=} \DecValTok{180} \OperatorTok{{-}}\NormalTok{ angle\_jib\_vertical }\OperatorTok{{-}}\NormalTok{ angle\_jib\_tie  }\CommentTok{\# degrees}

\CommentTok{\# Convert angles to radians for math.sin}

\NormalTok{angle\_jib\_vertical\_rad }\OperatorTok{=}\NormalTok{ math.radians(angle\_jib\_vertical)}
\NormalTok{angle\_jib\_tie\_rad }\OperatorTok{=}\NormalTok{ math.radians(angle\_jib\_tie)}
\NormalTok{angle\_tie\_load\_rad }\OperatorTok{=}\NormalTok{ math.radians(angle\_tie\_load)}

\CommentTok{\# Law of sines}

\NormalTok{F\_jib }\OperatorTok{=}\NormalTok{ load }\OperatorTok{*}\NormalTok{ math.sin(angle\_tie\_load\_rad) }\OperatorTok{/}\NormalTok{ math.sin(angle\_jib\_tie\_rad)}
\NormalTok{F\_tie }\OperatorTok{=}\NormalTok{ load }\OperatorTok{*}\NormalTok{ math.sin(angle\_jib\_vertical\_rad) }\OperatorTok{/}\NormalTok{ math.sin(angle\_jib\_tie\_rad)}

\CommentTok{\# Print results}
\BuiltInTok{print}\NormalTok{(}\SpecialStringTok{f"Force in jib: }\SpecialCharTok{\{}\NormalTok{F\_jib}\SpecialCharTok{:.4f\}}\SpecialStringTok{ kN"}\NormalTok{)}
\BuiltInTok{print}\NormalTok{(}\SpecialStringTok{f"Force in tie: }\SpecialCharTok{\{}\NormalTok{F\_tie}\SpecialCharTok{:.4f\}}\SpecialStringTok{ kN"}\NormalTok{)}
\end{Highlighting}
\end{Shaded}

\end{example}

\begin{example}[]\protect\hypertarget{exm-1_11}{}\label{exm-1_11}

The lengths of the vertical post, jib, and tie of a jib crane are 8, 13,
and 9 m, respectively. Find the forces in the jib and tie when a load of
20 kN hangs from the crane head.

\begin{Shaded}
\begin{Highlighting}[]
\ImportTok{import}\NormalTok{ math}

\CommentTok{\# Given}
\NormalTok{P }\OperatorTok{=} \DecValTok{8}   \CommentTok{\# post m}
\NormalTok{J }\OperatorTok{=} \DecValTok{13}  \CommentTok{\# jib  m}
\NormalTok{T }\OperatorTok{=} \DecValTok{9}   \CommentTok{\# tie m}
\NormalTok{W }\OperatorTok{=} \DecValTok{20}  \CommentTok{\# load in kN}

\CommentTok{\# Law of cosines}

\CommentTok{\# Angle opposite P (theta\_p)}
\NormalTok{theta\_p }\OperatorTok{=}\NormalTok{ math.acos((J}\OperatorTok{**}\DecValTok{2} \OperatorTok{+}\NormalTok{ T}\OperatorTok{**}\DecValTok{2} \OperatorTok{{-}}\NormalTok{ P}\OperatorTok{**}\DecValTok{2}\NormalTok{) }\OperatorTok{/}\NormalTok{ (}\DecValTok{2} \OperatorTok{*}\NormalTok{ J }\OperatorTok{*}\NormalTok{ T))}

\CommentTok{\# Angle opposite T (theta\_t)}
\NormalTok{theta\_t }\OperatorTok{=}\NormalTok{ math.acos((J}\OperatorTok{**}\DecValTok{2} \OperatorTok{+}\NormalTok{ P}\OperatorTok{**}\DecValTok{2} \OperatorTok{{-}}\NormalTok{ T}\OperatorTok{**}\DecValTok{2}\NormalTok{) }\OperatorTok{/}\NormalTok{ (}\DecValTok{2} \OperatorTok{*}\NormalTok{ J }\OperatorTok{*}\NormalTok{ P))}

\CommentTok{\# Angle opposite J (theta\_j)}
\NormalTok{theta\_j }\OperatorTok{=}\NormalTok{ math.pi }\OperatorTok{{-}}\NormalTok{ theta\_p }\OperatorTok{{-}}\NormalTok{ theta\_t}

\CommentTok{\# Law of sines}

\NormalTok{F\_J }\OperatorTok{=}\NormalTok{ W }\OperatorTok{*}\NormalTok{ math.sin(theta\_j) }\OperatorTok{/}\NormalTok{ math.sin(theta\_p)  }\CommentTok{\# force in jib}
\NormalTok{F\_T }\OperatorTok{=}\NormalTok{ W }\OperatorTok{*}\NormalTok{ math.sin(theta\_t) }\OperatorTok{/}\NormalTok{ math.sin(theta\_p)  }\CommentTok{\# force in tie}

\CommentTok{\# Convert angles to degrees}

\NormalTok{theta\_p\_deg }\OperatorTok{=}\NormalTok{ math.degrees(theta\_p)}
\NormalTok{theta\_t\_deg }\OperatorTok{=}\NormalTok{ math.degrees(theta\_t)}
\NormalTok{theta\_j\_deg }\OperatorTok{=}\NormalTok{ math.degrees(theta\_j)}

\BuiltInTok{print}\NormalTok{(}\SpecialStringTok{f"theta\_p = }\SpecialCharTok{\{}\NormalTok{theta\_p\_deg}\SpecialCharTok{:.4f\}}\SpecialStringTok{ degrees"}\NormalTok{)}
\BuiltInTok{print}\NormalTok{(}\SpecialStringTok{f"theta\_t = }\SpecialCharTok{\{}\NormalTok{theta\_t\_deg}\SpecialCharTok{:.4f\}}\SpecialStringTok{ degrees"}\NormalTok{)}
\BuiltInTok{print}\NormalTok{(}\SpecialStringTok{f"theta\_j = }\SpecialCharTok{\{}\NormalTok{theta\_j\_deg}\SpecialCharTok{:.4f\}}\SpecialStringTok{ degrees"}\NormalTok{)}

\BuiltInTok{print}\NormalTok{()}
\BuiltInTok{print}\NormalTok{(}\SpecialStringTok{f"Force in jib (F\_J): }\SpecialCharTok{\{}\NormalTok{F\_J}\SpecialCharTok{:.4f\}}\SpecialStringTok{ kN"}\NormalTok{)}
\BuiltInTok{print}\NormalTok{(}\SpecialStringTok{f"Force in tie (F\_T): }\SpecialCharTok{\{}\NormalTok{F\_T}\SpecialCharTok{:.4f\}}\SpecialStringTok{ kN"}\NormalTok{)}
\end{Highlighting}
\end{Shaded}

\end{example}

\begin{example}[]\protect\hypertarget{exm-1_15}{}\label{exm-1_15}

A ship sailing due east at 18 knots runs into a 3-knot current moving
40° east of north. Find the resultant speed and direction of the ship.

\begin{tcolorbox}[enhanced jigsaw, toprule=.15mm, leftrule=.75mm, colbacktitle=quarto-callout-note-color!10!white, bottomrule=.15mm, bottomtitle=1mm, toptitle=1mm, colback=white, colframe=quarto-callout-note-color-frame, opacitybacktitle=0.6, coltitle=black, opacityback=0, breakable, titlerule=0mm, title=\textcolor{quarto-callout-note-color}{\faInfo}\hspace{0.5em}{Note}, left=2mm, rightrule=.15mm, arc=.35mm]

Degrees, Minutes, and Seconds (DMS) notation is widely used in
navigation and surveying because decimal degrees alone often lack
sufficient precision for small-scale positioning. On Earth's surface:

\begin{itemize}
\tightlist
\item
  1° of latitude ≈ 111 km\\
\item
  1′ (minute of arc) ≈ 1.85 km\\
\item
  1″ (second of arc) ≈ 30 m
\end{itemize}

Thus, a position given as 40° 26′ 46″ N instantly conveys location to
within tens of metres, whereas the equivalent decimal form, 40.44611°,
is less intuitive at a glance despite being mathematically identical.

\end{tcolorbox}

\begin{Shaded}
\begin{Highlighting}[]
\ImportTok{import}\NormalTok{ math}

\CommentTok{\# Given}
\NormalTok{ship }\OperatorTok{=} \DecValTok{18}          \CommentTok{\# knots (due east)}
\NormalTok{current }\OperatorTok{=} \DecValTok{3}        \CommentTok{\# knots}
\NormalTok{angle\_deg }\OperatorTok{=} \DecValTok{50}     \CommentTok{\# angle between ship and current in degrees}

\CommentTok{\# Convert to radians}

\NormalTok{angle\_rad }\OperatorTok{=}\NormalTok{ math.radians(angle\_deg)}

\CommentTok{\# Law of cosines}

\NormalTok{R }\OperatorTok{=}\NormalTok{ math.sqrt(ship}\OperatorTok{**}\DecValTok{2} \OperatorTok{+}\NormalTok{ current}\OperatorTok{**}\DecValTok{2} \OperatorTok{+} \DecValTok{2}\OperatorTok{*}\NormalTok{ship}\OperatorTok{*}\NormalTok{current}\OperatorTok{*}\NormalTok{math.cos(angle\_rad))}

\CommentTok{\# Law of sines}

\NormalTok{theta\_rad }\OperatorTok{=}\NormalTok{ math.asin((current }\OperatorTok{*}\NormalTok{ math.sin(angle\_rad)) }\OperatorTok{/}\NormalTok{ R)}
\NormalTok{theta\_deg }\OperatorTok{=}\NormalTok{ math.degrees(theta\_rad)}

\CommentTok{\# Convert direction to degrees, minutes, seconds}

\NormalTok{deg }\OperatorTok{=} \BuiltInTok{int}\NormalTok{(theta\_deg)}
\NormalTok{minutes\_float }\OperatorTok{=}\NormalTok{ (theta\_deg }\OperatorTok{{-}}\NormalTok{ deg) }\OperatorTok{*} \DecValTok{60}
\NormalTok{minutes }\OperatorTok{=} \BuiltInTok{int}\NormalTok{(minutes\_float)}
\NormalTok{seconds }\OperatorTok{=}\NormalTok{ (minutes\_float }\OperatorTok{{-}}\NormalTok{ minutes) }\OperatorTok{*} \DecValTok{60}

\BuiltInTok{print}\NormalTok{(}\SpecialStringTok{f"Resultant speed: }\SpecialCharTok{\{}\NormalTok{R}\SpecialCharTok{:.4f\}}\SpecialStringTok{ knots"}\NormalTok{)}
\BuiltInTok{print}\NormalTok{(}\SpecialStringTok{f"Direction: }\SpecialCharTok{\{}\NormalTok{theta\_deg}\SpecialCharTok{:.4f\}}\SpecialStringTok{° north of east"}\NormalTok{)}
\BuiltInTok{print}\NormalTok{(}\SpecialStringTok{f"Direction (DMS): }\SpecialCharTok{\{}\NormalTok{deg}\SpecialCharTok{\}}\SpecialStringTok{° }\SpecialCharTok{\{}\NormalTok{minutes}\SpecialCharTok{\}}\SpecialStringTok{\textquotesingle{} }\SpecialCharTok{\{}\NormalTok{seconds}\SpecialCharTok{:.0f\}}\CharTok{\textbackslash{}"}\SpecialStringTok{ north of east"}\NormalTok{)}
\end{Highlighting}
\end{Shaded}

\end{example}

\bookmarksetup{startatroot}

\chapter{Kinematics}\label{sec-kinematics}

\section{Linear Motion Definitions}\label{linear-motion-definitions}

\begin{itemize}
\item
  \(\textbf{Speed} (v)\): The scalar measure of the rate of change of
  distance. Mathematically, \[
    v = \frac{d}{t}
    \] where (d) is distance and (t) is time.
\item
  \(\textbf{Velocity} (\vec{v})\): The vector measure of the rate of
  change of displacement. It is given by: \[
    \vec{v} = \frac{\Delta \vec{x}}{\Delta t}
    \] where \(\Delta \vec{x}\) is the displacement vector and
  \(\Delta t\) is the time interval.
\item
  \(\textbf{Acceleration} (\vec{a})\): The rate of change of velocity
  with respect to time. It is defined as: \[
    \vec{a} = \frac{\Delta \vec{v}}{\Delta t}
    \]
\end{itemize}

\begin{example}[]\protect\hypertarget{exm-ex2-2}{}\label{exm-ex2-2}

A ship's engines are stopped when travelling at a speed of 18 knots, and
the ship comes to rest after 20 min. Assuming uniform deceleration, find
the deceleration (m/s²) and the distance travelled in nautical miles in
that time.

\begin{Shaded}
\begin{Highlighting}[]
\ImportTok{import}\NormalTok{ math}

\CommentTok{\# Given}

\NormalTok{v\_knots }\OperatorTok{=} \DecValTok{18}           \CommentTok{\# initial speed in knots}
\NormalTok{t\_minutes }\OperatorTok{=} \DecValTok{20}         \CommentTok{\# time to stop in minutes}
\NormalTok{kn\_to\_ms }\OperatorTok{=} \FloatTok{0.5144}      \CommentTok{\# conversion factor (m/s per knot)}
\NormalTok{nmi\_in\_m }\OperatorTok{=} \DecValTok{1852}        \CommentTok{\# meters in a nautical mile}

\CommentTok{\# Convert units}

\NormalTok{vi }\OperatorTok{=}\NormalTok{ v\_knots }\OperatorTok{*}\NormalTok{ kn\_to\_ms      }\CommentTok{\# initial speed in m/s}
\NormalTok{vf }\OperatorTok{=} \DecValTok{0}                       \CommentTok{\# final speed}
\NormalTok{t }\OperatorTok{=}\NormalTok{ t\_minutes }\OperatorTok{*} \DecValTok{60}           \CommentTok{\# seconds}

\CommentTok{\# Deceleration (positive magnitude)}

\NormalTok{deceleration }\OperatorTok{=}\NormalTok{ (vi }\OperatorTok{{-}}\NormalTok{ vf) }\OperatorTok{/}\NormalTok{ t     }\CommentTok{\# m/s\^{}2}

\CommentTok{\# Average velocity for uniform deceleration}

\NormalTok{v\_avg }\OperatorTok{=}\NormalTok{ (vi }\OperatorTok{+}\NormalTok{ vf) }\OperatorTok{/} \DecValTok{2}            \CommentTok{\# m/s}

\CommentTok{\# Distance travelled}

\NormalTok{s }\OperatorTok{=}\NormalTok{ v\_avg }\OperatorTok{*}\NormalTok{ t                    }\CommentTok{\# meters}
\NormalTok{s\_nmi }\OperatorTok{=}\NormalTok{ s }\OperatorTok{/}\NormalTok{ nmi\_in\_m             }\CommentTok{\# nautical miles}

\BuiltInTok{print}\NormalTok{(}\SpecialStringTok{f"Deceleration: }\SpecialCharTok{\{}\NormalTok{deceleration}\SpecialCharTok{:.6f\}}\SpecialStringTok{ m/s\^{}2"}\NormalTok{)}
\BuiltInTok{print}\NormalTok{(}\SpecialStringTok{f"Distance travelled: }\SpecialCharTok{\{}\NormalTok{s\_nmi}\SpecialCharTok{:.3f\}}\SpecialStringTok{ nautical miles"}\NormalTok{)}
\end{Highlighting}
\end{Shaded}

\end{example}

\section{Equations of Motion (Constant
Acceleration)}\label{equations-of-motion-constant-acceleration}

\[
\vec{v} = \vec{u} + \vec{a}t
\]

\[
\vec{s} = \frac{\vec{u}+\vec{v}}{2}t
\]

\[
\vec{s} = \vec{u}t + \frac{1}{2}\vec{a}t^2
\]

\[
\vec{v}^2 = \vec{u}^2 + 2\vec{a} \cdot \vec{s}
\]

where:

\begin{itemize}
\tightlist
\item
  \(\vec{u}\): Initial velocity
\item
  \(\vec{v}\): Final velocity
\item
  \(\vec{s}\): Displacement
\item
  \(\vec{a}\): Acceleration
\item
  \(t\): Time
\end{itemize}

\begin{example}[]\protect\hypertarget{exm-ex2-6}{}\label{exm-ex2-6}

A ship's propellers are stopped when travelling at 25 knots. From that
point, it travels 4 kilometres before coming to a complete stop.
Calculate the time it takes to stop in minutes and the average
deceleration in m/s².

\begin{Shaded}
\begin{Highlighting}[]
\CommentTok{\# Given}

\NormalTok{u\_knots }\OperatorTok{=} \DecValTok{25}          \CommentTok{\# initial speed in knots}
\NormalTok{v\_knots }\OperatorTok{=} \DecValTok{0}           \CommentTok{\# final speed in knots}
\NormalTok{s\_m }\OperatorTok{=} \DecValTok{4000}            \CommentTok{\# stopping distance in meters}
\NormalTok{kn\_to\_ms }\OperatorTok{=} \FloatTok{0.5144}     \CommentTok{\# conversion factor knots to m/s}

\CommentTok{\# Convert to m/s}

\NormalTok{u }\OperatorTok{=}\NormalTok{ u\_knots }\OperatorTok{*}\NormalTok{ kn\_to\_ms}
\NormalTok{v }\OperatorTok{=}\NormalTok{ v\_knots }\OperatorTok{*}\NormalTok{ kn\_to\_ms}

\CommentTok{\# Average deceleration (from s = 1/2*(u+v)*t)}

\NormalTok{t }\OperatorTok{=}\NormalTok{ (}\DecValTok{2} \OperatorTok{*}\NormalTok{ s\_m) }\OperatorTok{/}\NormalTok{ (u }\OperatorTok{+}\NormalTok{ v)}

\CommentTok{\# Deceleration using v = u {-} a*t}

\NormalTok{a }\OperatorTok{=}\NormalTok{ (u }\OperatorTok{{-}}\NormalTok{ v) }\OperatorTok{/}\NormalTok{ t        }

\CommentTok{\# Convert time to minutes}

\NormalTok{t\_minutes }\OperatorTok{=}\NormalTok{ t }\OperatorTok{/} \DecValTok{60}

\BuiltInTok{print}\NormalTok{(}\SpecialStringTok{f"Average deceleration: }\SpecialCharTok{\{}\NormalTok{a}\SpecialCharTok{:.5f\}}\SpecialStringTok{ m/s²"}\NormalTok{)}
\BuiltInTok{print}\NormalTok{(}\SpecialStringTok{f"Time to stop: }\SpecialCharTok{\{}\NormalTok{t\_minutes}\SpecialCharTok{:.2f\}}\SpecialStringTok{ minutes"}\NormalTok{)}
\end{Highlighting}
\end{Shaded}

\end{example}

\begin{example}[]\protect\hypertarget{exm-river}{}\label{exm-river}

A boat is rowed at 6 km/h perpendicular to a river flowing at 4 km/h.
The river is 50 meters wide. How far downstream will the boat reach the
opposite bank? What is the boat's actual velocity (magnitude and
direction)?

We treat this as vector addition: the boat's rowing velocity across the
river and the river's current velocity downstream.

Given

\begin{itemize}
\item
  Boat speed across: \(v_b = 6\ \text{km/h}\)
\item
  River current speed: \(v_c = 4\ \text{km/h}\)
\item
  River width: \(w = 50\ \text{m}\)
\end{itemize}

Convert speeds to m/s:

\[
1\ \text{km/h} = \frac{1000}{3600} = 0.277777\ \text{m/s}
\]

\[
v_b = 6 \times 0.277777 = 1.6666667\ \text{m/s}
\]

\[
v_c = 4 \times 0.277777 = 1.1111111\ \text{m/s}
\]

Time to cross the river

\[
t = \frac{w}{v_b} = \frac{50}{1.6666667} = 30\ \text{s}
\]

Downstream drift

\[
\text{drift} = v_c \times t = 1.1111111 \times 30 = 33.3333\ \text{m}
\]

Actual velocity magnitude \[
v = \sqrt{v_b^2 + v_c^2} = 2.00308\ \text{m/s}
\]

Direction (downstream from perpendicular)

\[
\theta = \tan^{-1}\!\left( \frac{v_c}{v_b} \right) = 33.69
\]

\begin{Shaded}
\begin{Highlighting}[]
\ImportTok{import}\NormalTok{ math}

\CommentTok{\# Given}
\NormalTok{boat\_speed\_kmh }\OperatorTok{=} \DecValTok{6}      \CommentTok{\# km/h (perpendicular to river)}
\NormalTok{current\_speed\_kmh }\OperatorTok{=} \DecValTok{4}   \CommentTok{\# km/h (downstream)}
\NormalTok{width }\OperatorTok{=} \DecValTok{50}              \CommentTok{\# meters (river width)}

\CommentTok{\# Convert speeds to m/s}

\NormalTok{boat\_speed }\OperatorTok{=}\NormalTok{ boat\_speed\_kmh }\OperatorTok{*} \DecValTok{1000} \OperatorTok{/} \DecValTok{3600}
\NormalTok{current\_speed }\OperatorTok{=}\NormalTok{ current\_speed\_kmh }\OperatorTok{*} \DecValTok{1000} \OperatorTok{/} \DecValTok{3600}

\CommentTok{\# Time to cross}

\NormalTok{t }\OperatorTok{=}\NormalTok{ width }\OperatorTok{/}\NormalTok{ boat\_speed}

\CommentTok{\# Downstream drift}

\NormalTok{drift }\OperatorTok{=}\NormalTok{ current\_speed }\OperatorTok{*}\NormalTok{ t}

\CommentTok{\# Actual velocity (magnitude)}

\NormalTok{v\_actual }\OperatorTok{=}\NormalTok{ math.sqrt(boat\_speed}\OperatorTok{**}\DecValTok{2} \OperatorTok{+}\NormalTok{ current\_speed}\OperatorTok{**}\DecValTok{2}\NormalTok{)}

\CommentTok{\# Convert to km/h}

\NormalTok{v\_actual\_kmh }\OperatorTok{=}\NormalTok{ v\_actual }\OperatorTok{*} \FloatTok{3.6}

\CommentTok{\# Direction (downstream from perpendicular)}

\NormalTok{theta\_deg }\OperatorTok{=}\NormalTok{ math.degrees(math.atan(current\_speed }\OperatorTok{/}\NormalTok{ boat\_speed))}

\BuiltInTok{print}\NormalTok{(}\SpecialStringTok{f"Time to cross: }\SpecialCharTok{\{}\NormalTok{t}\SpecialCharTok{:.2f\}}\SpecialStringTok{ s"}\NormalTok{)}
\BuiltInTok{print}\NormalTok{(}\SpecialStringTok{f"Downstream drift: }\SpecialCharTok{\{}\NormalTok{drift}\SpecialCharTok{:.2f\}}\SpecialStringTok{ m"}\NormalTok{)}
\BuiltInTok{print}\NormalTok{(}\SpecialStringTok{f"Actual speed: }\SpecialCharTok{\{}\NormalTok{v\_actual}\SpecialCharTok{:.2f\}}\SpecialStringTok{ m/s (}\SpecialCharTok{\{}\NormalTok{v\_actual\_kmh}\SpecialCharTok{:.2f\}}\SpecialStringTok{ km/h)"}\NormalTok{)}
\BuiltInTok{print}\NormalTok{(}\SpecialStringTok{f"Direction: }\SpecialCharTok{\{}\NormalTok{theta\_deg}\SpecialCharTok{:.2f\}}\SpecialStringTok{ degrees downstream"}\NormalTok{)}
\end{Highlighting}
\end{Shaded}

\end{example}

\begin{example}[]\protect\hypertarget{exm-ships}{}\label{exm-ships}

Two ships are moving toward each other in still water. Ship 1 is 310
meters long and travelling at 22 knots, while Ship 2 is 350 meters long
and travelling at 19 knots. Calculate the time it takes for the ships to
completely pass each other..

\begin{Shaded}
\begin{Highlighting}[]
\ImportTok{import}\NormalTok{ math}

\CommentTok{\# Given}

\NormalTok{length1 }\OperatorTok{=} \DecValTok{310}   \CommentTok{\# meters (Ship 1)}
\NormalTok{length2 }\OperatorTok{=} \DecValTok{350}   \CommentTok{\# meters (Ship 2)}
\NormalTok{speed1 }\OperatorTok{=} \DecValTok{22}     \CommentTok{\# knots}
\NormalTok{speed2 }\OperatorTok{=} \DecValTok{19}     \CommentTok{\# knots}

\CommentTok{\# Total distance to be closed (sum of lengths)}

\NormalTok{distance\_m }\OperatorTok{=}\NormalTok{ length1 }\OperatorTok{+}\NormalTok{ length2  }\CommentTok{\# meters}

\CommentTok{\# Convert meters to nautical miles}

\NormalTok{distance\_nm }\OperatorTok{=}\NormalTok{ distance\_m }\OperatorTok{/} \DecValTok{1852}

\CommentTok{\# Relative (closing) speed in knots}

\NormalTok{speed\_rel }\OperatorTok{=}\NormalTok{ speed1 }\OperatorTok{+}\NormalTok{ speed2}

\CommentTok{\# Time in hours}

\NormalTok{time\_hours }\OperatorTok{=}\NormalTok{ distance\_nm }\OperatorTok{/}\NormalTok{ speed\_rel}

\CommentTok{\# Convert to seconds}
\NormalTok{time\_seconds }\OperatorTok{=}\NormalTok{ time\_hours }\OperatorTok{*} \DecValTok{3600}

\BuiltInTok{print}\NormalTok{(}\SpecialStringTok{f"Time for the ships to pass each other: }\SpecialCharTok{\{}\NormalTok{time\_seconds}\SpecialCharTok{:.2f\}}\SpecialStringTok{ seconds"}\NormalTok{)}
\end{Highlighting}
\end{Shaded}

\end{example}

\section{Angular Motion Definitions}\label{angular-motion-definitions}

\begin{itemize}
\item
  \(\textbf{Angular Displacement} (\theta)\): The angle through which an
  object rotates, measured in radians.

  \includegraphics[width=3.22917in,height=\textheight,keepaspectratio]{images/angulardisp.png}
\item
  \(\textbf{Angular Velocity} (\omega)\): The rate of change of angular
  displacement. Mathematically: \[
      \omega = \frac{\Delta \theta}{\Delta t}
      \] \[
  \omega (rad/s) =2\pi n \:where\:n=speed\:in\:rev/s
  \]
\item
  \(\textbf{Angular Acceleration} (\alpha)\): The rate of change of
  angular velocity with respect to time: \[
       \alpha = \frac{\Delta \omega}{\Delta t}
       \]
\end{itemize}

\section{Equations of Angular Motion (Constant Angular
Acceleration)}\label{equations-of-angular-motion-constant-angular-acceleration}

\[
\omega_2 = \omega_1 \mp \alpha t
\]

\[
\theta = \frac{\omega_1+\omega_2}{2}t
\]

\[
\theta = \omega_1 t \mp \frac{1}{2}\alpha t^2 
\]

\[
\omega_2^2 = \omega_1^2 \mp 2\alpha \theta
\]

where:

\begin{itemize}
\item
  \(\omega_1\): Initial angular velocity (rad/s)
\item
  \(\omega_2\): Final angular velocity (rad/s)
\item
  \(\theta\) Angular displacement (rad)
\item
  \(\alpha\): Angular acceleration (rad/s\textsuperscript{2})
\item
  \(t\): Time (s)
\end{itemize}

\begin{example}[]\protect\hypertarget{exm-ex2-10}{}\label{exm-ex2-10}

A shaft is rotating at 40 revolutions per minute (rev/min). It is
uniformly decelerated at a rate of 0.017 rad/s² for 15 seconds.
Calculate: 1. The angular velocity of the shaft at the end of 15 seconds
(in rad/s and rev/min). 2. The total number of revolutions completed by
the shaft during these 15 seconds.

\begin{Shaded}
\begin{Highlighting}[]
\ImportTok{import}\NormalTok{ math}

\CommentTok{\# Given}

\NormalTok{omega1\_rpm }\OperatorTok{=} \DecValTok{40} \CommentTok{\# initial angular speed in rev/min}
\NormalTok{alpha }\OperatorTok{=} \OperatorTok{{-}}\FloatTok{0.017}  \CommentTok{\# angular acceleration (rad/s\^{}2), ({-}) for deceleration}
\NormalTok{t }\OperatorTok{=} \DecValTok{15}          \CommentTok{\# time in seconds}

\CommentTok{\# Convert initial speed to rad/s}

\NormalTok{omega1 }\OperatorTok{=}\NormalTok{ omega1\_rpm }\OperatorTok{*} \DecValTok{2} \OperatorTok{*}\NormalTok{ math.pi }\OperatorTok{/} \DecValTok{60}  

\CommentTok{\# Angular velocity after time t}

\NormalTok{omega2 }\OperatorTok{=}\NormalTok{ omega1 }\OperatorTok{+}\NormalTok{ alpha }\OperatorTok{*}\NormalTok{ t              }

\CommentTok{\# Angular displacement during time t}

\NormalTok{theta }\OperatorTok{=}\NormalTok{ omega1 }\OperatorTok{*}\NormalTok{ t }\OperatorTok{+} \FloatTok{0.5} \OperatorTok{*}\NormalTok{ alpha }\OperatorTok{*}\NormalTok{ t}\OperatorTok{**}\DecValTok{2}  

\CommentTok{\# Convert angular displacement to revolutions}

\NormalTok{revolutions }\OperatorTok{=}\NormalTok{ theta }\OperatorTok{/}\NormalTok{ (}\DecValTok{2} \OperatorTok{*}\NormalTok{ math.pi)}

\CommentTok{\# Convert omega2 to rpm}

\NormalTok{omega2\_rpm }\OperatorTok{=}\NormalTok{ omega2 }\OperatorTok{*} \DecValTok{60} \OperatorTok{/}\NormalTok{ (}\DecValTok{2} \OperatorTok{*}\NormalTok{ math.pi)}

\BuiltInTok{print}\NormalTok{(}\SpecialStringTok{f"Final angular velocity: }\SpecialCharTok{\{}\NormalTok{omega2}\SpecialCharTok{:.4f\}}\SpecialStringTok{ rad/s"}\NormalTok{)}
\BuiltInTok{print}\NormalTok{(}\SpecialStringTok{f"Final angular velocity: }\SpecialCharTok{\{}\NormalTok{omega2\_rpm}\SpecialCharTok{:.4f\}}\SpecialStringTok{ rpm"}\NormalTok{)}
\BuiltInTok{print}\NormalTok{(}\SpecialStringTok{f"Angular displacement: }\SpecialCharTok{\{}\NormalTok{revolutions}\SpecialCharTok{:.4f\}}\SpecialStringTok{ revolutions"}\NormalTok{)}
\end{Highlighting}
\end{Shaded}

\end{example}

\section{Relation Between Linear and Angular
Motion}\label{relation-between-linear-and-angular-motion}

The relationship between linear and angular motion is described by the
following equations:

\(s = r \theta\) (linear displacement \(s\) and angular displacement
\(\theta\)).

\(v = r \omega\) (linear velocity \(v\) and angular velocity
\(\omega\)).

\(a = r \alpha\) (linear acceleration \(a\) and angular acceleration
\(\alpha\)).

\subsection{Variables}\label{variables-1}

\begin{itemize}
\item
  \(v\): Linear velocity, the rate of change of linear displacement
  \((v = \frac{ds}{dt})\).
\item
  \(a\): Linear acceleration, the rate of change of linear velocity
  \((a = \frac{dv}{dt})\).
\item
  \(s\): Linear displacement, the distance moved along the circular
  path.
\item
  \(r\): Radius of the circular path.
\item
  \(\omega\): Angular velocity, the rate of change of angular
  displacement \((\omega = \frac{d\theta}{dt})\).
\item
  \(\alpha\): Angular acceleration, the rate of change of angular
  velocity \((\alpha = \frac{d\omega}{dt})\).
\item
  \(\theta\): Angular displacement, the angle swept by the radius in
  radians.
\end{itemize}

\section{Summary}\label{summary-1}

\begin{itemize}
\item
  Linear motion is directly proportional to angular motion, with the
  radius (\(r\)) acting as the proportionality constant.
\item
  Units for the variables:

  \begin{itemize}
  \item
    \(v\): meters per second \((m/s)\),
  \item
    \(a\): meters per second squared \((m/s^2)\),
  \item
    \(s\): meters \((m)\),
  \item
    \(r\): meters \((m)\),
  \item
    \(\omega\): radians per second \((rad/s)\),
  \item
    \(\alpha\): radians per second squared \((rad/s^2)\),
  \item
    \(\theta\): radians (\(rad\)).
  \end{itemize}
\end{itemize}

\bookmarksetup{startatroot}

\chapter{Dynamics}\label{dynamics}

Dynamics is the branch of mechanics that studies the motion of bodies
under the action of forces. This chapter examines situations where
forces produce acceleration, and it presents methods for analyzing the
resulting motion, including the application of Newton's laws, energy
principles, and momentum conservation.

\section{Inertia}\label{inertia}

Inertia is the property of a body that resists any change in its state
of motion and is directly proportional to its mass. It can be understood
as a form of ``sluggishness'' inherent to matter. For an object at rest,
a net force is required to set it in motion; the greater the mass, the
greater the force needed. Similarly, if an object is already moving, a
net force is required to change its speed or direction, and the
magnitude of this force is again proportional to the mass.

The term ``inertia'' is also used in structural mechanics to describe
the resistance of a beam's cross-section to bending, quantified by the
second moment of area (see lecture notes on Second Moments).

\section{Momentum}\label{momentum}

Momentum is the product of an object's mass and its velocity and
quantifies the amount of motion a moving body possesses. This concept is
particularly important when considering a large vessel approaching a
dock. Once the vessel is underway, its considerable momentum means that
a substantial force, applied over sufficient time, is required to bring
it to a stop or significantly alter its course. Without such
intervention, the vessel will not stop quickly on its own and may
collide with the dock.

\section{Newton's Laws of Motion}\label{newtons-laws-of-motion}

\begin{enumerate}
\def\labelenumi{\arabic{enumi}.}
\item
  \textbf{First Law (Law of Inertia)}:\\
  An object remains at rest, or in uniform motion in a straight line,
  unless acted upon by a net external force.\\
  \[
  \Sigma F = 0 \implies v = \text{constant}
  \]
\item
  \textbf{Second Law (Law of Acceleration)}:\\
  The acceleration of an object is directly proportional to the net
  force acting on it and inversely proportional to its mass.\\
  \[
  \vec{F} = m \vec{a}
  \]
\item
  \textbf{Third Law (Action-Reaction Law)}:\\
  For every action, there is an equal and opposite reaction.\\
  \[
  \vec{F}_{\text{action}} = -\vec{F}_{\text{reaction}}
  \]
\end{enumerate}

\begin{example}[]\protect\hypertarget{exm-3-1}{}\label{exm-3-1}

Find the accelerating force required to increase the velocity of a body
which has a mass of 20 kg from 30 m/s to 70 m/s in 4 s.

\begin{Shaded}
\begin{Highlighting}[]
\CommentTok{\# Given}

\NormalTok{m }\OperatorTok{=} \DecValTok{20}           \CommentTok{\# mass in kg}
\NormalTok{u }\OperatorTok{=} \DecValTok{30}           \CommentTok{\# initial velocity in m/s}
\NormalTok{v }\OperatorTok{=} \DecValTok{70}           \CommentTok{\# final velocity in m/s}
\NormalTok{t }\OperatorTok{=} \DecValTok{4}            \CommentTok{\# time in seconds}

\CommentTok{\# Calculate acceleration}

\NormalTok{a }\OperatorTok{=}\NormalTok{ (v }\OperatorTok{{-}}\NormalTok{ u) }\OperatorTok{/}\NormalTok{ t}

\CommentTok{\# Calculate force}

\NormalTok{F }\OperatorTok{=}\NormalTok{ m }\OperatorTok{*}\NormalTok{ a}

\BuiltInTok{print}\NormalTok{(}\SpecialStringTok{f"Acceleration: }\SpecialCharTok{\{}\NormalTok{a}\SpecialCharTok{:.2f\}}\SpecialStringTok{ m/s²"}\NormalTok{)}
\BuiltInTok{print}\NormalTok{(}\SpecialStringTok{f"Accelerating force: }\SpecialCharTok{\{}\NormalTok{F}\SpecialCharTok{:.2f\}}\SpecialStringTok{ N"}\NormalTok{)}
\end{Highlighting}
\end{Shaded}

\end{example}

\begin{example}[]\protect\hypertarget{exm-3-2}{}\label{exm-3-2}

A lift is supported by a steel wire rope, the total mass of the lift and
contents is 750 kg. Find the tension in the wire rope, in newtons, when
the lift is (i) moving at constant velocity, (ii) moving upwards and
accelerating at 1.2 m/s\textsuperscript{2}, (iii) moving upwards and
retarding at 1.2 m/s\textsuperscript{2}

\begin{Shaded}
\begin{Highlighting}[]
\CommentTok{\# Given}

\NormalTok{m }\OperatorTok{=} \DecValTok{750}          \CommentTok{\# mass in kg}
\NormalTok{g }\OperatorTok{=} \FloatTok{9.81}         \CommentTok{\# gravitational acceleration in m/s\^{}2}

\CommentTok{\# Accelerations for each case}

\NormalTok{a\_constant }\OperatorTok{=} \DecValTok{0}
\NormalTok{a\_up\_accel }\OperatorTok{=} \FloatTok{1.2}
\NormalTok{a\_up\_decel }\OperatorTok{=} \OperatorTok{{-}}\FloatTok{1.2}

\CommentTok{\# Tension calculations}

\NormalTok{T\_constant }\OperatorTok{=}\NormalTok{ m }\OperatorTok{*}\NormalTok{ (g }\OperatorTok{+}\NormalTok{ a\_constant)}
\NormalTok{T\_up\_accel }\OperatorTok{=}\NormalTok{ m }\OperatorTok{*}\NormalTok{ (g }\OperatorTok{+}\NormalTok{ a\_up\_accel)}
\NormalTok{T\_up\_decel }\OperatorTok{=}\NormalTok{ m }\OperatorTok{*}\NormalTok{ (g }\OperatorTok{+}\NormalTok{ a\_up\_decel)}

\BuiltInTok{print}\NormalTok{(}\SpecialStringTok{f"Tension at constant velocity: }\SpecialCharTok{\{}\NormalTok{T\_constant}\SpecialCharTok{:.4f\}}\SpecialStringTok{ N"}\NormalTok{)}
\BuiltInTok{print}\NormalTok{(}\SpecialStringTok{f"Tension during upward acceleration: }\SpecialCharTok{\{}\NormalTok{T\_up\_accel}\SpecialCharTok{:.4f\}}\SpecialStringTok{ N"}\NormalTok{)}
\BuiltInTok{print}\NormalTok{(}\SpecialStringTok{f"Tension during upward deceleration: }\SpecialCharTok{\{}\NormalTok{T\_up\_decel}\SpecialCharTok{:.4f\}}\SpecialStringTok{ N"}\NormalTok{)}
\end{Highlighting}
\end{Shaded}

\end{example}

\section{Linear momentum}\label{linear-momentum}

Linear momentum is a fundamental concept in physics that quantifies the
motion of an object. It is defined as the product of the object's mass
and its velocity. Because momentum is a vector quantity, it possesses
both magnitude and direction.

\[
Linear\ momentum = m \vec{v}
\]

Where:

\begin{itemize}
\item
  Linear momentum is in kg·m/s.
\item
  \(m\) is the mass of the object in kilograms.
\item
  \(\vec{v}\) is the velocity of the object in meters per second.
\end{itemize}

\subsection{Conservation of Linear
Momentum}\label{conservation-of-linear-momentum}

The law of conservation of linear momentum states that, in a closed
system with no external forces (or when the net external force is zero),
the total linear momentum remains constant over time. This principle is
particularly evident in collisions between two bodies. During a
collision, the force exerted by the first body on the second is equal in
magnitude and opposite in direction to the force exerted by the second
on the first (Newton's third law). These equal and opposite forces act
for the same duration, producing changes in momentum that are equal in
magnitude but opposite in direction. Consequently, the momentum gained
by one body exactly equals the momentum lost by the other. Therefore,
the total momentum of the system before the collision is equal to the
total momentum after the collision.

In the absence of external forces, linear momentum is neither created
nor destroyed; it is conserved. This is known as the law of conservation
of linear momentum.

\begin{example}[]\protect\hypertarget{exm-3-5}{}\label{exm-3-5}

A hand hammer with a head mass of 0.8 kg strikes a chisel at 9 m/s and
comes to rest in 1/250 seconds. Calculate the average force exerted
during the blow.

\begin{Shaded}
\begin{Highlighting}[]
\CommentTok{\# Given}

\NormalTok{m }\OperatorTok{=} \FloatTok{0.8}               \CommentTok{\# kg}
\NormalTok{u }\OperatorTok{=} \DecValTok{9}                 \CommentTok{\# m/s (initial velocity)}
\NormalTok{v }\OperatorTok{=} \DecValTok{0}                 \CommentTok{\# m/s (final velocity)}
\NormalTok{t }\OperatorTok{=} \DecValTok{1}\OperatorTok{/}\DecValTok{250}             \CommentTok{\# s (time to come to rest)}

\CommentTok{\# Change in momentum}

\NormalTok{delta\_p }\OperatorTok{=}\NormalTok{ m }\OperatorTok{*}\NormalTok{ (v }\OperatorTok{{-}}\NormalTok{ u)}

\CommentTok{\# Average force}

\NormalTok{F\_avg }\OperatorTok{=}\NormalTok{ delta\_p }\OperatorTok{/}\NormalTok{ t}

\BuiltInTok{print}\NormalTok{(}\SpecialStringTok{f"Average force during the blow: }\SpecialCharTok{\{}\NormalTok{F\_avg}\SpecialCharTok{:.4f\}}\SpecialStringTok{ N"}\NormalTok{)}
\end{Highlighting}
\end{Shaded}

\end{example}

\begin{example}[]\protect\hypertarget{exm-3-6}{}\label{exm-3-6}

A jet of fresh water, 20 mm in diameter, emerges horizontally from a
nozzle at a speed of 21 m/s and strikes a stationary vertical plate
normally. Assuming no splashback (i.e., the water comes to rest upon
impact with no velocity component normal to the plate after striking),
calculate: (a) the mass flow rate of water leaving the nozzle (in kg/s),
and (b) the force exerted by the jet on the plate. The density of fresh
water is 1000 kg/m³.

\begin{Shaded}
\begin{Highlighting}[]
\ImportTok{import}\NormalTok{ math}

\CommentTok{\# Given}

\NormalTok{d }\OperatorTok{=} \FloatTok{0.02}          \CommentTok{\# diameter in meters}
\NormalTok{v }\OperatorTok{=} \DecValTok{21}            \CommentTok{\# jet speed in m/s}
\NormalTok{rho }\OperatorTok{=} \DecValTok{1000}        \CommentTok{\# density of water in kg/m\^{}3}

\CommentTok{\# Mass flow rate}

\NormalTok{A }\OperatorTok{=}\NormalTok{ math.pi }\OperatorTok{*}\NormalTok{ (d}\OperatorTok{**}\DecValTok{2}\NormalTok{) }\OperatorTok{/} \DecValTok{4}        \CommentTok{\# cross{-}sectional area}
\NormalTok{v\_dot }\OperatorTok{=}\NormalTok{ A }\OperatorTok{*}\NormalTok{ v                   }\CommentTok{\# volumetric flow rate}
\NormalTok{m\_dot }\OperatorTok{=}\NormalTok{ rho }\OperatorTok{*}\NormalTok{ v\_dot             }\CommentTok{\# mass flow rate}

\CommentTok{\# Force on the plate}

\NormalTok{F }\OperatorTok{=}\NormalTok{ m\_dot }\OperatorTok{*}\NormalTok{ v                   }\CommentTok{\# momentum change}

\BuiltInTok{print}\NormalTok{(}\SpecialStringTok{f"Mass flow rate (kg/s): }\SpecialCharTok{\{}\NormalTok{m\_dot}\SpecialCharTok{:.4f\}}\SpecialStringTok{"}\NormalTok{)}
\BuiltInTok{print}\NormalTok{(}\SpecialStringTok{f"Force on plate (N): }\SpecialCharTok{\{}\NormalTok{F}\SpecialCharTok{:.4f\}}\SpecialStringTok{"}\NormalTok{)}
\end{Highlighting}
\end{Shaded}

\end{example}

\section{Angular Momentum}\label{angular-momentum}

Angular momentum is the moment of linear momentum about a chosen point
or axis. For a single particle, it is given by the cross product of its
position vector (measured from that point) and its linear momentum.

More broadly, angular momentum measures the rotational motion of a
system. For extended bodies, its evaluation depends on both the distance
of mass elements from the axis (the first moment of position) and how
the mass is distributed throughout the object---the moment of inertia,
or second moment of mass.

\begin{figure}[H]

{\centering \includegraphics[width=4cm,height=\textheight,keepaspectratio]{images/angularmomentum.png}

}

\caption{Angular momentum}

\end{figure}%

Given \textbf{linear velocity} \(\vec{v}\) and \textbf{angular velocity}
\(\omega\):

\[
\vec{v} = r \omega
\]

The \textbf{linear momentum} is determined by:

\[
\text{Linear momentum} = m \vec{v}
\]

Substituting for \(\vec{v}\):

\[
\text{Linear momentum} = m r \omega
\]

The \textbf{moment of linear momentum} can then be written as:

\[
\text{Moment of linear momentum} = m r \omega r
\]

The moment of linear momentum is referred to as the \textbf{angular
momentum}. Simplifying the above equation:

\[
\text{Angular momentum} = m r^2 \omega
\]

Here, \(m r^2\) is the \textbf{moment of inertia} (second moment of
mass) of the object about its axis of rotation, denoted as \(I\). Thus,
the angular momentum can be expressed as:

\[
\text{Angular momentum} = I \omega
\]

Where:

\begin{itemize}
\tightlist
\item
  \(I\) is the moment of inertia in \(kg·m²\).
\item
  \(\omega\) is the angular velocity in \(rad/s\).
\end{itemize}

Additionally, the moment of inertia \(I\) is given by:

\[
I = m k^2
\]

Where:

\begin{itemize}
\tightlist
\item
  \(I\) is the moment of inertia in \(kg·m²\).
\item
  \(m\) is the mass in \(kg\).
\item
  \(k\) is the radius of gyration in \(m\).
\end{itemize}

\begin{tcolorbox}[enhanced jigsaw, toprule=.15mm, leftrule=.75mm, colbacktitle=quarto-callout-note-color!10!white, bottomrule=.15mm, bottomtitle=1mm, toptitle=1mm, colback=white, colframe=quarto-callout-note-color-frame, opacitybacktitle=0.6, coltitle=black, opacityback=0, breakable, titlerule=0mm, title=\textcolor{quarto-callout-note-color}{\faInfo}\hspace{0.5em}{Note}, left=2mm, rightrule=.15mm, arc=.35mm]

In structural engineering, the moment of inertia (m⁴) is a measure of
resistance to bending (deflection or stiffness).

In angular motion and physics, the moment of inertia (kg·m²) is a
measure of resistance to angular acceleration (rotational inertia).

\end{tcolorbox}

\section{The Radius of Gyration}\label{the-radius-of-gyration}

The radius of gyration is a geometric property of a rigid body or mass
distribution that describes how the mass is distributed relative to an
axis of rotation. It represents the effective distance from the axis at
which the entire mass could be concentrated while producing the same
moment of inertia as the actual distributed mass.

In simpler terms, a smaller radius of gyration means the mass is closer
to the axis, which is good for reducing inertia in rotating machinery.
This is desirable in applications like engine crankshafts or turbine
rotors where rapid changes in speed are required. A larger radius of
gyration means the mass is farther from the axis, like a hollow cylinder
has a larger k than a solid cylinder of the same mass and outer radius.
This is beneficial when energy storage is needed (e.g., flywheels).

\begin{example}[]\protect\hypertarget{exm-3-9}{}\label{exm-3-9}

A flywheel of mass 500 kg and radius of gyration 1.2 m is running at 300
rev/min. By means of a clutch, this flywheel is suddenly connected to
another flywheel, mass 2000 kg and radius of gyration 0.6 m, initially
at rest. Calculate their common speed of rotation after engagement.

\begin{Shaded}
\begin{Highlighting}[]
\ImportTok{import}\NormalTok{ math}

\CommentTok{\# Given data}

\NormalTok{m1, k1 }\OperatorTok{=} \DecValTok{500}\NormalTok{, }\FloatTok{1.2}
\NormalTok{m2, k2 }\OperatorTok{=} \DecValTok{2000}\NormalTok{, }\FloatTok{0.6}
\NormalTok{rpm1 }\OperatorTok{=} \DecValTok{300}

\CommentTok{\# Moments of inertia}

\NormalTok{I1 }\OperatorTok{=}\NormalTok{ m1 }\OperatorTok{*}\NormalTok{ k1}\OperatorTok{**}\DecValTok{2}
\NormalTok{I2 }\OperatorTok{=}\NormalTok{ m2 }\OperatorTok{*}\NormalTok{ k2}\OperatorTok{**}\DecValTok{2}

\CommentTok{\# Convert rpm to rad/s}

\NormalTok{omega1 }\OperatorTok{=}\NormalTok{ (rpm1 }\OperatorTok{/} \DecValTok{60}\NormalTok{) }\OperatorTok{*} \DecValTok{2} \OperatorTok{*}\NormalTok{ math.pi}

\CommentTok{\# Conservation of angular momentum}

\NormalTok{omega\_f }\OperatorTok{=}\NormalTok{ (I1 }\OperatorTok{*}\NormalTok{ omega1) }\OperatorTok{/}\NormalTok{ (I1 }\OperatorTok{+}\NormalTok{ I2)}

\CommentTok{\# Convert rad/s to rpm}

\NormalTok{rpm\_f }\OperatorTok{=}\NormalTok{ (omega\_f }\OperatorTok{/}\NormalTok{ (}\DecValTok{2} \OperatorTok{*}\NormalTok{ math.pi)) }\OperatorTok{*} \DecValTok{60}

\BuiltInTok{print}\NormalTok{(}\SpecialStringTok{f"Final angular velocity (rad/s): }\SpecialCharTok{\{}\NormalTok{omega\_f}\SpecialCharTok{:.4f\}}\SpecialStringTok{"}\NormalTok{)}
\BuiltInTok{print}\NormalTok{(}\SpecialStringTok{f"Final angular velocity (rpm): }\SpecialCharTok{\{}\NormalTok{rpm\_f}\SpecialCharTok{:.4f\}}\SpecialStringTok{"}\NormalTok{)}
\end{Highlighting}
\end{Shaded}

\end{example}

\section{Turning Moment}\label{turning-moment}

\begin{figure}[H]

{\centering \includegraphics[width=4cm,height=3.3cm]{images/turningmoment.png}

}

\caption{Turning moment}

\end{figure}%

We know the relation between linear acceleration \(a\) and angular
acceleration \(\alpha\):

\[
a = r \alpha
\]

And

\[
F = ma
\]

Therefore,

\[
F = m \alpha r
\]

If the force is not applied directly on the rim but at a greater or
lesser leverage, say \(L\) from the centre, the effective force on the
rim causing it to accelerate will be greater or lesser accordingly, in
the ratio of \(L\) to \(r\). Thus:

\[
F \frac{L}{r} = m \alpha r
\]

Multiplying both sides by \(r\),

\[
FL = m \alpha r^2
\]

Now, \(FL\) is the torque applied. Therefore, the accelerating torque
is:

\[
\tau = m r^2 \alpha
\]

Or,

\[
\tau = m k^2 \alpha
\]

\(\tau\) is also given by:

\[
\tau = I \alpha
\]

Where:

\begin{itemize}
\tightlist
\item
  \(\tau\) is the torque in \(Nm\).
\item
  \(I\) is the moment of inertia in \(kg·m²\).
\item
  \(\alpha\): Angular acceleration in \(rad/s^2\).
\end{itemize}

\begin{example}[]\protect\hypertarget{exm-3-10}{}\label{exm-3-10}

The mass of a flywheel is 175 kg and its radius of gyration is 380 mm.
Find the torque required to attain a speed of 500 rev/min from rest in
30 s.

\begin{Shaded}
\begin{Highlighting}[]
\ImportTok{import}\NormalTok{ math}

\CommentTok{\# Given}
\NormalTok{m }\OperatorTok{=} \DecValTok{175}                 \CommentTok{\# mass in kg}
\NormalTok{k }\OperatorTok{=} \FloatTok{0.38}                \CommentTok{\# radius of gyration in m}
\NormalTok{rpm\_final }\OperatorTok{=} \DecValTok{500}         \CommentTok{\# final speed}
\NormalTok{t }\OperatorTok{=} \DecValTok{30}                  \CommentTok{\# time in seconds}

\CommentTok{\# Moment of inertia}
\NormalTok{I }\OperatorTok{=}\NormalTok{ m }\OperatorTok{*}\NormalTok{ k}\OperatorTok{**}\DecValTok{2}

\CommentTok{\# Convert rpm → rad/s}
\NormalTok{omega\_final }\OperatorTok{=}\NormalTok{ (rpm\_final }\OperatorTok{*} \DecValTok{2} \OperatorTok{*}\NormalTok{ math.pi) }\OperatorTok{/} \DecValTok{60}

\CommentTok{\# Angular acceleration}
\NormalTok{alpha }\OperatorTok{=}\NormalTok{ omega\_final }\OperatorTok{/}\NormalTok{ t}

\CommentTok{\# Torque}
\NormalTok{tau }\OperatorTok{=}\NormalTok{ I }\OperatorTok{*}\NormalTok{ alpha}

\BuiltInTok{print}\NormalTok{(}\SpecialStringTok{f"Moment of inertia I (kg·m\^{}2): }\SpecialCharTok{\{}\NormalTok{I}\SpecialCharTok{:.4f\}}\SpecialStringTok{"}\NormalTok{)}
\BuiltInTok{print}\NormalTok{(}\SpecialStringTok{f"Final angular velocity ω (rad/s): }\SpecialCharTok{\{}\NormalTok{omega\_final}\SpecialCharTok{:.4f\}}\SpecialStringTok{"}\NormalTok{)}
\BuiltInTok{print}\NormalTok{(}\SpecialStringTok{f"Angular acceleration α (rad/s\^{}2): }\SpecialCharTok{\{}\NormalTok{alpha}\SpecialCharTok{:.4f\}}\SpecialStringTok{"}\NormalTok{)}
\BuiltInTok{print}\NormalTok{(}\SpecialStringTok{f"Required torque τ (N·m): }\SpecialCharTok{\{}\NormalTok{tau}\SpecialCharTok{:.4f\}}\SpecialStringTok{"}\NormalTok{)}
\end{Highlighting}
\end{Shaded}

\end{example}

\begin{example}[]\protect\hypertarget{exm-3-11}{}\label{exm-3-11}

The torque to overcome frictional and other resistances of a turbine is
317 N m and may be considered constant for all speeds. The mass of the
rotating parts is 1.59 t and the radius of gyration is 0.686 m. If the
gas is cut off when the turbine is running free of load at 1920 rev/min,
find the time it will take to come to rest and the number of revolutions
turned during that time.

\begin{Shaded}
\begin{Highlighting}[]
\ImportTok{import}\NormalTok{ math}

\CommentTok{\# Given}

\NormalTok{tau }\OperatorTok{=} \FloatTok{317.0}           \CommentTok{\# N·m}
\NormalTok{m }\OperatorTok{=} \FloatTok{1.59} \OperatorTok{*} \DecValTok{1000}       \CommentTok{\# kg}
\NormalTok{k }\OperatorTok{=} \FloatTok{0.686}             \CommentTok{\# m}
\NormalTok{rpm0 }\OperatorTok{=} \FloatTok{1920.0}         \CommentTok{\# rev/min}


\NormalTok{I }\OperatorTok{=}\NormalTok{ m }\OperatorTok{*}\NormalTok{ k}\OperatorTok{**}\DecValTok{2}
\NormalTok{omega0 }\OperatorTok{=}\NormalTok{ (rpm0 }\OperatorTok{/} \FloatTok{60.0}\NormalTok{) }\OperatorTok{*} \FloatTok{2.0} \OperatorTok{*}\NormalTok{ math.pi   }\CommentTok{\# rad/s}
\NormalTok{alpha }\OperatorTok{=}\NormalTok{ tau }\OperatorTok{/}\NormalTok{ I                          }\CommentTok{\# rad/s\^{}2}
\NormalTok{t }\OperatorTok{=}\NormalTok{ omega0 }\OperatorTok{/}\NormalTok{ alpha                       }\CommentTok{\# s}
\NormalTok{theta }\OperatorTok{=} \FloatTok{0.5} \OperatorTok{*}\NormalTok{ omega0 }\OperatorTok{*}\NormalTok{ t                 }\CommentTok{\# rad}
\NormalTok{revs }\OperatorTok{=}\NormalTok{ theta }\OperatorTok{/}\NormalTok{ (}\FloatTok{2.0} \OperatorTok{*}\NormalTok{ math.pi)}


\BuiltInTok{print}\NormalTok{(}\SpecialStringTok{f"I (kg·m\^{}2):           }\SpecialCharTok{\{}\NormalTok{I}\SpecialCharTok{:.4f\}}\SpecialStringTok{"}\NormalTok{)}
\BuiltInTok{print}\NormalTok{(}\SpecialStringTok{f"omega0 (rad/s):       }\SpecialCharTok{\{}\NormalTok{omega0}\SpecialCharTok{:.4f\}}\SpecialStringTok{"}\NormalTok{)}
\BuiltInTok{print}\NormalTok{(}\SpecialStringTok{f"alpha (rad/s\^{}2):      }\SpecialCharTok{\{}\NormalTok{alpha}\SpecialCharTok{:.6f\}}\SpecialStringTok{"}\NormalTok{)}
\BuiltInTok{print}\NormalTok{(}\SpecialStringTok{f"time to stop (s):     }\SpecialCharTok{\{}\NormalTok{t}\SpecialCharTok{:.4f\}}\SpecialStringTok{"}\NormalTok{)}
\BuiltInTok{print}\NormalTok{(}\SpecialStringTok{f"time to stop (min):   }\SpecialCharTok{\{}\NormalTok{t}\OperatorTok{/}\DecValTok{60}\SpecialCharTok{:.4f\}}\SpecialStringTok{"}\NormalTok{)}
\BuiltInTok{print}\NormalTok{(}\SpecialStringTok{f"revolutions:          }\SpecialCharTok{\{}\NormalTok{revs}\SpecialCharTok{:.4f\}}\SpecialStringTok{"}\NormalTok{)}
\end{Highlighting}
\end{Shaded}

\end{example}

\section{Power by Torque}\label{power-by-torque}

\begin{figure}[H]

{\centering \includegraphics[width=4cm,height=\textheight,keepaspectratio]{images/powerbytorque.png}

}

\caption{A rotating mechanism.}

\end{figure}%

Consider a force \(F\) applied at a radius \(r\) on a rotating
mechanism, as shown above.

The work done in one revolution is the product of the force and the
circumference. Therefore:

\[
W = F \cdot 2 \pi r
\]

If the mechanism is running at \(n\) revolutions per second:

\[
\text{Power} = F \cdot 2 \pi r n
\]

Since torque \(\tau = F r\), we can rewrite the equation as:

\[
P = \tau \cdot 2 \pi n
\]

Given that \(\omega = 2 \pi n\):

\[
P = \tau \cdot \omega
\]

Where:

\begin{itemize}
\tightlist
\item
  \(P\) is the power in watts (W),
\item
  \(\tau\) is the torque in newton-meters (Nm), and
\item
  \(\omega\) is the angular velocity in radians per second (rad/s).
\end{itemize}

\begin{example}[]\protect\hypertarget{exm-4-4}{}\label{exm-4-4}

The mean torque in a propeller shaft is 2.26 x 10\textsuperscript{5} Nm
when running at 140 rpm. Find the power transmitted.

\begin{Shaded}
\begin{Highlighting}[]
\ImportTok{import}\NormalTok{ math}

\CommentTok{\# Given}

\NormalTok{T }\OperatorTok{=} \FloatTok{2.26e5}      \CommentTok{\# Nm}
\NormalTok{rpm }\OperatorTok{=} \DecValTok{140}

\NormalTok{omega }\OperatorTok{=}\NormalTok{ rpm }\OperatorTok{*} \DecValTok{2} \OperatorTok{*}\NormalTok{ math.pi }\OperatorTok{/} \DecValTok{60}
\NormalTok{P }\OperatorTok{=}\NormalTok{ T }\OperatorTok{*}\NormalTok{ omega}

\BuiltInTok{print}\NormalTok{(}\SpecialStringTok{f"Angular speed (rad/s): }\SpecialCharTok{\{}\NormalTok{omega}\SpecialCharTok{:.4f\}}\SpecialStringTok{"}\NormalTok{)}
\BuiltInTok{print}\NormalTok{(}\SpecialStringTok{f"Power (W): }\SpecialCharTok{\{}\NormalTok{P}\SpecialCharTok{:.4f\}}\SpecialStringTok{"}\NormalTok{)}
\BuiltInTok{print}\NormalTok{(}\SpecialStringTok{f"Power (MW): }\SpecialCharTok{\{}\NormalTok{P}\OperatorTok{/}\FloatTok{1e6}\SpecialCharTok{:.4f\}}\SpecialStringTok{"}\NormalTok{)}
\end{Highlighting}
\end{Shaded}

\end{example}

\begin{example}[]\protect\hypertarget{exm-4-6}{}\label{exm-4-6}

One gear wheel with 100 teeth of 6 mm pitch running at 250 rev/min
drives another which has 50 teeth. If the power transmitted is 0.5 kW,
find the driving force on the teeth and the speed of the driven wheel.

\begin{Shaded}
\begin{Highlighting}[]
\ImportTok{import}\NormalTok{ math}

\CommentTok{\# Given}

\NormalTok{N1 }\OperatorTok{=} \DecValTok{100}          \CommentTok{\# teeth on driving wheel}
\NormalTok{N2 }\OperatorTok{=} \DecValTok{50}           \CommentTok{\# teeth on driven wheel}
\NormalTok{pitch }\OperatorTok{=} \FloatTok{0.006}     \CommentTok{\# circular pitch (m)}
\NormalTok{n1 }\OperatorTok{=} \DecValTok{250}          \CommentTok{\# driving rpm}
\NormalTok{P }\OperatorTok{=} \FloatTok{0.5e3}         \CommentTok{\# power in watts}

\CommentTok{\# Pitch diameter of driving wheel}

\NormalTok{D1 }\OperatorTok{=}\NormalTok{ (N1 }\OperatorTok{*}\NormalTok{ pitch) }\OperatorTok{/}\NormalTok{ math.pi}
\NormalTok{r1 }\OperatorTok{=}\NormalTok{ D1 }\OperatorTok{/} \DecValTok{2}

\CommentTok{\# Angular speed of driving wheel}

\NormalTok{omega1 }\OperatorTok{=}\NormalTok{ n1 }\OperatorTok{*} \DecValTok{2} \OperatorTok{*}\NormalTok{ math.pi }\OperatorTok{/} \DecValTok{60}

\CommentTok{\# Torque on driving wheel}

\NormalTok{T }\OperatorTok{=}\NormalTok{ P }\OperatorTok{/}\NormalTok{ omega1}

\CommentTok{\# Tangential driving force}

\NormalTok{F }\OperatorTok{=}\NormalTok{ T }\OperatorTok{/}\NormalTok{ r1}

\CommentTok{\# Driven wheel speed}

\NormalTok{n2 }\OperatorTok{=}\NormalTok{ n1 }\OperatorTok{*}\NormalTok{ (N1 }\OperatorTok{/}\NormalTok{ N2)}

\BuiltInTok{print}\NormalTok{(}\SpecialStringTok{f"Pitch diameter D1 (m): }\SpecialCharTok{\{}\NormalTok{D1}\SpecialCharTok{:.4f\}}\SpecialStringTok{"}\NormalTok{)}
\BuiltInTok{print}\NormalTok{(}\SpecialStringTok{f"Torque T (N·m): }\SpecialCharTok{\{}\NormalTok{T}\SpecialCharTok{:.4f\}}\SpecialStringTok{"}\NormalTok{)}
\BuiltInTok{print}\NormalTok{(}\SpecialStringTok{f"Tangential force F (N): }\SpecialCharTok{\{}\NormalTok{F}\SpecialCharTok{:.4f\}}\SpecialStringTok{"}\NormalTok{)}
\BuiltInTok{print}\NormalTok{(}\SpecialStringTok{f"Driven wheel speed n2 (rpm): }\SpecialCharTok{\{}\NormalTok{n2}\SpecialCharTok{:.4f\}}\SpecialStringTok{"}\NormalTok{)}
\end{Highlighting}
\end{Shaded}

\end{example}

\section{Kinetic Energy of Rotation}\label{kinetic-energy-of-rotation}

We know that \(K.E. = \frac{1}{2}mv^2\), where \(v\) is the linear
velocity of the body. For a rotating body, the effective linear velocity
is at the radius of gyration, as this is the radius at which the entire
mass of the rotating body can be considered to act.

Let \(k\) = radius of gyration (in meters),\\
Let \(\omega\) = angular velocity (in radians per second).

The relationship between linear and angular velocity is:

\[
v = \omega k
\]

Substituting \(v^2 = \omega^2 k^2\) into \(K.E. = \frac{1}{2}mv^2\)
gives:

\[
\text{Rotational K.E.} = \frac{1}{2} m k^2 \omega^2
\]

Since \(I = m k^2\), where \(I\) is the moment of inertia:

\[
\text{Rotational K.E.} = \frac{1}{2} I \omega^2
\]

\begin{example}[]\protect\hypertarget{exm-4-10}{}\label{exm-4-10}

The radius of gyration of the flywheel of a shearing machine is 0.46 m
and its mass is 750 kg. Find the kinetic energy stored in it when
running at 120 rev/min. If the speed falls to 100 rev/min during the
cutting stroke, find the kinetic energy given out by the wheel.

\begin{Shaded}
\begin{Highlighting}[]
\ImportTok{import}\NormalTok{ math}

\NormalTok{m }\OperatorTok{=} \DecValTok{750}
\NormalTok{k }\OperatorTok{=} \FloatTok{0.46}
\NormalTok{I }\OperatorTok{=}\NormalTok{ m }\OperatorTok{*}\NormalTok{ k}\OperatorTok{**}\DecValTok{2}

\NormalTok{omega1 }\OperatorTok{=} \DecValTok{120} \OperatorTok{*} \DecValTok{2} \OperatorTok{*}\NormalTok{ math.pi }\OperatorTok{/} \DecValTok{60}
\NormalTok{omega2 }\OperatorTok{=} \DecValTok{100} \OperatorTok{*} \DecValTok{2} \OperatorTok{*}\NormalTok{ math.pi }\OperatorTok{/} \DecValTok{60}

\NormalTok{K1 }\OperatorTok{=} \FloatTok{0.5} \OperatorTok{*}\NormalTok{ I }\OperatorTok{*}\NormalTok{ omega1}\OperatorTok{**}\DecValTok{2}
\NormalTok{K2 }\OperatorTok{=} \FloatTok{0.5} \OperatorTok{*}\NormalTok{ I }\OperatorTok{*}\NormalTok{ omega2}\OperatorTok{**}\DecValTok{2}
\NormalTok{dK }\OperatorTok{=}\NormalTok{ K1 }\OperatorTok{{-}}\NormalTok{ K2}

\BuiltInTok{print}\NormalTok{(}\SpecialStringTok{f"I (kg·m\^{}2):           }\SpecialCharTok{\{}\NormalTok{I}\SpecialCharTok{:.4f\}}\SpecialStringTok{"}\NormalTok{)}
\BuiltInTok{print}\NormalTok{(}\SpecialStringTok{f"KE at 120 rpm:        }\SpecialCharTok{\{}\NormalTok{K1}\OperatorTok{/}\FloatTok{1e3}\SpecialCharTok{:.4f\}}\SpecialStringTok{ kJ"}\NormalTok{)}
\BuiltInTok{print}\NormalTok{(}\SpecialStringTok{f"KE at 100 rpm:        }\SpecialCharTok{\{}\NormalTok{K2}\OperatorTok{/}\FloatTok{1e3}\SpecialCharTok{:.4f\}}\SpecialStringTok{ kJ"}\NormalTok{)}
\BuiltInTok{print}\NormalTok{(}\SpecialStringTok{f"Energy given out:     }\SpecialCharTok{\{}\NormalTok{dK}\OperatorTok{/}\FloatTok{1e3}\SpecialCharTok{:.4f\}}\SpecialStringTok{ kJ"}\NormalTok{)}
\end{Highlighting}
\end{Shaded}

\end{example}

\begin{example}[]\protect\hypertarget{exm-ex67-6-bird}{}\label{exm-ex67-6-bird}

A ship's anchor has a mass of 5 tonnes. Determine the work done in
raising the anchor from a depth of 100 m. If the hauling gear is driven
by a motor whose output is 80 kW and the efficiency of the haulage is
75\%, determine how long the lifting operation takes.

\begin{Shaded}
\begin{Highlighting}[]
\CommentTok{\# Given}

\NormalTok{m\_anchor }\OperatorTok{=} \DecValTok{5} \OperatorTok{*} \DecValTok{1000}        \CommentTok{\# tonnes to kg}
\NormalTok{depth }\OperatorTok{=} \DecValTok{100}                \CommentTok{\# m}
\NormalTok{g }\OperatorTok{=} \FloatTok{9.81}                   \CommentTok{\# m/s\^{}2}

\NormalTok{motor\_power }\OperatorTok{=} \DecValTok{80\_000}       \CommentTok{\# W}
\NormalTok{efficiency }\OperatorTok{=} \FloatTok{0.75}          \CommentTok{\# 75\%}

\CommentTok{\# Work done lifting the anchor}

\NormalTok{work\_done }\OperatorTok{=}\NormalTok{ m\_anchor }\OperatorTok{*}\NormalTok{ g }\OperatorTok{*}\NormalTok{ depth}

\CommentTok{\# Effective power available for lifting}

\NormalTok{useful\_power }\OperatorTok{=}\NormalTok{ motor\_power }\OperatorTok{*}\NormalTok{ efficiency}

\CommentTok{\# Time required}

\NormalTok{time\_seconds }\OperatorTok{=}\NormalTok{ work\_done }\OperatorTok{/}\NormalTok{ useful\_power}

\BuiltInTok{print}\NormalTok{(}\StringTok{"Work done in lifting the anchor (MJ): }\SpecialCharTok{\{:.4f\}}\StringTok{"}\NormalTok{.}\BuiltInTok{format}\NormalTok{(work\_done}\OperatorTok{/}\FloatTok{1e6}\NormalTok{))}
\BuiltInTok{print}\NormalTok{(}\StringTok{"Time required (s): }\SpecialCharTok{\{:.4f\}}\StringTok{"}\NormalTok{.}\BuiltInTok{format}\NormalTok{(time\_seconds))}
\end{Highlighting}
\end{Shaded}

\end{example}

\bookmarksetup{startatroot}

\chapter{Summary of Key Terms}\label{summary-of-key-terms}

\subsection{Inertia}\label{inertia-1}

The tendency of an object to resist any change in its motion. An object
with greater mass has greater inertia (it is ``harder to start or
stop'').

\begin{itemize}
\tightlist
\item
  In \textbf{Naval Architecture}, the \emph{moment of inertia} (units:
  m⁴) is a measure of resistance to bending (related to
  stiffness/deflection of beams and hull girders).
\item
  In \textbf{angular motion and physics}, the \emph{moment of inertia}
  (units: kg·m²) is a measure of resistance to angular acceleration
  (rotational inertia).
\end{itemize}

\subsection{Linear Momentum (p)}\label{linear-momentum-p}

Momentum is the ``quantity of motion'' an object has. It is calculated
as:

\[p = m v\]

where\\
- \(p\) = linear momentum (kg·m/s)\\
- \(m\) = mass (kg)\\
- \(v\) = velocity (m/s)

Momentum is a \textbf{vector quantity} (it has both magnitude and
direction).

\subsection{Newton's Laws of Motion}\label{newtons-laws-of-motion-1}

\subsubsection{First Law (Law of
Inertia)}\label{first-law-law-of-inertia}

An object will stay at rest or keep moving in a straight line at
constant speed unless an unbalanced (net) external force acts on it.

\subsubsection{Second Law (Law of
Acceleration)}\label{second-law-law-of-acceleration}

The acceleration of an object is directly proportional to the net force
acting on it and inversely proportional to its mass.

\[F_{\text{net}} = m a \quad \text{or} \quad a = \frac{F_{\text{net}}}{m}\]

\subsubsection{Third Law
(Action-Reaction)}\label{third-law-action-reaction}

Whenever two objects interact, they exert equal and opposite forces on
each other.\\
``For every action force, there is an equal and opposite reaction
force.''

\subsection{Conservation of Linear
Momentum}\label{conservation-of-linear-momentum-1}

If no external forces act on a system (or if the external forces cancel
out), the total momentum of the system stays constant.

In collisions or explosions:

\textbf{Total momentum before = Total momentum after}

Example: When two objects collide and stick together or bounce apart,
the momentum lost by one is exactly gained by the other, so the total
remains the same.

\subsection{Angular Momentum}\label{angular-momentum-1}

The rotational equivalent of linear momentum. It describes how much
``rotational motion'' a spinning or orbiting object has.

Angular momentum depends on mass, speed, and distance from the axis of
rotation.

\subsection{Radius of Gyration (k)}\label{radius-of-gyration-k}

A geometric property of a rigid body that indicates how the mass is
distributed relative to a specified axis of rotation. It is defined such
that:

\[I = m k^2\]

where\\
- \(I\) = moment of inertia (kg·m²)\\
- \(m\) = total mass (kg)\\
- \(k\) = radius of gyration (m)

\subsubsection{Practical Examples for Radius of
Gyration}\label{practical-examples-for-radius-of-gyration}

\begin{itemize}
\item
  \textbf{Smaller} \(k\) → mass is concentrated closer to the axis →
  \textbf{lower} rotational inertia\\
  → easier/faster to speed up or slow down rotation\\
  (e.g., engine crankshafts, turbine rotors, figure skater pulling arms
  in).
\item
  \textbf{Larger} \(k\) → mass is distributed farther from the axis →
  \textbf{higher} rotational inertia\\
  → better for storing rotational energy\\
  (e.g., flywheels; a hollow cylinder has a larger \(k\) than a solid
  cylinder of the same mass and outer radius).
\end{itemize}

\bookmarksetup{startatroot}

\chapter{Hydrostatics}\label{hydrostatics}

Hydrostatics is the branch of fluid mechanics that studies fluids at
rest and the forces and pressures they exert on immersed or containing
surfaces. Key principles include Pascal's law (pressure applied to an
enclosed fluid is transmitted undiminished in all directions),
Archimedes' principle (buoyant force on a submerged body equals the
weight of displaced fluid), and the hydrostatic pressure distribution in
a fluid column, expressed as \(p = \rho g h\), where \(p\) is gauge
pressure, \(\rho\) is fluid density, \(g\) is gravitational
acceleration, and \(h\) is depth below the free surface.

\section{Pressure Head}\label{pressure-head}

In fluid mechanics, pressure head is the height of a fluid column that
corresponds to a specific pressure at a given point, assuming the fluid
is static. It's expressed as \(h_p = \frac{p}{\rho g}\), where \(p\) is
the pressure above atmospheric, \(\rho\) is the fluid density, and \(g\)
is the acceleration due to gravity. The unit is typically meters (or
feet) of the fluid column. Pressure head represents the potential energy
per unit weight due to pressure and is a component of the total
hydraulic head, which is the sum of pressure head and elevation head.

\section{Load On Immersed Surfaces}\label{load-on-immersed-surfaces}

The load on an immersed surface is the total hydrostatic force exerted
by a fluid on a surface submerged in a static fluid. This force arises
from the pressure variation with depth, governed by the hydrostatic
pressure equation \(p = \rho g h\), where \(\rho\) is fluid density,
\(g\) is gravitational acceleration, and \(h\) is the depth below the
free surface.

The pressure acts perpendicular to the surface, resulting in a
distributed load. The total force \(F\) is calculated as
\(F = \rho g A \bar{h}\), where \(A\) is the surface area and
\(\bar{h}\) is the depth of the centroid of the area.

\begin{example}[]\protect\hypertarget{exm-reeds-p319}{}\label{exm-reeds-p319}

A vertical rectangular bulkhead is 7 m wide and extends over the full
height of the water column. Fresh water stands 6 m deep on one side and
is assumed to be at zero level on the other side. Calculate the total
hydrostatic load (thrust) on the bulkhead. Density of fresh water = 1000
kg/m³.

\begin{Shaded}
\begin{Highlighting}[]
\CommentTok{\# Given}
\NormalTok{width          }\OperatorTok{=} \FloatTok{7.0}    \CommentTok{\# m (horizontal width of bulkhead)}
\NormalTok{height\_water   }\OperatorTok{=} \FloatTok{6.0}    \CommentTok{\# m (water depth on one side)}
\NormalTok{rho\_water      }\OperatorTok{=} \FloatTok{1000.0} \CommentTok{\# kg/m³ (fresh water)}
\NormalTok{g              }\OperatorTok{=} \FloatTok{9.81}   \CommentTok{\# m/s²}

\CommentTok{\# Calculations}
\NormalTok{area           }\OperatorTok{=}\NormalTok{ width }\OperatorTok{*}\NormalTok{ height\_water                    }\CommentTok{\# m²}
\NormalTok{h\_centroid     }\OperatorTok{=}\NormalTok{ height\_water }\OperatorTok{/} \DecValTok{2}                        \CommentTok{\# m (from free surface)}
\NormalTok{force\_N        }\OperatorTok{=}\NormalTok{ rho\_water }\OperatorTok{*}\NormalTok{ g }\OperatorTok{*}\NormalTok{ h\_centroid }\OperatorTok{*}\NormalTok{ area       }\CommentTok{\# Newtons}
\NormalTok{force\_kN       }\OperatorTok{=}\NormalTok{ force\_N }\OperatorTok{/} \DecValTok{1000}


\BuiltInTok{print}\NormalTok{(}\StringTok{"=== Hydrostatic Load on Bulkhead ==="}\NormalTok{)}
\BuiltInTok{print}\NormalTok{(}\SpecialStringTok{f"Bulkhead width         : }\SpecialCharTok{\{}\NormalTok{width}\SpecialCharTok{\}}\SpecialStringTok{ m"}\NormalTok{)}
\BuiltInTok{print}\NormalTok{(}\SpecialStringTok{f"Water depth            : }\SpecialCharTok{\{}\NormalTok{height\_water}\SpecialCharTok{\}}\SpecialStringTok{ m"}\NormalTok{)}
\BuiltInTok{print}\NormalTok{(}\SpecialStringTok{f"Submerged area         : }\SpecialCharTok{\{}\NormalTok{area}\SpecialCharTok{:.1f\}}\SpecialStringTok{ m²"}\NormalTok{)}
\BuiltInTok{print}\NormalTok{(}\SpecialStringTok{f"Depth of centroid      : }\SpecialCharTok{\{}\NormalTok{h\_centroid}\SpecialCharTok{\}}\SpecialStringTok{ m"}\NormalTok{)}
\BuiltInTok{print}\NormalTok{(}\SpecialStringTok{f"Total water load       : }\SpecialCharTok{\{}\NormalTok{force\_kN}\SpecialCharTok{:,.0f\}}\SpecialStringTok{ kN  (}\SpecialCharTok{\{}\NormalTok{force\_kN}\OperatorTok{/}\DecValTok{1000}\SpecialCharTok{:.3f\}}\SpecialStringTok{ MN)"}\NormalTok{)}
\end{Highlighting}
\end{Shaded}

\end{example}

\begin{example}[]\protect\hypertarget{exm-reeds-p320}{}\label{exm-reeds-p320}

A 10-meter-long, 4-meter-wide, and 6-meter-high tank is filled with oil
(specific gravity = 0.9). The oil rises 5 meters up a vent pipe above
the tank's top. Calculate the load on one end plate and the bottom of
the tank.

\begin{Shaded}
\begin{Highlighting}[]
\CommentTok{\# {-}{-}{-}{-}{-}{-}{-}{-}{-}{-}{-}{-}{-}{-}{-}{-}{-}{-}{-}{-}{-}{-}{-}{-}{-}{-}{-}{-}{-}}
\CommentTok{\# Given data}
\CommentTok{\# {-}{-}{-}{-}{-}{-}{-}{-}{-}{-}{-}{-}{-}{-}{-}{-}{-}{-}{-}{-}{-}{-}{-}{-}{-}{-}{-}{-}{-}}
\NormalTok{tank\_length }\OperatorTok{=} \FloatTok{10.0}  \CommentTok{\# m}
\NormalTok{tank\_width }\OperatorTok{=} \FloatTok{4.0}  \CommentTok{\# m}
\NormalTok{tank\_height }\OperatorTok{=} \FloatTok{6.0}  \CommentTok{\# m}
\NormalTok{sounding\_pipe }\OperatorTok{=} \FloatTok{5.0}  \CommentTok{\# m (oil rises 5 m above the tank top in vent/pipe)}

\NormalTok{specific\_gravity }\OperatorTok{=} \FloatTok{0.9}
\NormalTok{rho\_oil }\OperatorTok{=}\NormalTok{ specific\_gravity }\OperatorTok{*} \DecValTok{1000}  \CommentTok{\# kg/m³}
\NormalTok{g }\OperatorTok{=} \FloatTok{9.81}  \CommentTok{\# m/s²}

\CommentTok{\# {-}{-}{-}{-}{-}{-}{-}{-}{-}{-}{-}{-}{-}{-}{-}{-}{-}{-}{-}{-}{-}{-}{-}{-}{-}{-}{-}{-}{-}}
\CommentTok{\# Calculated values}
\CommentTok{\# {-}{-}{-}{-}{-}{-}{-}{-}{-}{-}{-}{-}{-}{-}{-}{-}{-}{-}{-}{-}{-}{-}{-}{-}{-}{-}{-}{-}{-}}
\NormalTok{total\_head }\OperatorTok{=}\NormalTok{ tank\_height }\OperatorTok{+}\NormalTok{ sounding\_pipe  }\CommentTok{\# from bottom to oil surface}

\CommentTok{\# End plate dimensions and force}
\NormalTok{end\_height }\OperatorTok{=}\NormalTok{ tank\_height  }\CommentTok{\# m}
\NormalTok{end\_width }\OperatorTok{=}\NormalTok{ tank\_width  }\CommentTok{\# m}
\NormalTok{area\_end }\OperatorTok{=}\NormalTok{ end\_height }\OperatorTok{*}\NormalTok{ end\_width  }\CommentTok{\# m²}

\CommentTok{\# Depth of centroid of end plate below oil surface}
\NormalTok{h\_centroid\_end }\OperatorTok{=}\NormalTok{ sounding\_pipe }\OperatorTok{+}\NormalTok{ (tank\_height }\OperatorTok{/} \DecValTok{2}\NormalTok{)  }\CommentTok{\# m}

\CommentTok{\# Hydrostatic force on end plate}
\NormalTok{force\_end\_plate }\OperatorTok{=}\NormalTok{ rho\_oil }\OperatorTok{*}\NormalTok{ g }\OperatorTok{*}\NormalTok{ h\_centroid\_end }\OperatorTok{*}\NormalTok{ area\_end}

\CommentTok{\# Bottom plate force}
\NormalTok{area\_bottom }\OperatorTok{=}\NormalTok{ tank\_length }\OperatorTok{*}\NormalTok{ tank\_width  }\CommentTok{\# m²}
\NormalTok{force\_bottom }\OperatorTok{=}\NormalTok{ rho\_oil }\OperatorTok{*}\NormalTok{ g }\OperatorTok{*}\NormalTok{ total\_head }\OperatorTok{*}\NormalTok{ area\_bottom}

\CommentTok{\# {-}{-}{-}{-}{-}{-}{-}{-}{-}{-}{-}{-}{-}{-}{-}{-}{-}{-}{-}{-}{-}{-}{-}{-}{-}{-}{-}{-}{-}}
\CommentTok{\# Output results}
\CommentTok{\# {-}{-}{-}{-}{-}{-}{-}{-}{-}{-}{-}{-}{-}{-}{-}{-}{-}{-}{-}{-}{-}{-}{-}{-}{-}{-}{-}{-}{-}}
\BuiltInTok{print}\NormalTok{(}\StringTok{"=== Hydrostatic Loads on Oil Tank ==="}\NormalTok{)}
\BuiltInTok{print}\NormalTok{(}
    \SpecialStringTok{f"Tank dimensions       : }\SpecialCharTok{\{}\NormalTok{tank\_length}\SpecialCharTok{\}}\SpecialStringTok{ m × }\SpecialCharTok{\{}\NormalTok{tank\_width}\SpecialCharTok{\}}\SpecialStringTok{ m × "}
    \SpecialStringTok{f"}\SpecialCharTok{\{}\NormalTok{tank\_height}\SpecialCharTok{\}}\SpecialStringTok{ m (L×W×H)"}
\NormalTok{)}
\BuiltInTok{print}\NormalTok{(}\SpecialStringTok{f"Oil rise in sounding pipe : }\SpecialCharTok{\{}\NormalTok{sounding\_pipe}\SpecialCharTok{\}}\SpecialStringTok{ m"}\NormalTok{)}
\BuiltInTok{print}\NormalTok{(}\SpecialStringTok{f"Oil density           : }\SpecialCharTok{\{}\NormalTok{rho\_oil}\SpecialCharTok{\}}\SpecialStringTok{ kg/m³"}\NormalTok{)}
\BuiltInTok{print}\NormalTok{()}
\BuiltInTok{print}\NormalTok{(}
    \SpecialStringTok{f"End plate (one wall)  : }\SpecialCharTok{\{}\NormalTok{end\_height}\SpecialCharTok{\}}\SpecialStringTok{ m high × }\SpecialCharTok{\{}\NormalTok{end\_width}\SpecialCharTok{\}}\SpecialStringTok{ m "}
    \SpecialStringTok{f"wide = }\SpecialCharTok{\{}\NormalTok{area\_end}\SpecialCharTok{\}}\SpecialStringTok{ m²"}
\NormalTok{)}
\BuiltInTok{print}\NormalTok{(}
    \SpecialStringTok{f"Horizontal load on one end plate : }\SpecialCharTok{\{}\NormalTok{force\_end\_plate}\SpecialCharTok{:,.0f\}}\SpecialStringTok{ N = "}
    \SpecialStringTok{f"}\SpecialCharTok{\{}\NormalTok{force\_end\_plate}\OperatorTok{/}\DecValTok{1000}\SpecialCharTok{:.0f\}}\SpecialStringTok{ kN"}
\NormalTok{)}
\BuiltInTok{print}\NormalTok{()}
\BuiltInTok{print}\NormalTok{(}
    \SpecialStringTok{f"Bottom plate          : }\SpecialCharTok{\{}\NormalTok{tank\_length}\SpecialCharTok{\}}\SpecialStringTok{ m × }\SpecialCharTok{\{}\NormalTok{tank\_width}\SpecialCharTok{\}}\SpecialStringTok{ m = "}
    \SpecialStringTok{f"}\SpecialCharTok{\{}\NormalTok{area\_bottom}\SpecialCharTok{\}}\SpecialStringTok{ m²"}
\NormalTok{)}
\BuiltInTok{print}\NormalTok{(}
    \SpecialStringTok{f"Vertical load on bottom          : }\SpecialCharTok{\{}\NormalTok{force\_bottom}\SpecialCharTok{:,.0f\}}\SpecialStringTok{ N = "}
    \SpecialStringTok{f"}\SpecialCharTok{\{}\NormalTok{force\_bottom}\OperatorTok{/}\DecValTok{1000}\SpecialCharTok{:.0f\}}\SpecialStringTok{ kN"}
\NormalTok{)}
\end{Highlighting}
\end{Shaded}

\end{example}

\begin{example}[]\protect\hypertarget{exm-bolton-10}{}\label{exm-bolton-10}

A rectangular dock gate measures 4 meters wide. If the water level is
5.5 meters high on one side and 3.5 meters high on the other, what
horizontal thrust will act on the gate? Assume the water density is 1000
kg/m³.

\begin{Shaded}
\begin{Highlighting}[]

\CommentTok{\# Given}

\NormalTok{width }\OperatorTok{=} \FloatTok{4.0}          \CommentTok{\# gate width in meters}
\NormalTok{h1 }\OperatorTok{=} \FloatTok{5.5}             \CommentTok{\# water depth on higher side (m}
\NormalTok{h2 }\OperatorTok{=} \FloatTok{3.5}             \CommentTok{\# water depth on lower side (m)}
\NormalTok{rho }\OperatorTok{=} \FloatTok{1000.0}         \CommentTok{\# density of water (kg/m³)}
\NormalTok{g }\OperatorTok{=} \FloatTok{9.81}             \CommentTok{\# acceleration due to gravity (m/s²)}

\CommentTok{\# Hydrostatic force on higher side}
\NormalTok{A1 }\OperatorTok{=}\NormalTok{ h1 }\OperatorTok{*}\NormalTok{ width}
\NormalTok{h\_c1 }\OperatorTok{=}\NormalTok{ h1 }\OperatorTok{/} \DecValTok{2}
\NormalTok{F1 }\OperatorTok{=}\NormalTok{ rho }\OperatorTok{*}\NormalTok{ g }\OperatorTok{*}\NormalTok{ h\_c1 }\OperatorTok{*}\NormalTok{ A1}

\CommentTok{\# Hydrostatic force on lower side}
\NormalTok{A2 }\OperatorTok{=}\NormalTok{ h2 }\OperatorTok{*}\NormalTok{ width}
\NormalTok{h\_c2 }\OperatorTok{=}\NormalTok{ h2 }\OperatorTok{/} \DecValTok{2}
\NormalTok{F2 }\OperatorTok{=}\NormalTok{ rho }\OperatorTok{*}\NormalTok{ g }\OperatorTok{*}\NormalTok{ h\_c2 }\OperatorTok{*}\NormalTok{ A2}

\CommentTok{\# Net horizontal thrust (from higher to lower side)}
\NormalTok{F\_net }\OperatorTok{=}\NormalTok{ F1 }\OperatorTok{{-}}\NormalTok{ F2}

\BuiltInTok{print}\NormalTok{(}\StringTok{"=== Hydrostatic Thrust Calculation ==="}\NormalTok{)}
\BuiltInTok{print}\NormalTok{(}\SpecialStringTok{f"Force from higher side (5.5 m): }\SpecialCharTok{\{}\NormalTok{F1}\SpecialCharTok{:.0f\}}\SpecialStringTok{ N"}\NormalTok{)}
\BuiltInTok{print}\NormalTok{(}\SpecialStringTok{f"Force from lower side  (3.5 m): }\SpecialCharTok{\{}\NormalTok{F2}\SpecialCharTok{:.0f\}}\SpecialStringTok{ N"}\NormalTok{)}
\BuiltInTok{print}\NormalTok{(}\SpecialStringTok{f"Net horizontal thrust on gate: }\SpecialCharTok{\{}\NormalTok{F\_net}\SpecialCharTok{:.0f\}}\SpecialStringTok{ N"}\NormalTok{)}
\end{Highlighting}
\end{Shaded}

\end{example}

\begin{example}[]\protect\hypertarget{exm-bolton-11}{}\label{exm-bolton-11}

A rectangular dock gate measures 12 meters wide. When the sea is 9
meters deep on one side and 4.5 meters deep on the other, what
horizontal thrust will act on the gate? Assume the density of seawater
is 1025 kg/m³.

\begin{Shaded}
\begin{Highlighting}[]
\CommentTok{\# {-}{-}{-}{-}{-}{-}{-}{-}{-}{-}{-}{-}{-}{-}{-}{-}{-}{-}{-}{-}{-}{-}{-}{-}{-}{-}{-}{-}{-}}
\CommentTok{\# Given data}
\CommentTok{\# {-}{-}{-}{-}{-}{-}{-}{-}{-}{-}{-}{-}{-}{-}{-}{-}{-}{-}{-}{-}{-}{-}{-}{-}{-}{-}{-}{-}{-}}
\NormalTok{width }\OperatorTok{=} \FloatTok{12.0}         \CommentTok{\# width of gate (m)}
\NormalTok{h1 }\OperatorTok{=} \FloatTok{9.0}             \CommentTok{\# deeper side water depth (m)}
\NormalTok{h2 }\OperatorTok{=} \FloatTok{4.5}             \CommentTok{\# shallower side water depth (m)}
\NormalTok{rho }\OperatorTok{=} \FloatTok{1025.0}         \CommentTok{\# density of seawater (kg/m³)}
\NormalTok{g }\OperatorTok{=} \FloatTok{9.81}             \CommentTok{\# acceleration due to gravity (m/s²)}

\CommentTok{\# {-}{-}{-}{-}{-}{-}{-}{-}{-}{-}{-}{-}{-}{-}{-}{-}{-}{-}{-}{-}{-}{-}{-}{-}{-}{-}{-}{-}{-}}
\CommentTok{\# Hydrostatic force on deeper side}
\CommentTok{\# {-}{-}{-}{-}{-}{-}{-}{-}{-}{-}{-}{-}{-}{-}{-}{-}{-}{-}{-}{-}{-}{-}{-}{-}{-}{-}{-}{-}{-}}
\NormalTok{A1 }\OperatorTok{=}\NormalTok{ h1 }\OperatorTok{*}\NormalTok{ width           }\CommentTok{\# area of gate submerged}
\NormalTok{h\_c1 }\OperatorTok{=}\NormalTok{ h1 }\OperatorTok{/} \DecValTok{2}             \CommentTok{\# depth of centroid}
\NormalTok{F1 }\OperatorTok{=}\NormalTok{ rho }\OperatorTok{*}\NormalTok{ g }\OperatorTok{*}\NormalTok{ h\_c1 }\OperatorTok{*}\NormalTok{ A1  }\CommentTok{\# hydrostatic force}

\CommentTok{\# {-}{-}{-}{-}{-}{-}{-}{-}{-}{-}{-}{-}{-}{-}{-}{-}{-}{-}{-}{-}{-}{-}{-}{-}{-}{-}{-}{-}{-}}
\CommentTok{\# Hydrostatic force on shallower side}
\CommentTok{\# {-}{-}{-}{-}{-}{-}{-}{-}{-}{-}{-}{-}{-}{-}{-}{-}{-}{-}{-}{-}{-}{-}{-}{-}{-}{-}{-}{-}{-}}
\NormalTok{A2 }\OperatorTok{=}\NormalTok{ h2 }\OperatorTok{*}\NormalTok{ width           }\CommentTok{\# area of gate submerged}
\NormalTok{h\_c2 }\OperatorTok{=}\NormalTok{ h2 }\OperatorTok{/} \DecValTok{2}             \CommentTok{\# depth of centroid}
\NormalTok{F2 }\OperatorTok{=}\NormalTok{ rho }\OperatorTok{*}\NormalTok{ g }\OperatorTok{*}\NormalTok{ h\_c2 }\OperatorTok{*}\NormalTok{ A2  }\CommentTok{\# hydrostatic force}

\CommentTok{\# {-}{-}{-}{-}{-}{-}{-}{-}{-}{-}{-}{-}{-}{-}{-}{-}{-}{-}{-}{-}{-}{-}{-}{-}{-}{-}{-}{-}{-}}
\CommentTok{\# Net horizontal thrust}
\CommentTok{\# {-}{-}{-}{-}{-}{-}{-}{-}{-}{-}{-}{-}{-}{-}{-}{-}{-}{-}{-}{-}{-}{-}{-}{-}{-}{-}{-}{-}{-}}
\NormalTok{F\_net }\OperatorTok{=}\NormalTok{ F1 }\OperatorTok{{-}}\NormalTok{ F2}

\CommentTok{\# {-}{-}{-}{-}{-}{-}{-}{-}{-}{-}{-}{-}{-}{-}{-}{-}{-}{-}{-}{-}{-}{-}{-}{-}{-}{-}{-}{-}{-}}
\CommentTok{\# Output results}
\CommentTok{\# {-}{-}{-}{-}{-}{-}{-}{-}{-}{-}{-}{-}{-}{-}{-}{-}{-}{-}{-}{-}{-}{-}{-}{-}{-}{-}{-}{-}{-}}
\BuiltInTok{print}\NormalTok{(}\StringTok{"=== Dock Gate Hydrostatic Thrust Calculation ==="}\NormalTok{)}
\BuiltInTok{print}\NormalTok{(}\SpecialStringTok{f"Force from }\SpecialCharTok{\{}\NormalTok{h1}\SpecialCharTok{\}}\SpecialStringTok{ m side : }\SpecialCharTok{\{}\NormalTok{F1}\SpecialCharTok{:,.0f\}}\SpecialStringTok{ N  (}\SpecialCharTok{\{}\NormalTok{F1}\OperatorTok{/}\FloatTok{1e6}\SpecialCharTok{:.3f\}}\SpecialStringTok{ MN)"}\NormalTok{)}
\BuiltInTok{print}\NormalTok{(}\SpecialStringTok{f"Force from }\SpecialCharTok{\{}\NormalTok{h2}\SpecialCharTok{\}}\SpecialStringTok{ m side : }\SpecialCharTok{\{}\NormalTok{F2}\SpecialCharTok{:,.0f\}}\SpecialStringTok{ N  (}\SpecialCharTok{\{}\NormalTok{F2}\OperatorTok{/}\FloatTok{1e6}\SpecialCharTok{:.3f\}}\SpecialStringTok{ MN)"}\NormalTok{)}
\BuiltInTok{print}\NormalTok{(}
    \SpecialStringTok{f"Resultant horizontal thrust: }\SpecialCharTok{\{}\NormalTok{F\_net}\SpecialCharTok{:,.0f\}}\SpecialStringTok{ N  "}
    \SpecialStringTok{f"(}\SpecialCharTok{\{}\NormalTok{F\_net}\OperatorTok{/}\DecValTok{1000}\SpecialCharTok{:.1f\}}\SpecialStringTok{ kN or }\SpecialCharTok{\{}\NormalTok{F\_net}\OperatorTok{/}\FloatTok{1e6}\SpecialCharTok{:.3f\}}\SpecialStringTok{ MN)"}
\NormalTok{)}
\end{Highlighting}
\end{Shaded}

\end{example}

\begin{example}[]\protect\hypertarget{exm-fe2}{}\label{exm-fe2}

A vertical rectangular bulkhead, 2 meters wide, divides two liquids in a
tank. On one side, oil with a density of 850 kg/m³ stands 4 meters deep.
On the other side, fresh water with a density of 1000 kg/m³ is 6 meters
deep. The bottom of the bulkhead rests on the tank bottom, and both
liquids are open to the atmosphere at the top, resulting in zero gauge
pressure at the free surfaces. Calculate the magnitude and direction of
the net hydrostatic thrust on the bulkhead.

\begin{Shaded}
\begin{Highlighting}[]
\CommentTok{\# {-}{-}{-}{-}{-}{-}{-}{-}{-}{-}{-}{-}{-}{-}{-}{-}{-}{-}{-}{-}{-}{-}{-}{-}{-}{-}{-}{-}{-}}
\CommentTok{\# Given data}
\CommentTok{\# {-}{-}{-}{-}{-}{-}{-}{-}{-}{-}{-}{-}{-}{-}{-}{-}{-}{-}{-}{-}{-}{-}{-}{-}{-}{-}{-}{-}{-}}
\NormalTok{width }\OperatorTok{=} \FloatTok{2.0}  \CommentTok{\# m (width of bulkhead)}
\NormalTok{h\_oil }\OperatorTok{=} \FloatTok{4.0}  \CommentTok{\# m (oil depth)}
\NormalTok{h\_water }\OperatorTok{=} \FloatTok{6.0}  \CommentTok{\# m (water depth)}
\NormalTok{rho\_oil }\OperatorTok{=} \FloatTok{850.0}  \CommentTok{\# kg/m³}
\NormalTok{rho\_water }\OperatorTok{=} \FloatTok{1000.0}  \CommentTok{\# kg/m³}
\NormalTok{g }\OperatorTok{=} \FloatTok{9.81}  \CommentTok{\# m/s²}

\CommentTok{\# {-}{-}{-}{-}{-}{-}{-}{-}{-}{-}{-}{-}{-}{-}{-}{-}{-}{-}{-}{-}{-}{-}{-}{-}{-}{-}{-}{-}{-}}
\CommentTok{\# Display input parameters}
\CommentTok{\# {-}{-}{-}{-}{-}{-}{-}{-}{-}{-}{-}{-}{-}{-}{-}{-}{-}{-}{-}{-}{-}{-}{-}{-}{-}{-}{-}{-}{-}}
\BuiltInTok{print}\NormalTok{(}\StringTok{"Bulkhead Separating Oil and Water"}\NormalTok{)}
\BuiltInTok{print}\NormalTok{(}\SpecialStringTok{f"Bulkhead width          : }\SpecialCharTok{\{}\NormalTok{width}\SpecialCharTok{\}}\SpecialStringTok{ m"}\NormalTok{)}
\BuiltInTok{print}\NormalTok{(}\SpecialStringTok{f"Oil side    : }\SpecialCharTok{\{}\NormalTok{h\_oil}\SpecialCharTok{\}}\SpecialStringTok{ m deep,   ρ = }\SpecialCharTok{\{}\NormalTok{rho\_oil}\SpecialCharTok{\}}\SpecialStringTok{ kg/m³"}\NormalTok{)}
\BuiltInTok{print}\NormalTok{(}\SpecialStringTok{f"Water side  : }\SpecialCharTok{\{}\NormalTok{h\_water}\SpecialCharTok{\}}\SpecialStringTok{ m deep,   ρ = }\SpecialCharTok{\{}\NormalTok{rho\_water}\SpecialCharTok{\}}\SpecialStringTok{ kg/m³"}\NormalTok{)}
\BuiltInTok{print}\NormalTok{()}

\CommentTok{\# {-}{-}{-}{-}{-}{-}{-}{-}{-}{-}{-}{-}{-}{-}{-}{-}{-}{-}{-}{-}{-}{-}{-}{-}{-}{-}{-}{-}{-}}
\CommentTok{\# Hydrostatic force on oil side}
\CommentTok{\# {-}{-}{-}{-}{-}{-}{-}{-}{-}{-}{-}{-}{-}{-}{-}{-}{-}{-}{-}{-}{-}{-}{-}{-}{-}{-}{-}{-}{-}}
\NormalTok{A\_oil }\OperatorTok{=}\NormalTok{ width }\OperatorTok{*}\NormalTok{ h\_oil}
\NormalTok{h\_c\_oil }\OperatorTok{=}\NormalTok{ h\_oil }\OperatorTok{/} \DecValTok{2}
\NormalTok{F\_oil }\OperatorTok{=}\NormalTok{ rho\_oil }\OperatorTok{*}\NormalTok{ g }\OperatorTok{*}\NormalTok{ h\_c\_oil }\OperatorTok{*}\NormalTok{ A\_oil}

\CommentTok{\# {-}{-}{-}{-}{-}{-}{-}{-}{-}{-}{-}{-}{-}{-}{-}{-}{-}{-}{-}{-}{-}{-}{-}{-}{-}{-}{-}{-}{-}}
\CommentTok{\# Hydrostatic force on water side}
\CommentTok{\# {-}{-}{-}{-}{-}{-}{-}{-}{-}{-}{-}{-}{-}{-}{-}{-}{-}{-}{-}{-}{-}{-}{-}{-}{-}{-}{-}{-}{-}}
\NormalTok{A\_water }\OperatorTok{=}\NormalTok{ width }\OperatorTok{*}\NormalTok{ h\_water}
\NormalTok{h\_c\_water }\OperatorTok{=}\NormalTok{ h\_water }\OperatorTok{/} \DecValTok{2}
\NormalTok{F\_water }\OperatorTok{=}\NormalTok{ rho\_water }\OperatorTok{*}\NormalTok{ g }\OperatorTok{*}\NormalTok{ h\_c\_water }\OperatorTok{*}\NormalTok{ A\_water}

\CommentTok{\# {-}{-}{-}{-}{-}{-}{-}{-}{-}{-}{-}{-}{-}{-}{-}{-}{-}{-}{-}{-}{-}{-}{-}{-}{-}{-}{-}{-}{-}}
\CommentTok{\# Net horizontal thrust}
\CommentTok{\# {-}{-}{-}{-}{-}{-}{-}{-}{-}{-}{-}{-}{-}{-}{-}{-}{-}{-}{-}{-}{-}{-}{-}{-}{-}{-}{-}{-}{-}}
\NormalTok{F\_net\_N }\OperatorTok{=}\NormalTok{ F\_water }\OperatorTok{{-}}\NormalTok{ F\_oil  }\CommentTok{\# positive towards oil side}
\NormalTok{F\_net\_kN }\OperatorTok{=}\NormalTok{ F\_net\_N }\OperatorTok{/} \DecValTok{1000}

\CommentTok{\# {-}{-}{-}{-}{-}{-}{-}{-}{-}{-}{-}{-}{-}{-}{-}{-}{-}{-}{-}{-}{-}{-}{-}{-}{-}{-}{-}{-}{-}}
\CommentTok{\# Output results}
\CommentTok{\# {-}{-}{-}{-}{-}{-}{-}{-}{-}{-}{-}{-}{-}{-}{-}{-}{-}{-}{-}{-}{-}{-}{-}{-}{-}{-}{-}{-}{-}}
\BuiltInTok{print}\NormalTok{(}\SpecialStringTok{f"Force from oil side     : }\SpecialCharTok{\{}\NormalTok{F\_oil}\SpecialCharTok{:8.0f\}}\SpecialStringTok{ N  = }\SpecialCharTok{\{}\NormalTok{F\_oil}\OperatorTok{/}\DecValTok{1000}\SpecialCharTok{:6.1f\}}\SpecialStringTok{ kN"}\NormalTok{)}
\BuiltInTok{print}\NormalTok{(}\SpecialStringTok{f"Force from water side   : }\SpecialCharTok{\{}\NormalTok{F\_water}\SpecialCharTok{:8.0f\}}\SpecialStringTok{ N  = }\SpecialCharTok{\{}\NormalTok{F\_water}\OperatorTok{/}\DecValTok{1000}\SpecialCharTok{:6.1f\}}\SpecialStringTok{ kN"}\NormalTok{)}
\BuiltInTok{print}\NormalTok{()}
\BuiltInTok{print}\NormalTok{(}\SpecialStringTok{f"NET THRUST              : }\SpecialCharTok{\{}\NormalTok{F\_net\_N}\SpecialCharTok{:8.0f\}}\SpecialStringTok{ N  = }\SpecialCharTok{\{}\NormalTok{F\_net\_kN}\SpecialCharTok{:6.1f\}}\SpecialStringTok{ kN"}\NormalTok{)}
\BuiltInTok{print}\NormalTok{(}\StringTok{"Direction               : Towards the oil side"}\NormalTok{)}
\end{Highlighting}
\end{Shaded}

\end{example}

\begin{example}[]\protect\hypertarget{exm-assignment29}{}\label{exm-assignment29}

Seawater has flooded an oil tanker's oil tank to a depth of 4 meters,
and a 10-meter layer of oil sits atop the seawater. Calculate the
pressure at the bottom of the tank. Note: The density of oil is 0.85
g/cm³, the density of seawater is 1.02 t/m³, and the atmospheric
pressure is 101.3 kPa.

\begin{Shaded}
\begin{Highlighting}[]
\CommentTok{\# {-}{-}{-}{-}{-}{-}{-}{-}{-}{-}{-}{-}{-}{-}{-}{-}{-}{-}{-}{-}{-}{-}{-}{-}{-}{-}{-}{-}{-}}
\CommentTok{\# Given data}
\CommentTok{\# {-}{-}{-}{-}{-}{-}{-}{-}{-}{-}{-}{-}{-}{-}{-}{-}{-}{-}{-}{-}{-}{-}{-}{-}{-}{-}{-}{-}{-}}
\NormalTok{oil\_density\_g\_per\_cm3 }\OperatorTok{=} \FloatTok{0.85}  \CommentTok{\# Density of oil in g/cm³}
\NormalTok{oil\_height\_m }\OperatorTok{=} \FloatTok{10.0}  \CommentTok{\# Height of oil layer in meters}
\NormalTok{seawater\_density\_t\_per\_m3 }\OperatorTok{=} \FloatTok{1.02}  \CommentTok{\# Density of seawater in t/m³}
\NormalTok{seawater\_height\_m }\OperatorTok{=} \FloatTok{4.0}  \CommentTok{\# Height of seawater layer in meters}
\NormalTok{atmospheric\_pressure\_kPa }\OperatorTok{=} \FloatTok{101.3}  \CommentTok{\# Atmospheric pressure in kPa}
\NormalTok{g }\OperatorTok{=} \FloatTok{9.81}  \CommentTok{\# Acceleration due to gravity (m/s²)}

\CommentTok{\# {-}{-}{-}{-}{-}{-}{-}{-}{-}{-}{-}{-}{-}{-}{-}{-}{-}{-}{-}{-}{-}{-}{-}{-}{-}{-}{-}{-}{-}}
\CommentTok{\# Convert densities to kg/m³}
\CommentTok{\# {-}{-}{-}{-}{-}{-}{-}{-}{-}{-}{-}{-}{-}{-}{-}{-}{-}{-}{-}{-}{-}{-}{-}{-}{-}{-}{-}{-}{-}}
\NormalTok{oil\_density }\OperatorTok{=}\NormalTok{ oil\_density\_g\_per\_cm3 }\OperatorTok{*} \DecValTok{1000}  \CommentTok{\# g/cm³ → kg/m³}
\NormalTok{seawater\_density }\OperatorTok{=}\NormalTok{ seawater\_density\_t\_per\_m3 }\OperatorTok{*} \DecValTok{1000}  \CommentTok{\# t/m³ → kg/m³}

\CommentTok{\# {-}{-}{-}{-}{-}{-}{-}{-}{-}{-}{-}{-}{-}{-}{-}{-}{-}{-}{-}{-}{-}{-}{-}{-}{-}{-}{-}{-}{-}}
\CommentTok{\# Hydrostatic pressure calculations}
\CommentTok{\# {-}{-}{-}{-}{-}{-}{-}{-}{-}{-}{-}{-}{-}{-}{-}{-}{-}{-}{-}{-}{-}{-}{-}{-}{-}{-}{-}{-}{-}}
\NormalTok{pressure\_oil\_kPa }\OperatorTok{=}\NormalTok{ (oil\_density }\OperatorTok{*}\NormalTok{ g }\OperatorTok{*}\NormalTok{ oil\_height\_m) }\OperatorTok{/} \DecValTok{1000}
\NormalTok{pressure\_seawater\_kPa }\OperatorTok{=}\NormalTok{ (seawater\_density }\OperatorTok{*}\NormalTok{ g }\OperatorTok{*}\NormalTok{ seawater\_height\_m) }\OperatorTok{/} \DecValTok{1000}

\CommentTok{\# Total gauge and absolute pressure}
\NormalTok{gauge\_pressure\_kPa }\OperatorTok{=}\NormalTok{ pressure\_oil\_kPa }\OperatorTok{+}\NormalTok{ pressure\_seawater\_kPa}
\NormalTok{absolute\_pressure\_kPa }\OperatorTok{=}\NormalTok{ atmospheric\_pressure\_kPa }\OperatorTok{+}\NormalTok{ gauge\_pressure\_kPa}

\CommentTok{\# {-}{-}{-}{-}{-}{-}{-}{-}{-}{-}{-}{-}{-}{-}{-}{-}{-}{-}{-}{-}{-}{-}{-}{-}{-}{-}{-}{-}{-}}
\CommentTok{\# Output results}
\CommentTok{\# {-}{-}{-}{-}{-}{-}{-}{-}{-}{-}{-}{-}{-}{-}{-}{-}{-}{-}{-}{-}{-}{-}{-}{-}{-}{-}{-}{-}{-}}
\BuiltInTok{print}\NormalTok{(}\SpecialStringTok{f"Pressure due to oil layer       : }\SpecialCharTok{\{}\NormalTok{pressure\_oil\_kPa}\SpecialCharTok{:.4f\}}\SpecialStringTok{ kPa"}\NormalTok{)}
\BuiltInTok{print}\NormalTok{(}\SpecialStringTok{f"Pressure due to seawater layer  : }\SpecialCharTok{\{}\NormalTok{pressure\_seawater\_kPa}\SpecialCharTok{:.4f\}}\SpecialStringTok{ kPa"}\NormalTok{)}
\BuiltInTok{print}\NormalTok{(}\SpecialStringTok{f"Total gauge pressure            : }\SpecialCharTok{\{}\NormalTok{gauge\_pressure\_kPa}\SpecialCharTok{:.4f\}}\SpecialStringTok{ kPa"}\NormalTok{)}
\BuiltInTok{print}\NormalTok{(}\SpecialStringTok{f"Absolute pressure at the bottom : }\SpecialCharTok{\{}\NormalTok{absolute\_pressure\_kPa}\SpecialCharTok{:.4f\}}\SpecialStringTok{ kPa"}\NormalTok{)}
\end{Highlighting}
\end{Shaded}

\end{example}

\section{Hydraulic Jacks}\label{hydraulic-jacks}

Hydraulic jacks use Pascal's law to lift heavy loads with minimal input
force. Pascal's law states that pressure applied to an enclosed fluid is
transmitted equally in all directions. In a hydraulic jack, two pistons
of different sizes are connected by a confined fluid. Applying an effort
force to the smaller piston generates a pressure that is transmitted to
the larger piston, producing an output force. Calculations involving
hydraulic jacks, such as determining system pressure, required input
force, or lifted load, rely on this principle under ideal conditions,
neglecting friction and fluid compressibility.

\begin{example}[]\protect\hypertarget{exm-hydraulic-jack-hajer-p2}{}\label{exm-hydraulic-jack-hajer-p2}

A force of 150 N is transmitted from a piston of 25 mm² to one of 100
mm². Determine the system pressure and the load carried by the larger
piston.

\begin{Shaded}
\begin{Highlighting}[]
\CommentTok{\# {-}{-}{-}{-}{-}{-}{-}{-}{-}{-}{-}{-}{-}{-}{-}{-}{-}{-}{-}{-}{-}{-}{-}{-}{-}{-}{-}{-}{-}}
\CommentTok{\# Given parameters}
\CommentTok{\# {-}{-}{-}{-}{-}{-}{-}{-}{-}{-}{-}{-}{-}{-}{-}{-}{-}{-}{-}{-}{-}{-}{-}{-}{-}{-}{-}{-}{-}}
\NormalTok{area\_small\_mm2 }\OperatorTok{=} \FloatTok{25.0}  \CommentTok{\# mm² (small piston)}
\NormalTok{area\_large\_mm2 }\OperatorTok{=} \FloatTok{100.0}  \CommentTok{\# mm² (large piston)}
\NormalTok{force\_small }\OperatorTok{=} \FloatTok{150.0}  \CommentTok{\# N (force applied to small piston)}

\CommentTok{\# {-}{-}{-}{-}{-}{-}{-}{-}{-}{-}{-}{-}{-}{-}{-}{-}{-}{-}{-}{-}{-}{-}{-}{-}{-}{-}{-}{-}{-}}
\CommentTok{\# Convert areas to SI units (m²)}
\CommentTok{\# {-}{-}{-}{-}{-}{-}{-}{-}{-}{-}{-}{-}{-}{-}{-}{-}{-}{-}{-}{-}{-}{-}{-}{-}{-}{-}{-}{-}{-}}
\NormalTok{area\_small }\OperatorTok{=}\NormalTok{ area\_small\_mm2 }\OperatorTok{*} \FloatTok{1e{-}6}  \CommentTok{\# 1 mm² = 10\^{}{-}6 m²}
\NormalTok{area\_large }\OperatorTok{=}\NormalTok{ area\_large\_mm2 }\OperatorTok{*} \FloatTok{1e{-}6}

\CommentTok{\# {-}{-}{-}{-}{-}{-}{-}{-}{-}{-}{-}{-}{-}{-}{-}{-}{-}{-}{-}{-}{-}{-}{-}{-}{-}{-}{-}{-}{-}}
\CommentTok{\# System pressure and load on large piston}
\CommentTok{\# {-}{-}{-}{-}{-}{-}{-}{-}{-}{-}{-}{-}{-}{-}{-}{-}{-}{-}{-}{-}{-}{-}{-}{-}{-}{-}{-}{-}{-}}
\NormalTok{pressure }\OperatorTok{=}\NormalTok{ force\_small }\OperatorTok{/}\NormalTok{ area\_small  }\CommentTok{\# Pa}
\NormalTok{load\_large }\OperatorTok{=}\NormalTok{ pressure }\OperatorTok{*}\NormalTok{ area\_large  }\CommentTok{\# N}

\CommentTok{\# {-}{-}{-}{-}{-}{-}{-}{-}{-}{-}{-}{-}{-}{-}{-}{-}{-}{-}{-}{-}{-}{-}{-}{-}{-}{-}{-}{-}{-}}
\CommentTok{\# Output results}
\CommentTok{\# {-}{-}{-}{-}{-}{-}{-}{-}{-}{-}{-}{-}{-}{-}{-}{-}{-}{-}{-}{-}{-}{-}{-}{-}{-}{-}{-}{-}{-}}
\BuiltInTok{print}\NormalTok{(}\SpecialStringTok{f"System pressure: }\SpecialCharTok{\{}\NormalTok{pressure}\SpecialCharTok{:.2f\}}\SpecialStringTok{ Pa "} \SpecialStringTok{f"(or }\SpecialCharTok{\{}\NormalTok{pressure }\OperatorTok{/} \FloatTok{1e6}\SpecialCharTok{:.1f\}}\SpecialStringTok{ MPa)"}\NormalTok{)}
\BuiltInTok{print}\NormalTok{(}\SpecialStringTok{f"Load carried by larger piston: }\SpecialCharTok{\{}\NormalTok{load\_large}\SpecialCharTok{:.1f\}}\SpecialStringTok{ N"}\NormalTok{)}
\end{Highlighting}
\end{Shaded}

\end{example}

\begin{example}[]\protect\hypertarget{exm-hydraulic-jack-ex8-5}{}\label{exm-hydraulic-jack-ex8-5}

A simple hydraulic jack consists of a small effort plunger with a
diameter of 25 mm and a large load piston with a diameter of 70 mm. The
jack lifts a load of 5.88 kN. Calculate:

\begin{enumerate}
\def\labelenumi{\arabic{enumi}.}
\item
  The pressure in the hydraulic fluid (system pressure).
\item
  The force exerted on the small plunger (load on the small plunger)
  required to support the given load, assuming ideal conditions (no
  losses).
\end{enumerate}

\begin{Shaded}
\begin{Highlighting}[]
\ImportTok{import}\NormalTok{ math}

\CommentTok{\# {-}{-}{-}{-}{-}{-}{-}{-}{-}{-}{-}{-}{-}{-}{-}{-}{-}{-}{-}{-}{-}{-}{-}{-}{-}{-}{-}{-}{-}}
\CommentTok{\# Given data}
\CommentTok{\# {-}{-}{-}{-}{-}{-}{-}{-}{-}{-}{-}{-}{-}{-}{-}{-}{-}{-}{-}{-}{-}{-}{-}{-}{-}{-}{-}{-}{-}}
\NormalTok{diameter\_small }\OperatorTok{=} \FloatTok{0.025}  \CommentTok{\# m (effort plunger)}
\NormalTok{diameter\_large }\OperatorTok{=} \FloatTok{0.070}  \CommentTok{\# m (load piston)}
\NormalTok{load }\OperatorTok{=} \FloatTok{5880.0}  \CommentTok{\# N (equivalent to 5.88 kN)}

\CommentTok{\# {-}{-}{-}{-}{-}{-}{-}{-}{-}{-}{-}{-}{-}{-}{-}{-}{-}{-}{-}{-}{-}{-}{-}{-}{-}{-}{-}{-}{-}}
\CommentTok{\# Calculate radii and cross{-}sectional areas}
\CommentTok{\# {-}{-}{-}{-}{-}{-}{-}{-}{-}{-}{-}{-}{-}{-}{-}{-}{-}{-}{-}{-}{-}{-}{-}{-}{-}{-}{-}{-}{-}}
\NormalTok{radius\_small }\OperatorTok{=}\NormalTok{ diameter\_small }\OperatorTok{/} \DecValTok{2}
\NormalTok{radius\_large }\OperatorTok{=}\NormalTok{ diameter\_large }\OperatorTok{/} \DecValTok{2}

\NormalTok{area\_small }\OperatorTok{=}\NormalTok{ math.pi }\OperatorTok{*}\NormalTok{ (radius\_small}\OperatorTok{**}\DecValTok{2}\NormalTok{)  }\CommentTok{\# m²}
\NormalTok{area\_large }\OperatorTok{=}\NormalTok{ math.pi }\OperatorTok{*}\NormalTok{ (radius\_large}\OperatorTok{**}\DecValTok{2}\NormalTok{)  }\CommentTok{\# m²}

\CommentTok{\# {-}{-}{-}{-}{-}{-}{-}{-}{-}{-}{-}{-}{-}{-}{-}{-}{-}{-}{-}{-}{-}{-}{-}{-}{-}{-}{-}{-}{-}}
\CommentTok{\# System pressure and ideal force on small plunger}
\CommentTok{\# {-}{-}{-}{-}{-}{-}{-}{-}{-}{-}{-}{-}{-}{-}{-}{-}{-}{-}{-}{-}{-}{-}{-}{-}{-}{-}{-}{-}{-}}
\NormalTok{pressure }\OperatorTok{=}\NormalTok{ load }\OperatorTok{/}\NormalTok{ area\_large  }\CommentTok{\# Pa}
\NormalTok{force\_small }\OperatorTok{=}\NormalTok{ pressure }\OperatorTok{*}\NormalTok{ area\_small  }\CommentTok{\# N}

\CommentTok{\# {-}{-}{-}{-}{-}{-}{-}{-}{-}{-}{-}{-}{-}{-}{-}{-}{-}{-}{-}{-}{-}{-}{-}{-}{-}{-}{-}{-}{-}}
\CommentTok{\# Output results}
\CommentTok{\# {-}{-}{-}{-}{-}{-}{-}{-}{-}{-}{-}{-}{-}{-}{-}{-}{-}{-}{-}{-}{-}{-}{-}{-}{-}{-}{-}{-}{-}}
\BuiltInTok{print}\NormalTok{(}\SpecialStringTok{f"System pressure: }\SpecialCharTok{\{}\NormalTok{pressure}\SpecialCharTok{:.2f\}}\SpecialStringTok{ Pa "} \SpecialStringTok{f"(or }\SpecialCharTok{\{}\NormalTok{pressure }\OperatorTok{/} \FloatTok{1e6}\SpecialCharTok{:.3f\}}\SpecialStringTok{ MPa)"}\NormalTok{)}
\BuiltInTok{print}\NormalTok{(}\SpecialStringTok{f"Ideal force on small plunger: }\SpecialCharTok{\{}\NormalTok{force\_small}\SpecialCharTok{:.1f\}}\SpecialStringTok{ N"}\NormalTok{)}
\end{Highlighting}
\end{Shaded}

\end{example}

\bookmarksetup{startatroot}

\chapter{Hydrodynamics}\label{sec-hydrodynamics}

Hydrodynamics, a branch of fluid mechanics, studies the motion of
liquids and the forces acting on them. Unlike aerodynamics, which
concerns compressible gases such as air, hydrodynamics primarily deals
with incompressible fluids like water. It is based on classical physics
principles, including Newton's laws of motion, the conservation of mass
and momentum. These principles are used to describe the velocity,
pressure and viscosity of a fluid in motion.

\section{Volume Flow Rate}\label{volume-flow-rate}

Volume rate of flow refers to the rate at which a fluid volume passes a
given section in a flow stream. Volume rate of flow may also be referred
to as capacity of flow, flow rate or discharge. It is generally given in
units of volume per unit of time, cubic meters per second
(m\textsuperscript{3}/s) or liters per second (L/s).

\begin{figure}[H]

{\centering \includegraphics[width=10.4cm,height=\textheight,keepaspectratio]{images/volumeflow.png}

}

\caption{Volume flow}

\end{figure}%

Consider an ideal fluid flow in a pipe of cross-section `A'
(m\textsuperscript{2}). For an ideal fluid all the particles move to the
right with a velocity of `C' (m/s). As the fluid flow to the right with
a velocity of C m/s, `C' m of fluid pass section 1 every second.
i.e.~the volume of fluid passing section-1 every second is A x C.

\[
\dot{v} = A \cdot C
\]

Where:

\begin{itemize}
\item
  \(\dot{v}\): Volume flow rate, \(\mathrm{m^3/s}\)
\item
  \(A\): Cross-sectional area of the flow, \(\mathrm{m^2}\)
\item
  \(C\): Mean (average) velocity of the fluid, \(\mathrm{m/s}\)
\end{itemize}

Consider a real fluid flowing in the same pipe. Here particle velocity
will be a variable across the flow stream.

\begin{figure}[H]

{\centering \includegraphics[width=11.2cm,height=\textheight,keepaspectratio]{images/volumeflow1.png}

}

\caption{Mean velocity}

\end{figure}%

If the mean velocity cm of all the fluid particles could be found, then
similar to the ideal fluid.

\[
Volume\ flow = area\ x\ mean\ velocity
\]

Unless otherwise indicated, the velocity of flow is understood to refer
to the average or mean velocity of all the particles in a flowing fluid.

\section{Mass Flow Rate}\label{mass-flow-rate}

Mass flow rate refers to the rate at which a fluid mass passes a given
section in a flow stream. It is given in units of mass per unit of time,
kilogram per second (kg/s). Mass flow rate is easily calculated from
volume flow rate as follows.

\[
\dot{m} = \rho \cdot \dot v
\]

Where:

\begin{itemize}
\item
  \(\dot{m}\): mass flow rate, \(\mathrm{kg/s}\)
\item
  \(\rho\): density, \(\mathrm{kg/m^3}\)
\item
  \(\dot{v}\): volume flow, \(\mathrm{m^3/s}\)
\end{itemize}

Or

\[
\dot{m} = \rho \cdot A \cdot C
\]

Where:

\begin{itemize}
\item
  \(\dot{m}\): mass flow rate, \(\mathrm{kg/s}\)
\item
  \(\rho\): density, \(\mathrm{kg/m^3}\)
\item
  \(A\): Cross-sectional area of the flow, \(\mathrm{m^2}\)
\item
  \(C\): Mean (average) velocity of the fluid, \(\mathrm{m/s}\)
\end{itemize}

\begin{example}[]\protect\hypertarget{exm-13-1}{}\label{exm-13-1}

Oil of relative density 0.9 flows at full bore through a pipe with an
internal diameter of 75 mm at a velocity of 1.2 m/s. Calculate the
volume flow rate in cubic metres per second and the mass flow rate in
tonnes per hour.

\begin{Shaded}
\begin{Highlighting}[]
\ImportTok{import}\NormalTok{ math}

\CommentTok{\# {-}{-}{-}{-}{-}{-}{-}{-}{-}{-}{-}{-}{-}{-}{-}{-}{-}{-}{-}{-}{-}{-}{-}{-}{-}{-}{-}{-}{-}}
\CommentTok{\# Given data}
\CommentTok{\# {-}{-}{-}{-}{-}{-}{-}{-}{-}{-}{-}{-}{-}{-}{-}{-}{-}{-}{-}{-}{-}{-}{-}{-}{-}{-}{-}{-}{-}}
\NormalTok{internal\_diameter\_mm }\OperatorTok{=} \DecValTok{75}        \CommentTok{\# mm (pipe internal diameter)}
\NormalTok{velocity\_m\_per\_s }\OperatorTok{=} \FloatTok{1.2}           \CommentTok{\# m/s (fluid velocity)}
\NormalTok{relative\_density }\OperatorTok{=} \FloatTok{0.9}           \CommentTok{\# relative to water}
\NormalTok{density\_water\_kg\_per\_m3 }\OperatorTok{=} \DecValTok{1000}   \CommentTok{\# standard density of water (kg/m³)}

\CommentTok{\# {-}{-}{-}{-}{-}{-}{-}{-}{-}{-}{-}{-}{-}{-}{-}{-}{-}{-}{-}{-}{-}{-}{-}{-}{-}{-}{-}{-}{-}}
\CommentTok{\# Convert units and calculate radius}
\CommentTok{\# {-}{-}{-}{-}{-}{-}{-}{-}{-}{-}{-}{-}{-}{-}{-}{-}{-}{-}{-}{-}{-}{-}{-}{-}{-}{-}{-}{-}{-}}
\NormalTok{internal\_diameter\_m }\OperatorTok{=}\NormalTok{ internal\_diameter\_mm }\OperatorTok{/} \DecValTok{1000}  \CommentTok{\# mm → m}
\NormalTok{radius\_m }\OperatorTok{=}\NormalTok{ internal\_diameter\_m }\OperatorTok{/} \DecValTok{2}                 \CommentTok{\# pipe radius (m)}

\CommentTok{\# {-}{-}{-}{-}{-}{-}{-}{-}{-}{-}{-}{-}{-}{-}{-}{-}{-}{-}{-}{-}{-}{-}{-}{-}{-}{-}{-}{-}{-}}
\CommentTok{\# Cross{-}sectional area}
\CommentTok{\# {-}{-}{-}{-}{-}{-}{-}{-}{-}{-}{-}{-}{-}{-}{-}{-}{-}{-}{-}{-}{-}{-}{-}{-}{-}{-}{-}{-}{-}}
\NormalTok{area\_m2 }\OperatorTok{=}\NormalTok{ math.pi }\OperatorTok{*}\NormalTok{ radius\_m }\OperatorTok{**} \DecValTok{2}                 \CommentTok{\# m²}

\CommentTok{\# {-}{-}{-}{-}{-}{-}{-}{-}{-}{-}{-}{-}{-}{-}{-}{-}{-}{-}{-}{-}{-}{-}{-}{-}{-}{-}{-}{-}{-}}
\CommentTok{\# Volume and mass flow rates}
\CommentTok{\# {-}{-}{-}{-}{-}{-}{-}{-}{-}{-}{-}{-}{-}{-}{-}{-}{-}{-}{-}{-}{-}{-}{-}{-}{-}{-}{-}{-}{-}}
\NormalTok{volume\_flow\_rate\_m3\_per\_s }\OperatorTok{=}\NormalTok{ area\_m2 }\OperatorTok{*}\NormalTok{ velocity\_m\_per\_s}

\CommentTok{\# Density of oil (kg/m³)}
\NormalTok{density\_oil\_kg\_per\_m3 }\OperatorTok{=}\NormalTok{ relative\_density }\OperatorTok{*}\NormalTok{ density\_water\_kg\_per\_m3}

\CommentTok{\# Mass flow rate (kg/s)}
\NormalTok{mass\_flow\_rate\_kg\_per\_s }\OperatorTok{=}\NormalTok{ density\_oil\_kg\_per\_m3 }\OperatorTok{*}\NormalTok{ volume\_flow\_rate\_m3\_per\_s}

\CommentTok{\# Mass flow rate in tonnes per hour}
\NormalTok{mass\_flow\_rate\_tonnes\_per\_hour }\OperatorTok{=}\NormalTok{ (mass\_flow\_rate\_kg\_per\_s }\OperatorTok{*} \DecValTok{3600}\NormalTok{) }\OperatorTok{/} \DecValTok{1000}

\CommentTok{\# {-}{-}{-}{-}{-}{-}{-}{-}{-}{-}{-}{-}{-}{-}{-}{-}{-}{-}{-}{-}{-}{-}{-}{-}{-}{-}{-}{-}{-}}
\CommentTok{\# Output results}
\CommentTok{\# {-}{-}{-}{-}{-}{-}{-}{-}{-}{-}{-}{-}{-}{-}{-}{-}{-}{-}{-}{-}{-}{-}{-}{-}{-}{-}{-}{-}{-}}
\BuiltInTok{print}\NormalTok{(}\SpecialStringTok{f"Volume flow rate, m³/s         : }\SpecialCharTok{\{}\NormalTok{volume\_flow\_rate\_m3\_per\_s}\SpecialCharTok{:.4f\}}\SpecialStringTok{"}\NormalTok{)}
\BuiltInTok{print}\NormalTok{(}\SpecialStringTok{f"Mass flow rate, tonnes/hour    : }\SpecialCharTok{\{}\NormalTok{mass\_flow\_rate\_tonnes\_per\_hour}\SpecialCharTok{:.4f\}}\SpecialStringTok{"}\NormalTok{)}
\end{Highlighting}
\end{Shaded}

\end{example}

\section{Flow Through Valves}\label{flow-through-valves}

Referring to the figure, the area of escape is the annular area of the
circumferential opening between the valve and seat, which is the
circumference multiplied by the lift.

\begin{figure}[H]

{\centering \pandocbounded{\includegraphics[keepaspectratio]{images/flow_through_valves.png}}

}

\caption{Flow through valves}

\end{figure}%

The lift beyond which it would cause no restriction is when the
circumferential area of the lift opening is equal to the cross-sectional
area of the bore, thus,

\[
\text{Circumference} \times \text{Lift} = \text{Cross-sectional area}.
\] \[
\pi d \times L = \frac{\pi d^{2}}{4},
\]

where \(L\) is the lift and \(d\) is the diameter of the valve.

Hence, a lift equal to one-quarter of the diameter of the valve allows
full bore flow:

\[
L = \frac{d}{4}.
\]

\begin{example}[]\protect\hypertarget{exm-13-2}{}\label{exm-13-2}

Calculate the volume flow rate of water in cubic metres per minute
through a 150-millimetre diameter valve when the velocity of the water
is 2.5 m/s and the valve lift is (i) 30 millimetres and (ii) 45
millimetres.

\begin{Shaded}
\begin{Highlighting}[]
\ImportTok{import}\NormalTok{ math}

\CommentTok{\# Given}
\NormalTok{d }\OperatorTok{=} \FloatTok{0.15}                  \CommentTok{\# diameter in metres}
\NormalTok{V }\OperatorTok{=} \FloatTok{2.5}                   \CommentTok{\# velocity in m/s}

\CommentTok{\# Maximum effective lift}
\NormalTok{critical\_lift\_mm }\OperatorTok{=}\NormalTok{ (d }\OperatorTok{/} \DecValTok{4}\NormalTok{) }\OperatorTok{*} \DecValTok{1000}

\CommentTok{\# Full bore area and flow rate}
\NormalTok{A\_bore }\OperatorTok{=}\NormalTok{ math.pi }\OperatorTok{*}\NormalTok{ d}\OperatorTok{**}\DecValTok{2} \OperatorTok{/} \DecValTok{4}
\NormalTok{Q\_full\_m3\_per\_s }\OperatorTok{=}\NormalTok{ V }\OperatorTok{*}\NormalTok{ A\_bore}
\NormalTok{Q\_full }\OperatorTok{=}\NormalTok{ Q\_full\_m3\_per\_s }\OperatorTok{*} \DecValTok{60}      \CommentTok{\# convert to m³/min}


\CommentTok{\# (i) Lift = 30 mm}
\NormalTok{L1 }\OperatorTok{=} \FloatTok{0.030}                \CommentTok{\# lift in metres}
\NormalTok{A\_annular1 }\OperatorTok{=}\NormalTok{ math.pi }\OperatorTok{*}\NormalTok{ d }\OperatorTok{*}\NormalTok{ L1}
\NormalTok{A\_effective1 }\OperatorTok{=} \BuiltInTok{min}\NormalTok{(A\_annular1, A\_bore)}
\NormalTok{Q1 }\OperatorTok{=}\NormalTok{ V }\OperatorTok{*}\NormalTok{ A\_effective1 }\OperatorTok{*} \DecValTok{60}


\CommentTok{\# (ii) Lift = 45 mm}
\NormalTok{L2 }\OperatorTok{=} \FloatTok{0.045}                \CommentTok{\# lift in metres}
\NormalTok{A\_annular2 }\OperatorTok{=}\NormalTok{ math.pi }\OperatorTok{*}\NormalTok{ d }\OperatorTok{*}\NormalTok{ L2}
\NormalTok{A\_effective2 }\OperatorTok{=} \BuiltInTok{min}\NormalTok{(A\_annular2, A\_bore)}
\NormalTok{Q2 }\OperatorTok{=}\NormalTok{ V }\OperatorTok{*}\NormalTok{ A\_effective2 }\OperatorTok{*} \DecValTok{60}


\BuiltInTok{print}\NormalTok{(}\SpecialStringTok{f"Maximum effective lift: }\SpecialCharTok{\{}\NormalTok{critical\_lift\_mm}\SpecialCharTok{:.4f\}}\SpecialStringTok{ mm"}\NormalTok{)}
\BuiltInTok{print}\NormalTok{(}\SpecialStringTok{f"Full bore flow rate: }\SpecialCharTok{\{}\NormalTok{Q\_full}\SpecialCharTok{:.4f\}}\SpecialStringTok{ m³/min"}\NormalTok{)}
\BuiltInTok{print}\NormalTok{(}\SpecialStringTok{f"At 30 mm lift: }\SpecialCharTok{\{}\NormalTok{Q1}\SpecialCharTok{:.4f\}}\SpecialStringTok{ m³/min"}\NormalTok{)}
\BuiltInTok{print}\NormalTok{(}\SpecialStringTok{f"At 45 mm lift: }\SpecialCharTok{\{}\NormalTok{Q2}\SpecialCharTok{:.4f\}}\SpecialStringTok{ m³/min"}\NormalTok{)}
\end{Highlighting}
\end{Shaded}

\end{example}

\section{Discharge Through an
Orifice}\label{discharge-through-an-orifice}

When water discharges through an orifice in the side of a tank, the
potential energy associated with the height of the water surface above
the orifice is converted into kinetic energy of the efflux.

\begin{figure}[H]

{\centering \includegraphics[width=8.2cm,height=\textheight,keepaspectratio]{images/OrificePlate.png}

}

\caption{Orifice}

\end{figure}%

This yields the theoretical velocity of the jet: \[
C = \sqrt{2gh}
\] where \(g\) is the acceleration due to gravity and \(h\) is the
height of the water surface above the orifice.

The theoretical volume flow rate is then: \[
\dot{v}_{\text{theoretical}} = A C = A \sqrt{2gh}
\] where \(A\) is the area of the orifice.

Due to viscous effects, the actual velocity is less than the theoretical
value. The coefficient of velocity, \(C_v\), is defined as the ratio of
actual velocity to theoretical velocity. Thus, the velocity-corrected
flow rate is: \[
\dot{v}_{\text{velocity-corrected}} = C_v A \sqrt{2gh}
\]

Downstream of a sharp-edged orifice, the jet contracts due to streamline
curvature, forming a vena contracta. The vena contracta refers to the
narrowest cross-section of a fluid jet shortly downstream of an orifice
or nozzle, where the streamlines converge, resulting in the smallest
area, maximum velocity, and minimum pressure.

\begin{figure}[H]

{\centering \pandocbounded{\includegraphics[keepaspectratio]{images/vena_contracta.png}}

}

\caption{Vena contracta}

\end{figure}%

The coefficient of contraction, \(C_A\), is the ratio of the jet area at
the vena contracta to the orifice area: \[
A_{\text{jet}} = C_A A
\]

The actual volume flow rate therefore becomes: \[
\dot{v}_{\text{actual}} = C_A A \cdot C_v \sqrt{2gh}
\]

The coefficient of discharge, \(C_d\), is the ratio of the actual volume
flow rate to the theoretical volume flow rate: \[
C_d = C_A \cdot C_v
\]

Consequently, \[
\dot{v}_{\text{actual}} = C_d A \sqrt{2gh}
\]

\begin{example}[]\protect\hypertarget{exm-13-3}{}\label{exm-13-3}

Water escapes through a hole 20 mm in diameter in the side of a tank,
with a water head of 3 m above the hole. Given a coefficient of velocity
of 0.97 and a coefficient of reduction of area of 0.64, calculate (i)
the velocity of the water jet as it exits the hole and (ii) the quantity
of water escaping tonne per hour.

\begin{Shaded}
\begin{Highlighting}[]
\ImportTok{import}\NormalTok{ math}

\CommentTok{\# Given}
\NormalTok{diameter\_mm }\OperatorTok{=} \DecValTok{20}
\NormalTok{diameter\_m }\OperatorTok{=}\NormalTok{ diameter\_mm }\OperatorTok{/} \DecValTok{1000}
\NormalTok{radius\_m }\OperatorTok{=}\NormalTok{ diameter\_m }\OperatorTok{/} \DecValTok{2}
\NormalTok{h }\OperatorTok{=} \DecValTok{3}  \CommentTok{\# m}
\NormalTok{Cv }\OperatorTok{=} \FloatTok{0.97}
\NormalTok{Ca }\OperatorTok{=} \FloatTok{0.64}
\NormalTok{g }\OperatorTok{=} \FloatTok{9.81}  \CommentTok{\# m/s²}
\NormalTok{density\_water }\OperatorTok{=} \DecValTok{1000}  \CommentTok{\# kg/m³}

\CommentTok{\# Calculations}
\NormalTok{area }\OperatorTok{=}\NormalTok{ math.pi }\OperatorTok{*}\NormalTok{ radius\_m }\OperatorTok{**} \DecValTok{2}

\NormalTok{v\_theoretical }\OperatorTok{=}\NormalTok{ math.sqrt(}\DecValTok{2} \OperatorTok{*}\NormalTok{ g }\OperatorTok{*}\NormalTok{ h)}
\NormalTok{v\_actual }\OperatorTok{=}\NormalTok{ Cv }\OperatorTok{*}\NormalTok{ v\_theoretical}

\NormalTok{volume\_flow\_actual }\OperatorTok{=}\NormalTok{ Ca }\OperatorTok{*}\NormalTok{ area }\OperatorTok{*}\NormalTok{ v\_actual  }\CommentTok{\# m³/s}

\NormalTok{mass\_flow }\OperatorTok{=}\NormalTok{ density\_water }\OperatorTok{*}\NormalTok{ volume\_flow\_actual  }\CommentTok{\# kg/s}

\NormalTok{mass\_per\_hour\_kg }\OperatorTok{=}\NormalTok{ mass\_flow }\OperatorTok{*} \DecValTok{3600}
\NormalTok{mass\_per\_hour\_tonne }\OperatorTok{=}\NormalTok{ mass\_per\_hour\_kg }\OperatorTok{/} \DecValTok{1000}

\CommentTok{\# Output results}
\BuiltInTok{print}\NormalTok{(}\SpecialStringTok{f"(i) Actual velocity of the water jet: }\SpecialCharTok{\{}\NormalTok{v\_actual}\SpecialCharTok{:.3f\}}\SpecialStringTok{ m/s"}\NormalTok{)}

\BuiltInTok{print}\NormalTok{(}\SpecialStringTok{f"}\CharTok{\textbackslash{}n}\SpecialStringTok{(ii) Quantity of water escaping per hour:"}\NormalTok{)}
\BuiltInTok{print}\NormalTok{(}\SpecialStringTok{f"    Volume flow rate: }\SpecialCharTok{\{}\NormalTok{volume\_flow\_actual}\SpecialCharTok{:.6f\}}\SpecialStringTok{ m³/s"}\NormalTok{)}
\BuiltInTok{print}\NormalTok{(}\SpecialStringTok{f"    Mass flow rate: }\SpecialCharTok{\{}\NormalTok{mass\_flow}\SpecialCharTok{:.4f\}}\SpecialStringTok{ kg/s"}\NormalTok{)}
\BuiltInTok{print}\NormalTok{(}\SpecialStringTok{f"    Mass per hour: }\SpecialCharTok{\{}\NormalTok{mass\_per\_hour\_kg}\SpecialCharTok{:.4f\}}\SpecialStringTok{ kg/hour"}\NormalTok{)}
\BuiltInTok{print}\NormalTok{(}\SpecialStringTok{f"    Mass per hour: }\SpecialCharTok{\{}\NormalTok{mass\_per\_hour\_tonne}\SpecialCharTok{:.4f\}}\SpecialStringTok{ tonnes/hour"}\NormalTok{)}
\end{Highlighting}
\end{Shaded}

\end{example}

\section{Continuity Equation}\label{continuity-equation}

Continuous flow exists in a flow system when the mass flow rate is
constant throughout the system. If in the diagram below, the mass flow
rate at 1 is equal to that at 2, then continuity exists.

\begin{figure}[H]

{\centering \includegraphics[width=10.2cm,height=\textheight,keepaspectratio]{images/continuity_equation.png}

}

\caption{Continuity equation}

\end{figure}%

Since \(\dot{m_1} = \dot{m_2}\), therefore

\[
\rho_1 \cdot A_1 \cdot C_1= \rho_2 \cdot A_2 \cdot C_2
\]

If the fluid is incompressible (most liquids), then density will remain
constant (\(\rho_1=\rho_2\)) and the above equations may be written as:

\[
A_1 \cdot C_1= A_2 \cdot C_2
\] Or

\[
\dot{v_1}= \dot{v_2}
\]

That is, the volume flow rate is constant for an incompressible fluid.

\begin{example}[]\protect\hypertarget{exm-conteq}{}\label{exm-conteq}

A pipe decreases in diameter from 300 mm to 200 mm. Water flows from the
larger to the smaller pipe at a constant rate of 18.4 kL/min. Calculate
the mass flow rate and the velocities in the larger and smaller pipe.

\begin{Shaded}
\begin{Highlighting}[]
\ImportTok{import}\NormalTok{ math}

\CommentTok{\# Given}
\NormalTok{flow\_rate\_kL\_per\_min }\OperatorTok{=} \FloatTok{18.4}   \CommentTok{\# Volumetric flow rate in kL/min}
\NormalTok{diameter\_large\_mm }\OperatorTok{=} \DecValTok{300}       \CommentTok{\# Diameter of larger pipe in mm}
\NormalTok{diameter\_small\_mm }\OperatorTok{=} \DecValTok{200}       \CommentTok{\# Diameter of smaller pipe in mm}
\NormalTok{density\_water }\OperatorTok{=} \DecValTok{1000}          \CommentTok{\# Density of water in kg/m³}

\CommentTok{\# Convert volumetric flow rate to m³/s}
\NormalTok{flow\_rate\_m3\_per\_s }\OperatorTok{=}\NormalTok{ (flow\_rate\_kL\_per\_min }\OperatorTok{/} \DecValTok{60}\NormalTok{)}

\CommentTok{\# Convert diameters to meters and calculate radii}
\NormalTok{d1 }\OperatorTok{=}\NormalTok{ diameter\_large\_mm }\OperatorTok{/} \DecValTok{1000}  \CommentTok{\# m}
\NormalTok{d2 }\OperatorTok{=}\NormalTok{ diameter\_small\_mm }\OperatorTok{/} \DecValTok{1000}  \CommentTok{\# m}
\NormalTok{r1 }\OperatorTok{=}\NormalTok{ d1 }\OperatorTok{/} \DecValTok{2}
\NormalTok{r2 }\OperatorTok{=}\NormalTok{ d2 }\OperatorTok{/} \DecValTok{2}

\CommentTok{\# Cross{-}sectional areas}
\NormalTok{A1 }\OperatorTok{=}\NormalTok{ math.pi }\OperatorTok{*}\NormalTok{ r1}\OperatorTok{**}\DecValTok{2}  \CommentTok{\# m²}
\NormalTok{A2 }\OperatorTok{=}\NormalTok{ math.pi }\OperatorTok{*}\NormalTok{ r2}\OperatorTok{**}\DecValTok{2}  \CommentTok{\# m²}

\CommentTok{\# Velocities}
\NormalTok{v1 }\OperatorTok{=}\NormalTok{ flow\_rate\_m3\_per\_s }\OperatorTok{/}\NormalTok{ A1  }\CommentTok{\# m/s (larger pipe)}
\NormalTok{v2 }\OperatorTok{=}\NormalTok{ flow\_rate\_m3\_per\_s }\OperatorTok{/}\NormalTok{ A2  }\CommentTok{\# m/s (smaller pipe)}

\CommentTok{\# Mass flow rate}
\NormalTok{mass\_flow\_rate }\OperatorTok{=}\NormalTok{ density\_water }\OperatorTok{*}\NormalTok{ flow\_rate\_m3\_per\_s  }\CommentTok{\# kg/s}

\CommentTok{\# Output results}
\BuiltInTok{print}\NormalTok{(}\SpecialStringTok{f"Mass flow rate: }\SpecialCharTok{\{}\NormalTok{mass\_flow\_rate}\SpecialCharTok{:.4f\}}\SpecialStringTok{ kg/s"}\NormalTok{)}
\BuiltInTok{print}\NormalTok{(}\SpecialStringTok{f"Velocity in larger pipe (300 mm): }\SpecialCharTok{\{}\NormalTok{v1}\SpecialCharTok{:.4f\}}\SpecialStringTok{ m/s"}\NormalTok{)}
\BuiltInTok{print}\NormalTok{(}\SpecialStringTok{f"Velocity in smaller pipe (200 mm): }\SpecialCharTok{\{}\NormalTok{v2}\SpecialCharTok{:.4f\}}\SpecialStringTok{ m/s"}\NormalTok{)}
\end{Highlighting}
\end{Shaded}

\end{example}

\section{The Energy Equation for an Ideal
Fluid}\label{the-energy-equation-for-an-ideal-fluid}

The steady flow equation is developed from the Law of Conservation of
Energy. If there are no energy losses or energy additions in a flow
system, then the total energy of a flowing fluid will remain constant.

A flowing fluid may lose energy as a result of fluid friction, heat
energy transfer and fluid motors. For an ideal fluid, frictional losses
are equal to zero. Energy may be added to a fluid via a pump or heat
energy addition.

The total energy possessed by a flowing fluid consists of:

\begin{itemize}
\tightlist
\item
  internal energy
\item
  potential energy
\item
  kinetic energy and
\item
  pressure energy (flow energy).
\end{itemize}

Generally in fluid mechanics the change in internal energy is considered
to be negligible. Heat energy transfers are usually not considered in
fluid mechanics and the frictional heat developed by a flowing fluid is
relatively small. Therefore, the internal energy term is omitted from
the energy balance.

Consider the flow system shown below. For this system, assume:

\begin{itemize}
\tightlist
\item
  an ideal fluid
\item
  that there are no energy losses or additions and
\item
  steady flow conditions exist.
\end{itemize}

\begin{figure}[H]

{\centering \includegraphics[width=9.3cm,height=\textheight,keepaspectratio]{images/energy_equation.png}

}

\caption{Energy equation}

\end{figure}%

From the Law of Conservation of Energy:

The Total Energy at section 1 = The Total Energy at section 2

\[
mZ_1g+\frac{m C_1^2}{2}+mP_1v_1=mZ_2g+\frac{m C_2^2}{2}+mP_2v_2
\]

Dividing through by \(mg\):

\[
\frac{mZ_1g}{mg}+\frac{m C_1^2}{2mg}+\frac{mP_1v_1}{mg}=\frac{mZ_2g}{mg}+\frac{m C_2^2}{2mg}+\frac{mP_2v_2}{mg}
\]

Therefore

\[
Z_1+\frac{C_1^2}{2g}+\frac{P_1v_1}{g}=Z_2+\frac{C_2^2}{2g}+\frac{P_2v_2}{g}
\]

In fluid mechanics density (\(\rho\)) is generally used in preference to
specific volume, i.e.~\(v=\frac{1}{\rho}\) Therefore:

\[
Z_1+\frac{C_1^2}{2g}+\frac{P_1}{g\rho_1}=Z_2+\frac{C_2^2}{2g}+\frac{P_2}{g\rho_2}
\]

In fluid mechanics, specific weight represents the force exerted by
gravity on a unit volume of a fluid. For this reason, units are
expressed as force per unit volume (e.g., \(N/m^3\))

Specific weight is given \(\gamma = g\rho\)

Where:

\(\gamma\): specific weight, \(\mathrm{N/m^3}\)

\(g\): gravitational acceleration, \(\mathrm{m/s^2}\)

\(\rho\): density, \(\mathrm{kg/m^3}\)

\[
Z_1+\frac{C_1^2}{2g}+\frac{P_1}{\gamma_1}=Z_2+\frac{C_2^2}{2g}+\frac{P_2}{\gamma_2}
\]

Each term has units of m, therefore:

\begin{itemize}
\item
  Potential energy \(Z\) is known as the elevation head.
\item
  Kinetic energy \(\frac{c^2}{2g}\) is known as the velocity head.
\item
  Pressure energy \(\frac{P}{\gamma}\) is known as the pressure head.
\end{itemize}

Similarly, the total energy of a flowing fluid is known as the total
head (H).

\[
Total\ Head= Elevation\ Head + Velocity\ Head + Pressure\ Head
\] \[
H=Z+\frac{C^2}{2g}+\frac{P}{\gamma}
\]

The total head (H) will be a constant throughout a flow system if:

\begin{enumerate}
\def\labelenumi{\arabic{enumi}.}
\item
  frictional losses (head loss) are equal to zero
\item
  work energy is not added by a pump (pump head) or removed by a motor.
\end{enumerate}

\begin{example}[]\protect\hypertarget{exm-energyeq}{}\label{exm-energyeq}

Consider a simple flow system consisting of a varying cross-section
pipe. Water flows through this system at a rate of 2000 L/min. As the
pipe increases in elevation from 30 m to 36 m it decreases in diameter
from 10 cm to 3.0 cm. If the pressure is 6.5 MPa at the 30 m elevation,
what is the pressure at 36 m?

\begin{figure}[H]

{\centering \includegraphics[width=9.4cm,height=\textheight,keepaspectratio]{images/energy_equation_exm.png}

}

\caption{Example}

\end{figure}%

\begin{Shaded}
\begin{Highlighting}[]
\ImportTok{import}\NormalTok{ math}

\CommentTok{\# Given}
\NormalTok{v\_dot\_lpm }\OperatorTok{=} \FloatTok{2000.0}                \CommentTok{\# Flow rate, L/min}
\NormalTok{v\_dot }\OperatorTok{=}\NormalTok{ v\_dot\_lpm }\OperatorTok{/} \FloatTok{60000.0}       \CommentTok{\# Convert to m³/s}
\NormalTok{d1 }\OperatorTok{=} \FloatTok{0.10}                         \CommentTok{\# Diameter at lower section, m}
\NormalTok{d2 }\OperatorTok{=} \FloatTok{0.03}                         \CommentTok{\# Diameter at higher section, m}
\NormalTok{z1 }\OperatorTok{=} \FloatTok{30.0}                         \CommentTok{\# Elevation at lower section, m}
\NormalTok{z2 }\OperatorTok{=} \FloatTok{36.0}                         \CommentTok{\# Elevation at higher section, m}
\NormalTok{P1 }\OperatorTok{=} \FloatTok{6.5e6}                        \CommentTok{\# Pressure at lower section, Pa}
\NormalTok{rho }\OperatorTok{=} \FloatTok{1000.0}                      \CommentTok{\# Density of water, kg/m³}

\CommentTok{\# Cross{-}sectional areas}
\NormalTok{A1 }\OperatorTok{=}\NormalTok{ math.pi }\OperatorTok{*}\NormalTok{ (d1 }\OperatorTok{/} \FloatTok{2.0}\NormalTok{)}\OperatorTok{**}\DecValTok{2}
\NormalTok{A2 }\OperatorTok{=}\NormalTok{ math.pi }\OperatorTok{*}\NormalTok{ (d2 }\OperatorTok{/} \FloatTok{2.0}\NormalTok{)}\OperatorTok{**}\DecValTok{2}

\CommentTok{\# Velocities}
\NormalTok{c1 }\OperatorTok{=}\NormalTok{ v\_dot }\OperatorTok{/}\NormalTok{ A1}
\NormalTok{c2 }\OperatorTok{=}\NormalTok{ v\_dot }\OperatorTok{/}\NormalTok{ A2}

\CommentTok{\# Bernoulli\textquotesingle{}s equation to find P2}
\NormalTok{P2 }\OperatorTok{=}\NormalTok{ P1 }\OperatorTok{+} \FloatTok{0.5} \OperatorTok{*}\NormalTok{ rho }\OperatorTok{*}\NormalTok{ (c1}\OperatorTok{**}\DecValTok{2} \OperatorTok{{-}}\NormalTok{ c2}\OperatorTok{**}\DecValTok{2}\NormalTok{) }\OperatorTok{+}\NormalTok{ rho }\OperatorTok{*}\NormalTok{ g }\OperatorTok{*}\NormalTok{ (z1 }\OperatorTok{{-}}\NormalTok{ z2)}

\CommentTok{\# Convert P2 to MPa}
\NormalTok{P2\_MPa }\OperatorTok{=}\NormalTok{ P2 }\OperatorTok{/} \FloatTok{1e6}

\CommentTok{\# Output results}
\BuiltInTok{print}\NormalTok{(}\SpecialStringTok{f"Flow rate: }\SpecialCharTok{\{}\NormalTok{v\_dot}\SpecialCharTok{:.5f\}}\SpecialStringTok{ m³/s"}\NormalTok{)}
\BuiltInTok{print}\NormalTok{(}\SpecialStringTok{f"Area 1: }\SpecialCharTok{\{}\NormalTok{A1}\SpecialCharTok{:.6f\}}\SpecialStringTok{ m²"}\NormalTok{)}
\BuiltInTok{print}\NormalTok{(}\SpecialStringTok{f"Area 2: }\SpecialCharTok{\{}\NormalTok{A2}\SpecialCharTok{:.6f\}}\SpecialStringTok{ m²"}\NormalTok{)}
\BuiltInTok{print}\NormalTok{(}\SpecialStringTok{f"Velocity 1: }\SpecialCharTok{\{}\NormalTok{c1}\SpecialCharTok{:.4f\}}\SpecialStringTok{ m/s"}\NormalTok{)}
\BuiltInTok{print}\NormalTok{(}\SpecialStringTok{f"Velocity 2: }\SpecialCharTok{\{}\NormalTok{c2}\SpecialCharTok{:.4f\}}\SpecialStringTok{ m/s"}\NormalTok{)}
\BuiltInTok{print}\NormalTok{(}\SpecialStringTok{f"Pressure at 36 m elevation: }\SpecialCharTok{\{}\NormalTok{P2\_MPa}\SpecialCharTok{:.4f\}}\SpecialStringTok{ MPa"}\NormalTok{)}
\end{Highlighting}
\end{Shaded}

\end{example}

\section{Bernoulli's Equation}\label{bernoullis-equation}

The Bernoulli's equation between two points in a fluid flow is given by:

\[
P_1 + \frac{1}{2} \rho C_1^2 + \rho g h_1 = P_2 + \frac{1}{2} \rho C_2^2 + \rho g h_2
\]

Where:

\begin{itemize}
\item
  \(P_1\) and \(P_2\) are the pressures at points 1 and 2, respectively.
\item
  \(\rho\) is the density of the fluid.
\item
  \(C_1\) and \(C_2\) are the velocities of the fluid at points 1 and 2,
  respectively.
\item
  \(g\) is the acceleration due to gravity.
\item
  \(h_1\) and \(h_2\) are the heights of the fluid at points 1 and 2,
  respectively.
\end{itemize}

\section{Venturi Meter}\label{venturi-meter}

A venturi meter measures liquid flow rates in pipelines. The device
features a pipe section that narrows in the middle (called the throat)
and widens at both ends, as shown below:

\begin{figure}[H]

{\centering \includegraphics[width=9.6cm,height=\textheight,keepaspectratio]{images/venturi.png}

}

\caption{Venturi meter}

\end{figure}%

When the entrance and throat areas are known, and the pressure readings
(or pressure difference) at these points are measured, Bernoulli's
equation can be used to calculate the liquid velocity and flow rate.
Typically, a venturi meter is installed horizontally in the pipeline,
which simplifies calculations since there is no elevation difference
between points (Z\textsubscript{1} = Z\textsubscript{2}), thus
eliminating the height terms.

\begin{example}[]\protect\hypertarget{exm-fe1}{}\label{exm-fe1}

Calculate the mass flow rate in kilograms per second through a smooth,
horizontal venturi meter with an entrance diameter of 375 mm and a
throat diameter of 125 mm, given a pressure difference of 457 mm of
water

\begin{Shaded}
\begin{Highlighting}[]
\ImportTok{import}\NormalTok{ math}

\CommentTok{\# Given}
\NormalTok{D1 }\OperatorTok{=} \FloatTok{0.375}  \CommentTok{\# Entrance diameter in meters}
\NormalTok{D2 }\OperatorTok{=} \FloatTok{0.125}  \CommentTok{\# Throat diameter in meters}
\NormalTok{h }\OperatorTok{=} \FloatTok{0.457}   \CommentTok{\# Pressure head difference in meters of water}
\NormalTok{g }\OperatorTok{=} \FloatTok{9.81}    \CommentTok{\# Gravitational acceleration in m/s²}
\NormalTok{rho }\OperatorTok{=} \DecValTok{1000}  \CommentTok{\# Density of water in kg/m³}

\CommentTok{\# Calculate cross{-}sectional areas}
\NormalTok{A1 }\OperatorTok{=}\NormalTok{ math.pi }\OperatorTok{*}\NormalTok{ (D1 }\OperatorTok{/} \DecValTok{2}\NormalTok{)}\OperatorTok{**}\DecValTok{2}
\NormalTok{A2 }\OperatorTok{=}\NormalTok{ math.pi }\OperatorTok{*}\NormalTok{ (D2 }\OperatorTok{/} \DecValTok{2}\NormalTok{)}\OperatorTok{**}\DecValTok{2}

\CommentTok{\# Compute the squared area ratio (A2/A1)\^{}2}
\NormalTok{ratio\_squared }\OperatorTok{=}\NormalTok{ (A2 }\OperatorTok{/}\NormalTok{ A1)}\OperatorTok{**}\DecValTok{2}

\CommentTok{\# Calculate throat velocity c2}
\NormalTok{c2 }\OperatorTok{=}\NormalTok{ math.sqrt((}\DecValTok{2} \OperatorTok{*}\NormalTok{ g }\OperatorTok{*}\NormalTok{ h) }\OperatorTok{/}\NormalTok{ (}\DecValTok{1} \OperatorTok{{-}}\NormalTok{ ratio\_squared))}

\CommentTok{\# Calculate volumetric flow rate v\_dot using throat section}
\NormalTok{v\_dot }\OperatorTok{=}\NormalTok{ A2 }\OperatorTok{*}\NormalTok{ c2}

\CommentTok{\# Calculate mass flow rate}
\NormalTok{mass\_flow\_rate }\OperatorTok{=}\NormalTok{ rho }\OperatorTok{*}\NormalTok{ v\_dot}

\CommentTok{\# Output results}
\BuiltInTok{print}\NormalTok{(}\SpecialStringTok{f"Entrance area: }\SpecialCharTok{\{}\NormalTok{A1}\SpecialCharTok{:.6f\}}\SpecialStringTok{ m²"}\NormalTok{)}
\BuiltInTok{print}\NormalTok{(}\SpecialStringTok{f"Throat area: }\SpecialCharTok{\{}\NormalTok{A2}\SpecialCharTok{:.6f\}}\SpecialStringTok{ m²"}\NormalTok{)}
\BuiltInTok{print}\NormalTok{(}\SpecialStringTok{f"Throat velocity: }\SpecialCharTok{\{}\NormalTok{c2}\SpecialCharTok{:.4f\}}\SpecialStringTok{ m/s"}\NormalTok{)}
\BuiltInTok{print}\NormalTok{(}\SpecialStringTok{f"Volumetric flow rate: }\SpecialCharTok{\{}\NormalTok{v\_dot}\SpecialCharTok{:.4f\}}\SpecialStringTok{ m³/s"}\NormalTok{)}
\BuiltInTok{print}\NormalTok{(}\SpecialStringTok{f"Mass flow rate: }\SpecialCharTok{\{}\NormalTok{mass\_flow\_rate}\SpecialCharTok{:.4f\}}\SpecialStringTok{ kg/s"}\NormalTok{)}
\end{Highlighting}
\end{Shaded}

\end{example}

\begin{example}[]\protect\hypertarget{exm-13_7}{}\label{exm-13_7}

A horizontal venturi meter has an inlet diameter of 450 mm and a throat
diameter of 225 mm. The pressure difference between these two points is
equivalent to 381 mm of water. Given the density of fresh water is 1000
kg/m³, calculate the mass flow rate through the meter.

\begin{Shaded}
\begin{Highlighting}[]
\ImportTok{import}\NormalTok{ math}

\CommentTok{\# Given}
\NormalTok{D1 }\OperatorTok{=} \FloatTok{0.450}  \CommentTok{\# Inlet diameter in meters}
\NormalTok{D2 }\OperatorTok{=} \FloatTok{0.225}  \CommentTok{\# Throat diameter in meters}
\NormalTok{h }\OperatorTok{=} \FloatTok{0.381}   \CommentTok{\# Pressure head difference in meters of water}
\NormalTok{g }\OperatorTok{=} \FloatTok{9.81}    \CommentTok{\# Gravitational acceleration in m/s²}
\NormalTok{rho }\OperatorTok{=} \DecValTok{1000}  \CommentTok{\# Density of water in kg/m³}

\CommentTok{\# Calculate cross{-}sectional areas}
\NormalTok{A1 }\OperatorTok{=}\NormalTok{ math.pi }\OperatorTok{*}\NormalTok{ (D1 }\OperatorTok{/} \DecValTok{2}\NormalTok{)}\OperatorTok{**}\DecValTok{2}
\NormalTok{A2 }\OperatorTok{=}\NormalTok{ math.pi }\OperatorTok{*}\NormalTok{ (D2 }\OperatorTok{/} \DecValTok{2}\NormalTok{)}\OperatorTok{**}\DecValTok{2}

\CommentTok{\# Compute the squared area ratio (A2/A1)\^{}2}
\NormalTok{ratio\_squared }\OperatorTok{=}\NormalTok{ (A2 }\OperatorTok{/}\NormalTok{ A1)}\OperatorTok{**}\DecValTok{2}

\CommentTok{\# Calculate throat velocity c2}
\NormalTok{c2 }\OperatorTok{=}\NormalTok{ math.sqrt((}\DecValTok{2} \OperatorTok{*}\NormalTok{ g }\OperatorTok{*}\NormalTok{ h) }\OperatorTok{/}\NormalTok{ (}\DecValTok{1} \OperatorTok{{-}}\NormalTok{ ratio\_squared))}

\CommentTok{\# Calculate volumetric flow rate v\_dot using throat section}
\NormalTok{v\_dot }\OperatorTok{=}\NormalTok{ A2 }\OperatorTok{*}\NormalTok{ c2}

\CommentTok{\# Calculate mass flow rate}
\NormalTok{mass\_flow\_rate }\OperatorTok{=}\NormalTok{ rho }\OperatorTok{*}\NormalTok{ v\_dot}

\CommentTok{\# Output results}
\BuiltInTok{print}\NormalTok{(}\SpecialStringTok{f"Entrance area: }\SpecialCharTok{\{}\NormalTok{A1}\SpecialCharTok{:.6f\}}\SpecialStringTok{ m²"}\NormalTok{)}
\BuiltInTok{print}\NormalTok{(}\SpecialStringTok{f"Throat area: }\SpecialCharTok{\{}\NormalTok{A2}\SpecialCharTok{:.6f\}}\SpecialStringTok{ m²"}\NormalTok{)}
\BuiltInTok{print}\NormalTok{(}\SpecialStringTok{f"Throat velocity: }\SpecialCharTok{\{}\NormalTok{c2}\SpecialCharTok{:.4f\}}\SpecialStringTok{ m/s"}\NormalTok{)}
\BuiltInTok{print}\NormalTok{(}\SpecialStringTok{f"Volumetric flow rate: }\SpecialCharTok{\{}\NormalTok{v\_dot}\SpecialCharTok{:.4f\}}\SpecialStringTok{ m³/s"}\NormalTok{)}
\BuiltInTok{print}\NormalTok{(}\SpecialStringTok{f"Mass flow rate: }\SpecialCharTok{\{}\NormalTok{mass\_flow\_rate}\SpecialCharTok{:.4f\}}\SpecialStringTok{ kg/s"}\NormalTok{)}
\end{Highlighting}
\end{Shaded}

\end{example}

\begin{example}[]\protect\hypertarget{exm-13_8}{}\label{exm-13_8}

A 300 mm diameter pipe has a venturi meter with a throat diameter of 100
mm. A U-tube gauge filled with water and mercury measures a pressure
head difference of 250 mm between the inlet and throat. The meter
coefficient is 0.95. Using the density of water (1000 kg/m³), calculate
the discharge rate through the pipe in m³/s.

\begin{Shaded}
\begin{Highlighting}[]
\ImportTok{import}\NormalTok{ math}

\CommentTok{\# {-}{-}{-}{-}{-}{-}{-}{-}{-}{-}{-}{-}{-}{-}{-}{-}{-}{-}{-}{-}{-}{-}{-}{-}{-}{-}{-}{-}{-}}
\CommentTok{\# Given data}
\CommentTok{\# {-}{-}{-}{-}{-}{-}{-}{-}{-}{-}{-}{-}{-}{-}{-}{-}{-}{-}{-}{-}{-}{-}{-}{-}{-}{-}{-}{-}{-}}
\NormalTok{D1 }\OperatorTok{=} \FloatTok{0.3}  \CommentTok{\# Inlet diameter in meters}
\NormalTok{D2 }\OperatorTok{=} \FloatTok{0.1}  \CommentTok{\# Throat diameter in meters}
\NormalTok{Cd }\OperatorTok{=} \FloatTok{0.95}  \CommentTok{\# Coefficient of discharge}
\NormalTok{h\_mercury }\OperatorTok{=} \FloatTok{0.250}  \CommentTok{\# Manometer reading in meters of mercury}
\NormalTok{SG }\OperatorTok{=} \FloatTok{13.6}  \CommentTok{\# Specific gravity of mercury relative to water}
\NormalTok{g }\OperatorTok{=} \FloatTok{9.81}  \CommentTok{\# Acceleration due to gravity in m/s²}

\CommentTok{\# {-}{-}{-}{-}{-}{-}{-}{-}{-}{-}{-}{-}{-}{-}{-}{-}{-}{-}{-}{-}{-}{-}{-}{-}{-}{-}{-}{-}{-}}
\CommentTok{\# Cross{-}sectional areas}
\CommentTok{\# {-}{-}{-}{-}{-}{-}{-}{-}{-}{-}{-}{-}{-}{-}{-}{-}{-}{-}{-}{-}{-}{-}{-}{-}{-}{-}{-}{-}{-}}
\NormalTok{A1 }\OperatorTok{=}\NormalTok{ math.pi }\OperatorTok{*}\NormalTok{ (D1}\OperatorTok{**}\DecValTok{2}\NormalTok{) }\OperatorTok{/} \DecValTok{4}  \CommentTok{\# Area at the inlet}
\NormalTok{A2 }\OperatorTok{=}\NormalTok{ math.pi }\OperatorTok{*}\NormalTok{ (D2}\OperatorTok{**}\DecValTok{2}\NormalTok{) }\OperatorTok{/} \DecValTok{4}  \CommentTok{\# Area at the throat}

\CommentTok{\# {-}{-}{-}{-}{-}{-}{-}{-}{-}{-}{-}{-}{-}{-}{-}{-}{-}{-}{-}{-}{-}{-}{-}{-}{-}{-}{-}{-}{-}}
\CommentTok{\# Equivalent water head}
\CommentTok{\# {-}{-}{-}{-}{-}{-}{-}{-}{-}{-}{-}{-}{-}{-}{-}{-}{-}{-}{-}{-}{-}{-}{-}{-}{-}{-}{-}{-}{-}}
\CommentTok{\# The difference in water levels (equivalent head) generates}
\CommentTok{\# a hydrostatic pressure in the connecting legs that counteracts}
\CommentTok{\# the mercury column. Multiplying by (SG {-} 1) accounts for the}
\CommentTok{\# specific gravity of mercury relative to water.}
\NormalTok{h\_eq }\OperatorTok{=}\NormalTok{ h\_mercury }\OperatorTok{*}\NormalTok{ (SG }\OperatorTok{{-}} \DecValTok{1}\NormalTok{)}

\CommentTok{\# {-}{-}{-}{-}{-}{-}{-}{-}{-}{-}{-}{-}{-}{-}{-}{-}{-}{-}{-}{-}{-}{-}{-}{-}{-}{-}{-}{-}{-}}
\CommentTok{\# Diameter ratio}
\CommentTok{\# {-}{-}{-}{-}{-}{-}{-}{-}{-}{-}{-}{-}{-}{-}{-}{-}{-}{-}{-}{-}{-}{-}{-}{-}{-}{-}{-}{-}{-}}
\NormalTok{beta }\OperatorTok{=}\NormalTok{ D2 }\OperatorTok{/}\NormalTok{ D1}

\CommentTok{\# {-}{-}{-}{-}{-}{-}{-}{-}{-}{-}{-}{-}{-}{-}{-}{-}{-}{-}{-}{-}{-}{-}{-}{-}{-}{-}{-}{-}{-}}
\CommentTok{\# Discharge calculations}
\CommentTok{\# {-}{-}{-}{-}{-}{-}{-}{-}{-}{-}{-}{-}{-}{-}{-}{-}{-}{-}{-}{-}{-}{-}{-}{-}{-}{-}{-}{-}{-}}
\CommentTok{\# Theoretical discharge based on the ideal flow equation}
\NormalTok{Q\_theoretical }\OperatorTok{=}\NormalTok{ A2 }\OperatorTok{*}\NormalTok{ math.sqrt((}\DecValTok{2} \OperatorTok{*}\NormalTok{ g }\OperatorTok{*}\NormalTok{ h\_eq) }\OperatorTok{/}\NormalTok{ (}\DecValTok{1} \OperatorTok{{-}}\NormalTok{ beta}\OperatorTok{**}\DecValTok{4}\NormalTok{))}

\CommentTok{\# Actual discharge accounting for the coefficient of discharge}
\NormalTok{Q\_actual }\OperatorTok{=}\NormalTok{ Cd }\OperatorTok{*}\NormalTok{ Q\_theoretical}

\CommentTok{\# {-}{-}{-}{-}{-}{-}{-}{-}{-}{-}{-}{-}{-}{-}{-}{-}{-}{-}{-}{-}{-}{-}{-}{-}{-}{-}{-}{-}{-}}
\CommentTok{\# Output results}
\CommentTok{\# {-}{-}{-}{-}{-}{-}{-}{-}{-}{-}{-}{-}{-}{-}{-}{-}{-}{-}{-}{-}{-}{-}{-}{-}{-}{-}{-}{-}{-}}
\BuiltInTok{print}\NormalTok{(}\SpecialStringTok{f"Cross{-}sectional area at inlet (A1): }\SpecialCharTok{\{}\NormalTok{A1}\SpecialCharTok{:.4f\}}\SpecialStringTok{ m²"}\NormalTok{)}
\BuiltInTok{print}\NormalTok{(}\SpecialStringTok{f"Cross{-}sectional area at throat (A2): }\SpecialCharTok{\{}\NormalTok{A2}\SpecialCharTok{:.4f\}}\SpecialStringTok{ m²"}\NormalTok{)}
\BuiltInTok{print}\NormalTok{(}\SpecialStringTok{f"Equivalent water head (h\_eq): }\SpecialCharTok{\{}\NormalTok{h\_eq}\SpecialCharTok{:.4f\}}\SpecialStringTok{ m"}\NormalTok{)}
\BuiltInTok{print}\NormalTok{(}\SpecialStringTok{f"Theoretical discharge: }\SpecialCharTok{\{}\NormalTok{Q\_theoretical}\SpecialCharTok{:.4f\}}\SpecialStringTok{ m³/s"}\NormalTok{)}
\BuiltInTok{print}\NormalTok{(}\SpecialStringTok{f"Actual discharge: }\SpecialCharTok{\{}\NormalTok{Q\_actual}\SpecialCharTok{:.4f\}}\SpecialStringTok{ m³/s"}\NormalTok{)}
\end{Highlighting}
\end{Shaded}

\end{example}

\section{Power Required for Pumping
Water}\label{power-required-for-pumping-water}

The power needed to pump water is the rate at which work is done to
increase the fluid's potential energy by lifting it against gravity,
often with friction or other losses.

In ideal conditions, the theoretical power \(P\) is calculated as

\[
P = \rho g h \dot{v}
\]

where \(\rho\) is the fluid density, \(g\) is gravitational
acceleration, \(h\) is the total head, and \(\dot{v}\) is the volumetric
flow rate. This is derived from the mass flow rate
\(\dot{m} = \rho \dot{v}\).

Actual input power accounts for pump efficiency \(\eta\), given by

\[
\eta = \frac{P_{\mathrm{output}}}{P_{\mathrm{input}}}
\]

\begin{example}[]\protect\hypertarget{exm-fe2-3}{}\label{exm-fe2-3}

Water is pumped to a height of 20 m at a rate of 12 litres per second.
If the pump efficiency is 75\%, calculate the input power.

\begin{Shaded}
\begin{Highlighting}[]
\CommentTok{\# Given}
\NormalTok{flow\_rate\_lps }\OperatorTok{=} \FloatTok{12.0}    \CommentTok{\# litres per second}
\NormalTok{height }\OperatorTok{=} \FloatTok{20.0}           \CommentTok{\# metres}
\NormalTok{efficiency }\OperatorTok{=} \FloatTok{0.75}       
\NormalTok{rho }\OperatorTok{=} \FloatTok{1000.0}            \CommentTok{\# density of water in kg/m³}
\NormalTok{g }\OperatorTok{=} \FloatTok{9.81}                \CommentTok{\# acceleration due to gravity in m/s²}

\CommentTok{\# Calculations}
\NormalTok{v\_dot }\OperatorTok{=}\NormalTok{ flow\_rate\_lps }\OperatorTok{/} \FloatTok{1000.0}                   \CommentTok{\# volumetric flow rate in m³/s}
\NormalTok{m\_dot }\OperatorTok{=}\NormalTok{ rho }\OperatorTok{*}\NormalTok{ v\_dot                              }\CommentTok{\#  mass flow rate in kg/s}
\NormalTok{output\_power }\OperatorTok{=}\NormalTok{ m\_dot }\OperatorTok{*}\NormalTok{ g }\OperatorTok{*}\NormalTok{ height           }\CommentTok{\# watts}
\NormalTok{input\_power }\OperatorTok{=}\NormalTok{ output\_power }\OperatorTok{/}\NormalTok{ efficiency     }\CommentTok{\# watts}

\CommentTok{\# Result}
\BuiltInTok{print}\NormalTok{(}\SpecialStringTok{f"Actual input power (with 75\% efficiency): }\SpecialCharTok{\{}\NormalTok{input\_power}\SpecialCharTok{:.4f\}}\SpecialStringTok{ W (}\SpecialCharTok{\{}\NormalTok{input\_power }\OperatorTok{/} \DecValTok{1000}\SpecialCharTok{:.4f\}}\SpecialStringTok{ kW)"}\NormalTok{)}
\end{Highlighting}
\end{Shaded}

\end{example}

\begin{example}[]\protect\hypertarget{exm-similar-fe2-3}{}\label{exm-similar-fe2-3}

A water pump operates with an actual input power of 3 kW and an overall
efficiency of 70\%. The pump delivers water at a rate of 12 litres per
second. Assuming the power is used solely to increase the gravitational
potential energy (negligible velocity and pressure heads), calculate the
height (head) to which the water is pumped.

\begin{Shaded}
\begin{Highlighting}[]
\CommentTok{\# Given}
\NormalTok{input\_power }\OperatorTok{=} \FloatTok{3000.0}    \CommentTok{\# Actual input power in W}
\NormalTok{efficiency }\OperatorTok{=} \FloatTok{0.70}       \CommentTok{\# Pump efficiency}
\NormalTok{flow\_rate\_lps }\OperatorTok{=} \FloatTok{12.0}    \CommentTok{\# Flow rate in litres per second}
\NormalTok{rho }\OperatorTok{=} \FloatTok{1000.0}            \CommentTok{\# Density of water in kg/m³}
\NormalTok{g }\OperatorTok{=} \FloatTok{9.81}                \CommentTok{\# Gravitational acceleration in m/s²}

\CommentTok{\# Volumetric flow rate v\_dot (m³/s)}
\NormalTok{v\_dot }\OperatorTok{=}\NormalTok{ flow\_rate\_lps }\OperatorTok{/} \FloatTok{1000.0}

\CommentTok{\# Mass flow rate m\_dot (kg/s)}
\NormalTok{m\_dot }\OperatorTok{=}\NormalTok{ rho }\OperatorTok{*}\NormalTok{ v\_dot}

\CommentTok{\# Output power delivered to the water (W)}
\NormalTok{output\_power }\OperatorTok{=}\NormalTok{ input\_power }\OperatorTok{*}\NormalTok{ efficiency}

\CommentTok{\# Head h (m) using P = m\_dot * g * h}
\NormalTok{h }\OperatorTok{=}\NormalTok{ output\_power }\OperatorTok{/}\NormalTok{ (m\_dot }\OperatorTok{*}\NormalTok{ g)}

\CommentTok{\# Result}
\BuiltInTok{print}\NormalTok{(}\SpecialStringTok{f"Head (height pumped): }\SpecialCharTok{\{}\NormalTok{h}\SpecialCharTok{:.2f\}}\SpecialStringTok{ m"}\NormalTok{)}
\end{Highlighting}
\end{Shaded}

\end{example}

\bookmarksetup{startatroot}

\chapter*{Colophon}\label{colophon}
\addcontentsline{toc}{chapter}{Colophon}

\markboth{Colophon}{Colophon}

This booklet was typeset with \href{https://quarto.org}{Quarto}
v.1.8.26, using \(\LaTeX\) and Pandoc.

\cleardoublepage
\phantomsection
\addcontentsline{toc}{part}{Appendices}
\appendix

\chapter{Greek Letters}\label{greek-letters}

The following tables present the names of Greek letters and selected
symbols commonly used in engineering courses, ensuring precise reference
and avoiding reliance on informal descriptors such as ``squiggle.''

\begin{longtable}[]{@{}ccl@{}}
\caption{Greek letters.}\tabularnewline
\toprule\noalign{}
Lower Case & Upper Case & Name \\
\midrule\noalign{}
\endfirsthead
\toprule\noalign{}
Lower Case & Upper Case & Name \\
\midrule\noalign{}
\endhead
\bottomrule\noalign{}
\endlastfoot
\(\alpha\) & A & alpha \\
\(\beta\) & B & beta \\
\(\gamma\) & \(\Gamma\) & gamma \\
\(\delta\) & \(\Delta\) & delta \\
\(\epsilon\) & E & epsilon \\
\(\zeta\) & Z & zeta \\
\(\eta\) & E & eta \\
\(\theta\) & \(\Theta\) & theta \\
\(\iota\) & I & iota \\
\(\kappa\) & K & kappa \\
\(\lambda\) & \(\Lambda\) & lambda \\
\(\mu\) & M & mu \\
\(\nu\) & N & nu \\
\(\xi\) & \(\Xi\) & xi \\
\(\omicron\) & O & omicron \\
\(\pi\) & \(\Pi\) & pi \\
\(\rho\) & P & rho \\
\(\sigma\) & \(\Sigma\) & sigma \\
\(\tau\) & T & tau \\
\(\upsilon\) & \(\Upsilon\) & upsilon \\
\(\phi\) & \(\Phi\) & phi \\
\(\chi\) & X & chi \\
\(\psi\) & \(\Psi\) & psi \\
\(\omega\) & \(\Omega\) & omega \\
\end{longtable}

\begin{longtable}[]{@{}
  >{\centering\arraybackslash}p{(\linewidth - 6\tabcolsep) * \real{0.2361}}
  >{\raggedright\arraybackslash}p{(\linewidth - 6\tabcolsep) * \real{0.2361}}
  >{\raggedright\arraybackslash}p{(\linewidth - 6\tabcolsep) * \real{0.2361}}
  >{\raggedright\arraybackslash}p{(\linewidth - 6\tabcolsep) * \real{0.2917}}@{}}
\caption{Commonly used symbols in engineering courses.}\tabularnewline
\toprule\noalign{}
\begin{minipage}[b]{\linewidth}\centering
Symbol
\end{minipage} & \begin{minipage}[b]{\linewidth}\raggedright
Name
\end{minipage} & \begin{minipage}[b]{\linewidth}\raggedright
Use
\end{minipage} & \begin{minipage}[b]{\linewidth}\raggedright
Course
\end{minipage} \\
\midrule\noalign{}
\endfirsthead
\toprule\noalign{}
\begin{minipage}[b]{\linewidth}\centering
Symbol
\end{minipage} & \begin{minipage}[b]{\linewidth}\raggedright
Name
\end{minipage} & \begin{minipage}[b]{\linewidth}\raggedright
Use
\end{minipage} & \begin{minipage}[b]{\linewidth}\raggedright
Course
\end{minipage} \\
\midrule\noalign{}
\endhead
\bottomrule\noalign{}
\endlastfoot
\(\Delta\) & Delta & Change & Thermodynamics \\
\(\Delta\) & Delta & Displacement & Naval Architecture \\
\(\nabla\) & Nabla & Volume & Naval Architecture \\
\(\Sigma\) & Sigma & Sum & Thermodynamics, Naval Architecture, Applied
Mechanics \\
\(\sigma\) & Sigma & Stress & Thermodynamics, Applied Mechanics \\
\(\epsilon\) & Epsilon & Modulus of elasticity & Thermodynamics, Applied
Mechanics \\
\(\eta\) & Eta & Efficiency & Thermodynamics \\
\(\mu\) & Mu & Friction & Thermodynamics, Applied Mechanics \\
\(\omega\) & Omega & Angular velocity & Thermodynamics, Applied
Mechanics \\
\(\rho\) & Rho & Density & Thermodynamics, Naval Architecture \\
\(\tau\) & Tau & Torque & Thermodynamics, Applied Mechanics \\
\end{longtable}


\backmatter

\clearpage
\printindex


\end{document}
