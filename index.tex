% Options for packages loaded elsewhere
% Options for packages loaded elsewhere
\PassOptionsToPackage{unicode}{hyperref}
\PassOptionsToPackage{hyphens}{url}
%
\documentclass[
  letterpaper,
]{book}
\usepackage{xcolor}
\usepackage{amsmath,amssymb}
\setcounter{secnumdepth}{5}
\usepackage{iftex}
\ifPDFTeX
  \usepackage[T1]{fontenc}
  \usepackage[utf8]{inputenc}
  \usepackage{textcomp} % provide euro and other symbols
\else % if luatex or xetex
  \usepackage{unicode-math} % this also loads fontspec
  \defaultfontfeatures{Scale=MatchLowercase}
  \defaultfontfeatures[\rmfamily]{Ligatures=TeX,Scale=1}
\fi
\usepackage{lmodern}
\ifPDFTeX\else
  % xetex/luatex font selection
\fi
% Use upquote if available, for straight quotes in verbatim environments
\IfFileExists{upquote.sty}{\usepackage{upquote}}{}
\IfFileExists{microtype.sty}{% use microtype if available
  \usepackage[]{microtype}
  \UseMicrotypeSet[protrusion]{basicmath} % disable protrusion for tt fonts
}{}
\makeatletter
\@ifundefined{KOMAClassName}{% if non-KOMA class
  \IfFileExists{parskip.sty}{%
    \usepackage{parskip}
  }{% else
    \setlength{\parindent}{0pt}
    \setlength{\parskip}{6pt plus 2pt minus 1pt}}
}{% if KOMA class
  \KOMAoptions{parskip=half}}
\makeatother
% Make \paragraph and \subparagraph free-standing
\makeatletter
\ifx\paragraph\undefined\else
  \let\oldparagraph\paragraph
  \renewcommand{\paragraph}{
    \@ifstar
      \xxxParagraphStar
      \xxxParagraphNoStar
  }
  \newcommand{\xxxParagraphStar}[1]{\oldparagraph*{#1}\mbox{}}
  \newcommand{\xxxParagraphNoStar}[1]{\oldparagraph{#1}\mbox{}}
\fi
\ifx\subparagraph\undefined\else
  \let\oldsubparagraph\subparagraph
  \renewcommand{\subparagraph}{
    \@ifstar
      \xxxSubParagraphStar
      \xxxSubParagraphNoStar
  }
  \newcommand{\xxxSubParagraphStar}[1]{\oldsubparagraph*{#1}\mbox{}}
  \newcommand{\xxxSubParagraphNoStar}[1]{\oldsubparagraph{#1}\mbox{}}
\fi
\makeatother

\usepackage{color}
\usepackage{fancyvrb}
\newcommand{\VerbBar}{|}
\newcommand{\VERB}{\Verb[commandchars=\\\{\}]}
\DefineVerbatimEnvironment{Highlighting}{Verbatim}{commandchars=\\\{\}}
% Add ',fontsize=\small' for more characters per line
\usepackage{framed}
\definecolor{shadecolor}{RGB}{241,243,245}
\newenvironment{Shaded}{\begin{snugshade}}{\end{snugshade}}
\newcommand{\AlertTok}[1]{\textcolor[rgb]{0.68,0.00,0.00}{#1}}
\newcommand{\AnnotationTok}[1]{\textcolor[rgb]{0.37,0.37,0.37}{#1}}
\newcommand{\AttributeTok}[1]{\textcolor[rgb]{0.40,0.45,0.13}{#1}}
\newcommand{\BaseNTok}[1]{\textcolor[rgb]{0.68,0.00,0.00}{#1}}
\newcommand{\BuiltInTok}[1]{\textcolor[rgb]{0.00,0.23,0.31}{#1}}
\newcommand{\CharTok}[1]{\textcolor[rgb]{0.13,0.47,0.30}{#1}}
\newcommand{\CommentTok}[1]{\textcolor[rgb]{0.37,0.37,0.37}{#1}}
\newcommand{\CommentVarTok}[1]{\textcolor[rgb]{0.37,0.37,0.37}{\textit{#1}}}
\newcommand{\ConstantTok}[1]{\textcolor[rgb]{0.56,0.35,0.01}{#1}}
\newcommand{\ControlFlowTok}[1]{\textcolor[rgb]{0.00,0.23,0.31}{\textbf{#1}}}
\newcommand{\DataTypeTok}[1]{\textcolor[rgb]{0.68,0.00,0.00}{#1}}
\newcommand{\DecValTok}[1]{\textcolor[rgb]{0.68,0.00,0.00}{#1}}
\newcommand{\DocumentationTok}[1]{\textcolor[rgb]{0.37,0.37,0.37}{\textit{#1}}}
\newcommand{\ErrorTok}[1]{\textcolor[rgb]{0.68,0.00,0.00}{#1}}
\newcommand{\ExtensionTok}[1]{\textcolor[rgb]{0.00,0.23,0.31}{#1}}
\newcommand{\FloatTok}[1]{\textcolor[rgb]{0.68,0.00,0.00}{#1}}
\newcommand{\FunctionTok}[1]{\textcolor[rgb]{0.28,0.35,0.67}{#1}}
\newcommand{\ImportTok}[1]{\textcolor[rgb]{0.00,0.46,0.62}{#1}}
\newcommand{\InformationTok}[1]{\textcolor[rgb]{0.37,0.37,0.37}{#1}}
\newcommand{\KeywordTok}[1]{\textcolor[rgb]{0.00,0.23,0.31}{\textbf{#1}}}
\newcommand{\NormalTok}[1]{\textcolor[rgb]{0.00,0.23,0.31}{#1}}
\newcommand{\OperatorTok}[1]{\textcolor[rgb]{0.37,0.37,0.37}{#1}}
\newcommand{\OtherTok}[1]{\textcolor[rgb]{0.00,0.23,0.31}{#1}}
\newcommand{\PreprocessorTok}[1]{\textcolor[rgb]{0.68,0.00,0.00}{#1}}
\newcommand{\RegionMarkerTok}[1]{\textcolor[rgb]{0.00,0.23,0.31}{#1}}
\newcommand{\SpecialCharTok}[1]{\textcolor[rgb]{0.37,0.37,0.37}{#1}}
\newcommand{\SpecialStringTok}[1]{\textcolor[rgb]{0.13,0.47,0.30}{#1}}
\newcommand{\StringTok}[1]{\textcolor[rgb]{0.13,0.47,0.30}{#1}}
\newcommand{\VariableTok}[1]{\textcolor[rgb]{0.07,0.07,0.07}{#1}}
\newcommand{\VerbatimStringTok}[1]{\textcolor[rgb]{0.13,0.47,0.30}{#1}}
\newcommand{\WarningTok}[1]{\textcolor[rgb]{0.37,0.37,0.37}{\textit{#1}}}

\usepackage{longtable,booktabs,array}
\usepackage{calc} % for calculating minipage widths
% Correct order of tables after \paragraph or \subparagraph
\usepackage{etoolbox}
\makeatletter
\patchcmd\longtable{\par}{\if@noskipsec\mbox{}\fi\par}{}{}
\makeatother
% Allow footnotes in longtable head/foot
\IfFileExists{footnotehyper.sty}{\usepackage{footnotehyper}}{\usepackage{footnote}}
\makesavenoteenv{longtable}
\usepackage{graphicx}
\makeatletter
\newsavebox\pandoc@box
\newcommand*\pandocbounded[1]{% scales image to fit in text height/width
  \sbox\pandoc@box{#1}%
  \Gscale@div\@tempa{\textheight}{\dimexpr\ht\pandoc@box+\dp\pandoc@box\relax}%
  \Gscale@div\@tempb{\linewidth}{\wd\pandoc@box}%
  \ifdim\@tempb\p@<\@tempa\p@\let\@tempa\@tempb\fi% select the smaller of both
  \ifdim\@tempa\p@<\p@\scalebox{\@tempa}{\usebox\pandoc@box}%
  \else\usebox{\pandoc@box}%
  \fi%
}
% Set default figure placement to htbp
\def\fps@figure{htbp}
\makeatother





\setlength{\emergencystretch}{3em} % prevent overfull lines

\providecommand{\tightlist}{%
  \setlength{\itemsep}{0pt}\setlength{\parskip}{0pt}}



 


\usepackage{makeidx}
\makeindex
\makeatletter
\@ifpackageloaded{bookmark}{}{\usepackage{bookmark}}
\makeatother
\makeatletter
\@ifpackageloaded{caption}{}{\usepackage{caption}}
\AtBeginDocument{%
\ifdefined\contentsname
  \renewcommand*\contentsname{Table of contents}
\else
  \newcommand\contentsname{Table of contents}
\fi
\ifdefined\listfigurename
  \renewcommand*\listfigurename{List of Figures}
\else
  \newcommand\listfigurename{List of Figures}
\fi
\ifdefined\listtablename
  \renewcommand*\listtablename{List of Tables}
\else
  \newcommand\listtablename{List of Tables}
\fi
\ifdefined\figurename
  \renewcommand*\figurename{Figure}
\else
  \newcommand\figurename{Figure}
\fi
\ifdefined\tablename
  \renewcommand*\tablename{Table}
\else
  \newcommand\tablename{Table}
\fi
}
\@ifpackageloaded{float}{}{\usepackage{float}}
\floatstyle{ruled}
\@ifundefined{c@chapter}{\newfloat{codelisting}{h}{lop}}{\newfloat{codelisting}{h}{lop}[chapter]}
\floatname{codelisting}{Listing}
\newcommand*\listoflistings{\listof{codelisting}{List of Listings}}
\usepackage{amsthm}
\theoremstyle{definition}
\newtheorem{example}{Example}[chapter]
\theoremstyle{remark}
\AtBeginDocument{\renewcommand*{\proofname}{Proof}}
\newtheorem*{remark}{Remark}
\newtheorem*{solution}{Solution}
\newtheorem{refremark}{Remark}[chapter]
\newtheorem{refsolution}{Solution}[chapter]
\makeatother
\makeatletter
\makeatother
\makeatletter
\@ifpackageloaded{caption}{}{\usepackage{caption}}
\@ifpackageloaded{subcaption}{}{\usepackage{subcaption}}
\makeatother
\usepackage{bookmark}
\IfFileExists{xurl.sty}{\usepackage{xurl}}{} % add URL line breaks if available
\urlstyle{same}
\hypersetup{
  pdftitle={Tutorial},
  pdfauthor={Serhat Beyenir},
  hidelinks,
  pdfcreator={LaTeX via pandoc}}


\title{Tutorial}
\usepackage{etoolbox}
\makeatletter
\providecommand{\subtitle}[1]{% add subtitle to \maketitle
  \apptocmd{\@title}{\par {\large #1 \par}}{}{}
}
\makeatother
\subtitle{A Brief Introduction to Computational Tools}
\author{Serhat Beyenir}
\date{21 November 2025}
\begin{document}
\frontmatter
\maketitle

\clearpage
\vspace*{\fill}  % pushes text to bottom of page
\begin{center}
{\small
This work is licensed under a Creative Commons Attribution-ShareAlike 4.0 International License (CC BY-SA 4.0).\\[0.5em]
Full text available at \href{https://creativecommons.org/licenses/by-sa/4.0/}{https://creativecommons.org/licenses/by-sa/4.0/}.
}
\end{center}
\clearpage

\renewcommand*\contentsname{Contents}
{
\setcounter{tocdepth}{2}
\tableofcontents
}
\listoffigures
\listoftables

\mainmatter
\bookmarksetup{startatroot}

\chapter*{Preface}\label{preface}
\addcontentsline{toc}{chapter}{Preface}

\markboth{Preface}{Preface}

A Brief Introduction to Computational Tools presents a collection of
tutorials based on lecture notes from classes, designed to give learners
clear and essential insights into key topics. No previous programming
experience is required. Each tutorial guides you step by step through
the concepts with hands-on examples.

\bookmarksetup{startatroot}

\chapter{Python Tutorial}\label{sec-tutorial-basic}

This tutorial introduces Python programming, covering basic concepts
with examples to illustrate key points. We will start by using Python as
a calculator, then explore variables, functions, and control flow.

\section{Requirements}\label{requirements}

To follow this tutorial, the easiest way to get started is by using a
web-based Python environment. This lets you write and run Python code
right in your browser; no downloads or setup needed. I recommend
\href{https://python-fiddle.com/}{python-fiddle.com}, an easy-to-use
online editor that lets you experiment with Python instantly and solve
your problem sets effortlessly.

If you prefer working on your own computer, make sure you have
\href{https://www.python.org/downloads/}{Python}(version 3.10 or later)
installed. Python works on Windows, macOS, and Linux. You'll also need a
text editor or an Integrated Development Environment (IDE) to write your
code. I recommend \href{https://positron.posit.co/}{Positron}, a
beginner-friendly IDE with a built-in terminal, though other editors
like VS Code or PyCharm are also great options.

\section{Basic Syntax}\label{basic-syntax}

Python uses indentation\index{indentation} (typically four spaces) to
define code blocks. A colon\index{colon} (\texttt{:}) introduces a
block, and statements within the block must be indented consistently.
Python is case-sensitive, so \texttt{Variable} and \texttt{variable} are
distinct identifiers. \emph{Statements typically end with a newline, but
you can use a backslash (\texttt{\textbackslash{}}) to continue a
statement across multiple lines.}

\begin{Shaded}
\begin{Highlighting}[]
\NormalTok{total }\OperatorTok{=} \DecValTok{1} \OperatorTok{+} \DecValTok{2} \OperatorTok{+} \DecValTok{3} \OperatorTok{+} \DecValTok{4} \OperatorTok{+} \DecValTok{5}
\BuiltInTok{print}\NormalTok{(total)  }\CommentTok{\# Output: 15}
\end{Highlighting}
\end{Shaded}

Basic syntax rules:

\begin{itemize}
\tightlist
\item
  Comments start with \texttt{\#} and extend to the end of the line.
\item
  Strings can be enclosed in single quotes
  (\texttt{\textquotesingle{}}), double quotes (\texttt{"}), or triple
  quotes
  (\texttt{\textquotesingle{}\textquotesingle{}\textquotesingle{}} or
  \texttt{"""}) for multi-line strings.
\item
  Python is case-sensitive, so \texttt{Variable} and \texttt{variable}
  are different identifiers.
\end{itemize}

\section{\texorpdfstring{The \texttt{print()}
Function}{The print() Function}}\label{the-print-function}

The \texttt{print()} function\index{print} displays output in Python.

\begin{Shaded}
\begin{Highlighting}[]
\NormalTok{name }\OperatorTok{=} \StringTok{"Rudolf Diesel"}
\NormalTok{year }\OperatorTok{=} \DecValTok{1858}
\BuiltInTok{print}\NormalTok{(}\SpecialStringTok{f"}\SpecialCharTok{\{}\NormalTok{name}\SpecialCharTok{\}}\SpecialStringTok{ was born in }\SpecialCharTok{\{}\NormalTok{year}\SpecialCharTok{\}}\SpecialStringTok{."}\NormalTok{)}
\end{Highlighting}
\end{Shaded}

Output: \texttt{Rudolf\ Diesel\ was\ born\ in\ 1858.}

\section{\texorpdfstring{Formatting in
\texttt{print()}}{Formatting in print()}}\label{formatting-in-print}

The following table illustrates common f-string formatting options for
the \texttt{print()} function:

\begin{longtable}[]{@{}
  >{\raggedright\arraybackslash}p{(\linewidth - 6\tabcolsep) * \real{0.2466}}
  >{\raggedright\arraybackslash}p{(\linewidth - 6\tabcolsep) * \real{0.2466}}
  >{\raggedright\arraybackslash}p{(\linewidth - 6\tabcolsep) * \real{0.2603}}
  >{\raggedright\arraybackslash}p{(\linewidth - 6\tabcolsep) * \real{0.2466}}@{}}
\toprule\noalign{}
\begin{minipage}[b]{\linewidth}\raggedright
Format
\end{minipage} & \begin{minipage}[b]{\linewidth}\raggedright
Code
\end{minipage} & \begin{minipage}[b]{\linewidth}\raggedright
Example
\end{minipage} & \begin{minipage}[b]{\linewidth}\raggedright
Output
\end{minipage} \\
\midrule\noalign{}
\endhead
\bottomrule\noalign{}
\endlastfoot
\textbf{Round to 2 decimals} & \texttt{f"\{x:.2f\}"} &
\texttt{print(f"\{3.14159:.2f\}")} & \texttt{3.14} \\
\textbf{Round to whole number} & \texttt{f"\{x:.0f\}"} &
\texttt{print(f"\{3.9:.0f\}")} & \texttt{4} \\
\textbf{Thousands separator} & \texttt{f"\{x:,.2f\}"} &
\texttt{print(f"\{1234567.89:,.2f\}")} & \texttt{1,234,567.89} \\
\textbf{Percentage} & \texttt{f"\{x:.1\%\}"} &
\texttt{print(f"\{0.756:.1\%\}")} & \texttt{75.6\%} \\
\textbf{Currency style} & \texttt{f"\$\{x:,.2f\}"} &
\texttt{print(f"\$\{1234.5:,.2f\}")} & \texttt{\$1,234.50} \\
\end{longtable}

Note: The currency symbol (e.g., \texttt{\$}) can be modified for other
currencies (e.g., \texttt{€}, \texttt{£}) based on the desired locale.

\section{Variables and Data Types}\label{variables-and-data-types}

Variables\index{variables} store data and are assigned values using the
\texttt{=} operator.

\begin{Shaded}
\begin{Highlighting}[]
\NormalTok{x }\OperatorTok{=} \DecValTok{10}
\NormalTok{y }\OperatorTok{=} \FloatTok{3.14}
\NormalTok{name }\OperatorTok{=} \StringTok{"Rudolph"}
\end{Highlighting}
\end{Shaded}

Python has several built-in data types, including:

\begin{itemize}
\tightlist
\item
  Integers (\texttt{int}): Whole numbers, e.g., \texttt{10}, \texttt{-5}
\item
  Floating-point numbers (\texttt{float}): Decimal numbers, e.g.,
  \texttt{3.14}, \texttt{-0.001}
\item
  Strings (\texttt{str}): Text, e.g., \texttt{"Hello"},
  \texttt{\textquotesingle{}World\textquotesingle{}}
\item
  Booleans (\texttt{bool}): \texttt{True} or \texttt{False}
\end{itemize}

\subsection{Arithmetic Operations}\label{arithmetic-operations}

\begin{Shaded}
\begin{Highlighting}[]
\NormalTok{a }\OperatorTok{=} \DecValTok{10}
\NormalTok{b }\OperatorTok{=} \DecValTok{3}
\BuiltInTok{print}\NormalTok{(a }\OperatorTok{+}\NormalTok{ b)  }\CommentTok{\# Addition: 13}
\BuiltInTok{print}\NormalTok{(a }\OperatorTok{{-}}\NormalTok{ b)  }\CommentTok{\# Subtraction: 7}
\BuiltInTok{print}\NormalTok{(a }\OperatorTok{*}\NormalTok{ b)  }\CommentTok{\# Multiplication: 30}
\BuiltInTok{print}\NormalTok{(a }\OperatorTok{/}\NormalTok{ b)  }\CommentTok{\# Division: 3.3333...}
\BuiltInTok{print}\NormalTok{(a }\OperatorTok{//}\NormalTok{ b) }\CommentTok{\# Integer Division: 3}
\BuiltInTok{print}\NormalTok{(a }\OperatorTok{**}\NormalTok{ b) }\CommentTok{\# Exponentiation: 1000}
\end{Highlighting}
\end{Shaded}

\subsection{String Operations}\label{string-operations}

\begin{Shaded}
\begin{Highlighting}[]
\NormalTok{first\_name }\OperatorTok{=} \StringTok{"Rudolph"}
\NormalTok{last\_name }\OperatorTok{=} \StringTok{"Diesel"}
\NormalTok{full\_name }\OperatorTok{=}\NormalTok{ first\_name }\OperatorTok{+} \StringTok{" "} \OperatorTok{+}\NormalTok{ last\_name  }\CommentTok{\# Concatenation using +}
\BuiltInTok{print}\NormalTok{(full\_name)  }\CommentTok{\# Output: Rudolph Diesel}
\BuiltInTok{print}\NormalTok{(}\SpecialStringTok{f"}\SpecialCharTok{\{}\NormalTok{first\_name}\SpecialCharTok{\}}\SpecialStringTok{ }\SpecialCharTok{\{}\NormalTok{last\_name}\SpecialCharTok{\}}\SpecialStringTok{"}\NormalTok{)  }\CommentTok{\# Concatenation using f{-}string}
\BuiltInTok{print}\NormalTok{(full\_name }\OperatorTok{*} \DecValTok{2}\NormalTok{)  }\CommentTok{\# Repetition: Rudolph DieselRudolph Diesel}
\BuiltInTok{print}\NormalTok{(full\_name.upper())  }\CommentTok{\# Uppercase: RUDOLPH DIESEL}
\end{Highlighting}
\end{Shaded}

Note: String repetition (\texttt{*}) concatenates the string multiple
times without spaces. For example, \texttt{full\_name\ *\ 2} produces
\texttt{Rudolph\ DieselRudolph\ Diesel}.

\section{Python as a Calculator in Interactive
Mode}\label{python-as-a-calculator-in-interactive-mode}

Python's interactive mode allows you to enter commands and see results
immediately, ideal for quick calculations. To start, open a terminal (on
macOS, Linux, or Windows) and type:

\begin{Shaded}
\begin{Highlighting}[]
\ExtensionTok{python3}  \CommentTok{\# Use \textquotesingle{}python\textquotesingle{} on Windows if \textquotesingle{}python3\textquotesingle{} is not recognized}
\end{Highlighting}
\end{Shaded}

You should see the Python prompt:

\begin{Shaded}
\begin{Highlighting}[]
\OperatorTok{\textgreater{}\textgreater{}\textgreater{}}
\end{Highlighting}
\end{Shaded}

Enter expressions and press \textbf{Enter} to see results:

\begin{Shaded}
\begin{Highlighting}[]
\DecValTok{2} \OperatorTok{+} \DecValTok{3}  \CommentTok{\# Output: 5}
\DecValTok{7} \OperatorTok{{-}} \DecValTok{4}  \CommentTok{\# Output: 3}
\DecValTok{6} \OperatorTok{*} \DecValTok{9}  \CommentTok{\# Output: 54}
\DecValTok{8} \OperatorTok{/} \DecValTok{2}  \CommentTok{\# Output: 4.0}
\DecValTok{8} \OperatorTok{//} \DecValTok{2} \CommentTok{\# Output: 4}
\DecValTok{2} \OperatorTok{**} \DecValTok{3} \CommentTok{\# Output: 8}
\end{Highlighting}
\end{Shaded}

\subsection{Parentheses for Grouping}\label{parentheses-for-grouping}

\begin{Shaded}
\begin{Highlighting}[]
\NormalTok{(}\DecValTok{2} \OperatorTok{+} \DecValTok{3}\NormalTok{) }\OperatorTok{*} \DecValTok{4}  \CommentTok{\# Output: 20}
\DecValTok{2} \OperatorTok{+}\NormalTok{ (}\DecValTok{3} \OperatorTok{*} \DecValTok{4}\NormalTok{)  }\CommentTok{\# Output: 14}
\end{Highlighting}
\end{Shaded}

\subsection{Variables}\label{variables}

\begin{Shaded}
\begin{Highlighting}[]
\NormalTok{x }\OperatorTok{=} \DecValTok{10}
\NormalTok{y }\OperatorTok{=} \DecValTok{3}
\NormalTok{x }\OperatorTok{/}\NormalTok{ y  }\CommentTok{\# Output: 3.3333333333333335}
\end{Highlighting}
\end{Shaded}

\subsection{Exiting Interactive Mode}\label{exiting-interactive-mode}

To exit, type:

\begin{Shaded}
\begin{Highlighting}[]
\NormalTok{exit()}
\end{Highlighting}
\end{Shaded}

Alternatively, use: - \textbf{Ctrl+D} (macOS/Linux) - \textbf{Ctrl+Z}
then Enter (Windows)

\section{Control Flow}\label{control-flow}

Control flow\index{control flow} statements direct the execution of code
based on conditions.

\subsection{Conditional Statements}\label{conditional-statements}

Conditional statements\index{conditional statements} allow you to
execute different code blocks based on specific conditions. Python
provides three keywords for this purpose:

\begin{itemize}
\tightlist
\item
  \textbf{\texttt{if}}: Evaluates a condition and executes its code
  block if the condition is \texttt{True}.
\item
  \textbf{\texttt{elif}}: Short for ``else if,'' it checks an additional
  condition if the preceding \texttt{if} or \texttt{elif} conditions are
  \texttt{False}. You can use multiple \texttt{elif} statements to test
  multiple conditions sequentially, and Python will execute the first
  \texttt{True} condition's block, skipping the rest.
\item
  \textbf{\texttt{else}}: Executes a code block if none of the preceding
  \texttt{if} or \texttt{elif} conditions are \texttt{True}. It serves
  as a fallback and does not require a condition.
\end{itemize}

The following example uses age to categorize a person as a Minor, Adult,
or Senior, demonstrating how \texttt{if}, \texttt{elif}, and
\texttt{else} work together.

\begin{Shaded}
\begin{Highlighting}[]
\CommentTok{\# Categorize a person based on their age}
\NormalTok{age }\OperatorTok{=} \DecValTok{19}
\ControlFlowTok{if}\NormalTok{ age }\OperatorTok{\textless{}} \DecValTok{18}\NormalTok{:}
    \BuiltInTok{print}\NormalTok{(}\StringTok{"Minor"}\NormalTok{)}
\ControlFlowTok{elif}\NormalTok{ age }\OperatorTok{\textless{}=} \DecValTok{64}\NormalTok{:}
    \BuiltInTok{print}\NormalTok{(}\StringTok{"Adult"}\NormalTok{)}
\ControlFlowTok{else}\NormalTok{:}
    \BuiltInTok{print}\NormalTok{(}\StringTok{"Senior"}\NormalTok{)}
\end{Highlighting}
\end{Shaded}

Output: \texttt{Adult}

\subsection{For Loop}\label{for-loop}

A \texttt{for} loop\index{for loop} iterates over a sequence (e.g., list
or string).

\begin{Shaded}
\begin{Highlighting}[]
\NormalTok{components }\OperatorTok{=}\NormalTok{ [}\StringTok{"piston"}\NormalTok{, }\StringTok{"liner"}\NormalTok{, }\StringTok{"connecting rod"}\NormalTok{]}
\ControlFlowTok{for}\NormalTok{ component }\KeywordTok{in}\NormalTok{ components:}
    \BuiltInTok{print}\NormalTok{(component)}
\end{Highlighting}
\end{Shaded}

Output:

\begin{verbatim}
piston
liner
connecting rod
\end{verbatim}

\subsection{While Loop}\label{while-loop}

A \texttt{while} loop\index{while loop} executes as long as a condition
is true. Ensure the condition eventually becomes false to avoid infinite
loops.

\begin{Shaded}
\begin{Highlighting}[]
\NormalTok{count }\OperatorTok{=} \DecValTok{0}
\ControlFlowTok{while}\NormalTok{ count }\OperatorTok{\textless{}=} \DecValTok{5}\NormalTok{:}
    \BuiltInTok{print}\NormalTok{(count)}
\NormalTok{    count }\OperatorTok{+=} \DecValTok{1}
\end{Highlighting}
\end{Shaded}

Output:

\begin{verbatim}
0
1
2
3
4
5
\end{verbatim}

\section{Functions}\label{functions}

\subsection{\texorpdfstring{The \texttt{def}
Keyword}{The def Keyword}}\label{the-def-keyword}

Functions are reusable code blocks defined using the \texttt{def}
keyword\index{def keyword}. They can include default parameters for
optional arguments.

\begin{Shaded}
\begin{Highlighting}[]
\KeywordTok{def}\NormalTok{ add(a, b}\OperatorTok{=}\DecValTok{0}\NormalTok{):}
    \ControlFlowTok{return}\NormalTok{ a }\OperatorTok{+}\NormalTok{ b}
\BuiltInTok{print}\NormalTok{(add(}\DecValTok{5}\NormalTok{))      }\CommentTok{\# Output: 5}
\BuiltInTok{print}\NormalTok{(add(}\DecValTok{5}\NormalTok{, }\DecValTok{3}\NormalTok{))   }\CommentTok{\# Output: 8}

\KeywordTok{def}\NormalTok{ multiply(}\OperatorTok{*}\NormalTok{args):}
\NormalTok{    result }\OperatorTok{=} \DecValTok{1}
    \ControlFlowTok{for}\NormalTok{ num }\KeywordTok{in}\NormalTok{ args:}
\NormalTok{        result }\OperatorTok{*=}\NormalTok{ num}
    \ControlFlowTok{return}\NormalTok{ result}
\BuiltInTok{print}\NormalTok{(multiply(}\DecValTok{2}\NormalTok{, }\DecValTok{3}\NormalTok{, }\DecValTok{4}\NormalTok{))  }\CommentTok{\# Output: 24}
\end{Highlighting}
\end{Shaded}

\subsection{\texorpdfstring{The \texttt{lambda}
Keyword}{The lambda Keyword}}\label{the-lambda-keyword}

The \texttt{lambda} keyword\index{lambda keyword} creates anonymous
functions for short, one-off operations, often used in functional
programming.

\begin{Shaded}
\begin{Highlighting}[]
\NormalTok{celsius\_to\_fahrenheit }\OperatorTok{=} \KeywordTok{lambda}\NormalTok{ c: (c }\OperatorTok{*} \DecValTok{9} \OperatorTok{/} \DecValTok{5}\NormalTok{) }\OperatorTok{+} \DecValTok{32}
\BuiltInTok{print}\NormalTok{(celsius\_to\_fahrenheit(}\DecValTok{25}\NormalTok{))  }\CommentTok{\# Output: 77.0}
\end{Highlighting}
\end{Shaded}

\section{\texorpdfstring{The \texttt{math}
Module}{The math Module}}\label{the-math-module}

The \texttt{math} module\index{math module} provides mathematical
functions and constants.

\begin{Shaded}
\begin{Highlighting}[]
\ImportTok{import}\NormalTok{ math}
\BuiltInTok{print}\NormalTok{(math.sqrt(}\DecValTok{16}\NormalTok{))  }\CommentTok{\# Output: 4.0}
\BuiltInTok{print}\NormalTok{(math.pi)        }\CommentTok{\# Output: 3.141592653589793}
\end{Highlighting}
\end{Shaded}

\begin{Shaded}
\begin{Highlighting}[]
\ImportTok{import}\NormalTok{ math}
\NormalTok{angle }\OperatorTok{=}\NormalTok{ math.pi }\OperatorTok{/} \DecValTok{4}  \CommentTok{\# 45 degrees in radians}
\BuiltInTok{print}\NormalTok{(math.sin(angle))  }\CommentTok{\# Output: 0.7071067811865475 (approximately √2/2)}
\BuiltInTok{print}\NormalTok{(math.cos(angle))  }\CommentTok{\# Output: 0.7071067811865476 (approximately √2/2)}
\BuiltInTok{print}\NormalTok{(math.tan(angle))  }\CommentTok{\# Output: 1.0}
\end{Highlighting}
\end{Shaded}

Note: Floating-point arithmetic may result in small precision
differences, as seen in the \texttt{sin} and \texttt{cos} outputs.

\begin{Shaded}
\begin{Highlighting}[]
\ImportTok{import}\NormalTok{ math}
\BuiltInTok{print}\NormalTok{(math.log(}\DecValTok{10}\NormalTok{))       }\CommentTok{\# Natural logarithm of 10: 2.302585092994046}
\BuiltInTok{print}\NormalTok{(math.log(}\DecValTok{100}\NormalTok{, }\DecValTok{10}\NormalTok{))  }\CommentTok{\# Logarithm of 100 with base 10: 2.0}
\end{Highlighting}
\end{Shaded}

\subsection{Converting Between Radians and
Degrees}\label{converting-between-radians-and-degrees}

The \texttt{math} module provides \texttt{math.radians()} to convert
degrees to radians and \texttt{math.degrees()} to convert radians to
degrees, which is useful for trigonometric calculations.

\begin{Shaded}
\begin{Highlighting}[]
\ImportTok{import}\NormalTok{ math}
\NormalTok{degrees }\OperatorTok{=} \DecValTok{180}
\NormalTok{radians }\OperatorTok{=}\NormalTok{ math.radians(degrees)}
\BuiltInTok{print}\NormalTok{(}\SpecialStringTok{f"}\SpecialCharTok{\{}\NormalTok{degrees}\SpecialCharTok{\}}\SpecialStringTok{ degrees is }\SpecialCharTok{\{}\NormalTok{radians}\SpecialCharTok{:.3f\}}\SpecialStringTok{ radians"}\NormalTok{)  }\CommentTok{\# Output: 180 degrees is 3.142 radians}

\NormalTok{radians }\OperatorTok{=}\NormalTok{ math.pi }\OperatorTok{/} \DecValTok{2}
\NormalTok{degrees }\OperatorTok{=}\NormalTok{ math.degrees(radians)}
\BuiltInTok{print}\NormalTok{(}\SpecialStringTok{f"}\SpecialCharTok{\{}\NormalTok{radians}\SpecialCharTok{:.3f\}}\SpecialStringTok{ radians is }\SpecialCharTok{\{}\NormalTok{degrees}\SpecialCharTok{:.1f\}}\SpecialStringTok{ degrees"}\NormalTok{)  }\CommentTok{\# Output: 1.571 radians is 90.0 degrees}
\end{Highlighting}
\end{Shaded}

\section{Writing Python Scripts}\label{writing-python-scripts}

Write Python code in a \texttt{.py} file and run it as a script. Create
a file named \texttt{script.py}:

\begin{Shaded}
\begin{Highlighting}[]
\CommentTok{\# script.py}
\ImportTok{import}\NormalTok{ math}
\BuiltInTok{print}\NormalTok{(}\StringTok{"Square root of 16 is:"}\NormalTok{, math.sqrt(}\DecValTok{16}\NormalTok{))}
\BuiltInTok{print}\NormalTok{(}\StringTok{"Value of pi is:"}\NormalTok{, math.pi)}
\BuiltInTok{print}\NormalTok{(}\StringTok{"Sine of 90 degrees is:"}\NormalTok{, math.sin(math.pi }\OperatorTok{/} \DecValTok{2}\NormalTok{))}
\BuiltInTok{print}\NormalTok{(}\StringTok{"Natural logarithm of 10 is:"}\NormalTok{, math.log(}\DecValTok{10}\NormalTok{))}
\BuiltInTok{print}\NormalTok{(}\StringTok{"Logarithm of 100 with base 10 is:"}\NormalTok{, math.log(}\DecValTok{100}\NormalTok{, }\DecValTok{10}\NormalTok{))}
\end{Highlighting}
\end{Shaded}

To run the script, open a terminal, navigate to the directory containing
\texttt{script.py} using the \texttt{cd} command (e.g.,
\texttt{cd\ /path/to/directory}), and type:

\begin{Shaded}
\begin{Highlighting}[]
\ExtensionTok{python3}\NormalTok{ script.py  }\CommentTok{\# or python script.py on Windows}
\end{Highlighting}
\end{Shaded}

Output:

\begin{verbatim}
Square root of 16 is: 4.0
Value of pi is: 3.141592653589793
Sine of 90 degrees is: 1.0
Natural logarithm of 10 is: 2.302585092994046
Logarithm of 100 with base 10 is: 2.0
\end{verbatim}

\section{Summary}\label{summary}

This tutorial covered Python basics, including syntax, variables, data
types, operations, control flow, and functions. Python's rich ecosystem
includes libraries like:

\begin{itemize}
\tightlist
\item
  \textbf{NumPy}: For numerical computations and array manipulations.
\item
  \textbf{Matplotlib}: For data visualization and plotting.
\item
  \textbf{Pandas}: For data manipulation and analysis with tabular data
  structures.
\item
  \textbf{Pint}: For handling physical quantities and performing unit
  conversions.
\end{itemize}

You can explore these libraries to enhance your Python programming
skills further. For example installing them can be done using
\texttt{pip}:

\begin{Shaded}
\begin{Highlighting}[]
\ExtensionTok{pip}\NormalTok{ install numpy matplotlib pandas pint}
\end{Highlighting}
\end{Shaded}

\texttt{pip} is Python's package manager for installing and managing
additional libraries.

\bookmarksetup{startatroot}

\chapter{SI Units}\label{si_units}

The International System of Units (SI)\index{SI} is the globally
accepted standard for measurement. Established to provide a consistent
framework for scientific and technical measurements, SI units facilitate
clear communication and data comparison across various fields and
countries. The system is based on seven fundamental units: the meter for
length, the kilogram for mass, the second for time, the ampere for
electric current, the kelvin for temperature, the mole for substance,
and the candela for luminous intensity.

\newpage{}

\begin{longtable}[]{@{}lll@{}}
\caption{Base SI units.}\tabularnewline
\toprule\noalign{}
Physical Quantity & SI Base Unit & Symbol \\
\midrule\noalign{}
\endfirsthead
\toprule\noalign{}
Physical Quantity & SI Base Unit & Symbol \\
\midrule\noalign{}
\endhead
\bottomrule\noalign{}
\endlastfoot
Length & Meter & m \\
Mass & Kilogram & kg \\
Time & Second & s \\
Electric Current & Ampere & A \\
Temperature & Kelvin & K \\
Amount of Substance & Mole & mol \\
Luminous Intensity & Candela & cd \\
\end{longtable}

\begin{longtable}[]{@{}lll@{}}
\caption{Derived SI units.}\tabularnewline
\toprule\noalign{}
Physical Quantity & Derived SI Unit & Symbol \\
\midrule\noalign{}
\endfirsthead
\toprule\noalign{}
Physical Quantity & Derived SI Unit & Symbol \\
\midrule\noalign{}
\endhead
\bottomrule\noalign{}
\endlastfoot
Area & Square meter & m² \\
Volume & Cubic meter & m³ \\
Speed & Meter per second & m/s \\
Acceleration & Meter per second squared & m/s\textsuperscript{2} \\
Force & Newton & N \\
Pressure & Pascal & Pa \\
Energy & Joule & J \\
Power & Watt & W \\
Electric Charge & Coulomb & C \\
Electric Potential & Volt & V \\
Resistance & Ohm & Ω \\
Capacitance & Farad & F \\
Frequency & Hertz & Hz \\
Luminous Flux & Lumen & lm \\
Illuminance & Lux & lx \\
Specific Energy & Joule per kilogram & J/kg \\
Specific Heat Capacity & Joule per kilogram Kelvin & J/(kg·K) \\
\end{longtable}

\newpage{}

\begin{longtable}[]{@{}lll@{}}
\caption{Common multiples and submultiples for SI units.}\tabularnewline
\toprule\noalign{}
Factor & Prefix & Symbol \\
\midrule\noalign{}
\endfirsthead
\toprule\noalign{}
Factor & Prefix & Symbol \\
\midrule\noalign{}
\endhead
\bottomrule\noalign{}
\endlastfoot
10\textsuperscript{9} & giga & G \\
10\textsuperscript{6} & mega & M \\
10\textsuperscript{3} & kilo & k \\
10\textsuperscript{2} & hecto & h \\
10\textsuperscript{1} & deca & da \\
10\textsuperscript{-1} & deci & d \\
10\textsuperscript{-2} & centi & c \\
10\textsuperscript{-3} & milli & m \\
10\textsuperscript{-6} & micro & µ \\
\end{longtable}

\section{Metric Multipliers for Length, Area, and
Volume}\label{metric-multipliers-for-length-area-and-volume}

The metric system uses powers of ten to scale measurements. When
converting \textbf{area} or \textbf{volume}, remember that the
conversion factor must be \textbf{squared} or \textbf{cubed},
respectively.

\begin{longtable}[]{@{}
  >{\raggedright\arraybackslash}p{(\linewidth - 10\tabcolsep) * \real{0.1667}}
  >{\centering\arraybackslash}p{(\linewidth - 10\tabcolsep) * \real{0.1667}}
  >{\centering\arraybackslash}p{(\linewidth - 10\tabcolsep) * \real{0.1667}}
  >{\centering\arraybackslash}p{(\linewidth - 10\tabcolsep) * \real{0.1667}}
  >{\centering\arraybackslash}p{(\linewidth - 10\tabcolsep) * \real{0.1667}}
  >{\centering\arraybackslash}p{(\linewidth - 10\tabcolsep) * \real{0.1667}}@{}}
\toprule\noalign{}
\begin{minipage}[b]{\linewidth}\raggedright
Unit
\end{minipage} & \begin{minipage}[b]{\linewidth}\centering
Symbol
\end{minipage} & \begin{minipage}[b]{\linewidth}\centering
Multiplier (to metres)
\end{minipage} & \begin{minipage}[b]{\linewidth}\centering
1 m = ?
\end{minipage} & \begin{minipage}[b]{\linewidth}\centering
Area (1 m² = ?)
\end{minipage} & \begin{minipage}[b]{\linewidth}\centering
Volume (1 m³ = ?)
\end{minipage} \\
\midrule\noalign{}
\endhead
\bottomrule\noalign{}
\endlastfoot
\textbf{Millimetre} & mm & ( \(10^{-3}\) ) & 1000 \(\text{mm}\) &
\((1000)^2 = 10^6  \text{mm}^2\) & \((1000)^3 = 10^9  \text{mm}^3\) \\
\textbf{Centimetre} & cm & ( \(10^{-2}\) ) & 100 \(\text{cm}\) &
\((100)^2 = 10^4  \text{cm}^2\) & \((100)^3 = 10^6  \text{cm}^3\) \\
\textbf{Decimetre} & dm & ( \(10^{-1}\) ) & 10 \(\text{dm}\) &
\((10)^2 = 10^2  \text{dm}^2\) & \((10)^3 = 10^3  \text{dm}^3\) \\
\textbf{Metre} & m & ( \(10^0\) ) & --- & --- & --- \\
\textbf{Decametre} & dam & ( \(10^{1}\) ) & 0.1 \(\text{dam}\) &
\((0.1)^2 = 10^{-2}  \text{dam}^2\) &
\((0.1)^3 = 10^{-3}  \text{dam}^3\) \\
\textbf{Hectometre} & hm & ( \(10^{2}\) ) & 0.01 \(\text{hm}\) &
\((0.01)^2 = 10^{-4}  \text{hm}^2\) &
\((0.01)^3 = 10^{-6}  \text{hm}^3\) \\
\textbf{Kilometre} & km & ( \(10^{3}\) ) & 0.001 \(\text{km}\) &
\((0.001)^2 = 10^{-6}  \text{km}^2\) &
\((0.001)^3 = 10^{-9}  \text{km}^3\) \\
\end{longtable}

\begin{center}\rule{0.5\linewidth}{0.5pt}\end{center}

\subsection{\texorpdfstring{\textbf{Quick
Reference}}{Quick Reference}}\label{quick-reference}

\begin{longtable}[]{@{}
  >{\raggedright\arraybackslash}p{(\linewidth - 2\tabcolsep) * \real{0.5000}}
  >{\raggedright\arraybackslash}p{(\linewidth - 2\tabcolsep) * \real{0.5000}}@{}}
\toprule\noalign{}
\begin{minipage}[b]{\linewidth}\raggedright
Conversion Type
\end{minipage} & \begin{minipage}[b]{\linewidth}\raggedright
Relationship
\end{minipage} \\
\midrule\noalign{}
\endhead
\bottomrule\noalign{}
\endlastfoot
Length & 1 \(\text{m} = 10^3\) \(\text{mm} = 10^2\) \(\text{cm}\) \\
Area & 1 \(\text{m}^2 = 10^6\) \(\text{mm}^2 = 10^4\) \(\text{cm}^2\) \\
Volume & 1 \(\text{m}^3 = 10^9\) \(\text{mm}^3 = 10^6\)
\(\text{cm}^3\) \\
\end{longtable}

\begin{center}\rule{0.5\linewidth}{0.5pt}\end{center}

\textbf{Note:} When converting between metric units:

\begin{itemize}
\item
  Multiply by \(10^n\) when going to a smaller unit.
\item
  Divide by \(10^n\) when going to a larger unit.
\item
  For \textbf{area}, square the length factor; for \textbf{volume}, cube
  it.
\end{itemize}

\section{Unity Fraction}\label{unity-fraction}

The \textbf{unity fraction} method, or \textbf{unit conversion using
unity fractions}, is a systematic way to convert one unit of measurement
into another. This method relies on multiplying by fractions that are
equal to one, where the numerator and the denominator represent the same
quantity in different units. Since any number multiplied by one remains
the same, unity fractions allow for seamless conversion without changing
the value.

The principle of unity fractions is based on:

\begin{enumerate}
\def\labelenumi{\arabic{enumi}.}
\item
  \textbf{Setting up equal values}: Write a fraction where the numerator
  and denominator are equivalent values in different units, so the
  fraction equals one. For example, \(\frac{1km}{1000m}\) is a unity
  fraction because 1 km equals 1000 m.
\item
  \textbf{Multiplying by unity fractions}: Multiply the initial quantity
  by the unity fraction(s) so that the undesired units cancel out,
  leaving only the desired units.
\end{enumerate}

\subsection{Classwork}\label{classwork}

\begin{example}[]\protect\hypertarget{exm-0}{}\label{exm-0}

Suppose we want to convert \(5\) kilometers to meters.

\begin{enumerate}
\def\labelenumi{\arabic{enumi}.}
\tightlist
\item
  Start with \(5\) kilometers: \[
  5 \, \text{km}
  \]
\item
  Multiply by a unity fraction that cancels kilometers and introduces
  meters. We use
  \((\frac{1000 \, \text{m}}{1 \, \text{km}}), since\:1 \, \text{km} = 1000 \, \text{m}\):
\end{enumerate}

\[5 \, \text{km} \times \frac{1000 \, \text{m}}{1 \, \text{km}} = 5000 \, \text{m}\]

\begin{enumerate}
\def\labelenumi{\arabic{enumi}.}
\setcounter{enumi}{2}
\tightlist
\item
  The kilometers \(\text{km}\) cancel out, leaving us with meters
  \(\text{m}\):
\end{enumerate}

\[
5 \, \text{km} = 5000 \, \text{m}
\]

This step-by-step approach illustrates how the unity fraction cancels
the undesired units and achieves the correct result in meters.

Unity fractions can be extended by using multiple conversion steps. For
example, converting hours to seconds would require two unity fractions:
one to convert hours to minutes and another to convert minutes to
seconds. This approach ensures accuracy and is widely used in science,
engineering, and other fields that require precise unit conversions.

\end{example}

\begin{example}[]\protect\hypertarget{exm-1}{}\label{exm-1}

Convert \(15 \, \text{m/s}\) to \(\text{km/h}\).

\begin{enumerate}
\def\labelenumi{\arabic{enumi}.}
\tightlist
\item
  Start with \(15 \, \text{m/s}\).
\item
  To convert meters to kilometers, multiply by
  \(\frac{1 \, \text{km}}{1000 \, \text{m}}\).
\item
  To convert seconds to hours, multiply by
  \(\frac{3600 \, \text{s}}{1 \, \text{h}}\).
\end{enumerate}

\[
15 \, \text{m/s} \times \frac{1 \, \text{km}}{1000 \, \text{m}} \times \frac{3600 \, \text{s}}{1 \, \text{h}} = 54 \, \text{km/h}
\]

The meters and seconds cancel out, leaving kilometers per hour:
\(54 \, \text{km/h}\).

\end{example}

\subsection{Self-Test}\label{self-test}

\textbf{Instructions:}

\begin{enumerate}
\def\labelenumi{\arabic{enumi}.}
\item
  Use unity fraction to convert between derived SI units.
\item
  Show each step of your work to ensure accuracy.
\item
  Simplify your answers and include correct units.
\end{enumerate}

\begin{center}\rule{0.5\linewidth}{0.5pt}\end{center}

\begin{enumerate}
\def\labelenumi{\arabic{enumi}.}
\item
  \textbf{Speed}\\
  Convert \(72 \, \text{km/h}\) to \(\text{m/s}\).
\item
  \textbf{Force}\\
  Convert \(980 \, \text{N}\) (newtons) to
  \(\text{kg} \cdot \text{m/s}^2\).
\item
  \textbf{Energy}\\
  Convert \(2500 \, \text{J}\) (joules) to \(\text{kJ}\).
\item
  \textbf{Power}\\
  Convert \(1500 \, \text{W}\) (watts) to \(\text{kW}\).
\item
  \textbf{Pressure}\\
  Convert \(101325 \, \text{Pa}\) (pascals) to \(\text{kPa}\).
\item
  \textbf{Volume Flow Rate}\\
  Convert \(3 \, \text{m}^3/\text{min}\) to \(\text{L/s}\).
\item
  \textbf{Density}\\
  Convert \(1000 \, \text{kg/m}^3\) to \(\text{g/cm}^3\).
\item
  \textbf{Acceleration}\\
  Convert \(9.8 \, \text{m/s}^2\) to \(\text{cm/s}^2\).
\item
  \textbf{Torque}\\
  Convert \(50 \, \text{N} \cdot \text{m}\) to
  \(\text{kN} \cdot \text{cm}\).
\item
  \textbf{Frequency}\\
  Convert \(500 \, \text{Hz}\) (hertz) to \(\text{kHz}\).
\item
  \textbf{Work to Energy Conversion}\\
  A force of \(20 \, \text{N}\) moves an object \(500 \, \text{cm}\).
  Convert the work done to joules.
\item
  \textbf{Kinetic Energy Conversion}\\
  Calculate the kinetic energy in kilojoules of a \(1500 \, \text{kg}\)
  car moving at \(72 \, \text{km/h}\).
\item
  \textbf{Power to Energy Conversion}\\
  A machine operates at \(2 \, \text{kW}\) for \(3\) hours. Convert the
  energy used to megajoules.
\item
  \textbf{Pressure to Force Conversion}\\
  Convert a pressure of \(200 \, \text{kPa}\) applied to an area of
  \(0.5 \, \text{m}^2\) to force in newtons.
\item
  \textbf{Density to Mass Conversion}\\
  Convert \(0.8 \, \text{g/cm}^3\) for an object with a volume of
  \(250 \, \text{cm}^3\) to mass in grams.
\end{enumerate}

\begin{center}\rule{0.5\linewidth}{0.5pt}\end{center}

\subsection{Answer Key}\label{answer-key}

\begin{enumerate}
\def\labelenumi{\arabic{enumi}.}
\tightlist
\item
  \(72 \, \text{km/h} = 20 \, \text{m/s}\)
\item
  \(980 \, \text{N} = 980 \, \text{kg} \cdot \text{m/s}^2\)
\item
  \(2500 \, \text{J} = 2.5 \, \text{kJ}\)
\item
  \(1500 \, \text{W} = 1.5 \, \text{kW}\)
\item
  \(101325 \, \text{Pa} = 101.325 \, \text{kPa}\)
\item
  \(3 \, \text{m}^3/\text{min} = 50 \, \text{L/s}\)
\item
  \(1000 \, \text{kg/m}^3 = 1 \, \text{g/cm}^3\)
\item
  \(9.8 \, \text{m/s}^2 = 980 \, \text{cm/s}^2\)
\item
  \(50 \, \text{N} \cdot \text{m} = 5 \, \text{kN} \cdot \text{cm}\)
\item
  \(500 \, \text{Hz} = 0.5 \, \text{kHz}\)
\item
  \(20 \, \text{N} \times 5 \, \text{m} = 100 \, \text{J}\)
\item
  \(\text{Kinetic energy} = 1500 \, \text{kg} \times \left(20 \, \text{m/s}\right)^2 / 2 = 300 \, \text{kJ}\)
\item
  \(2 \, \text{kW} \times 3 \, \text{hours} = 21.6 \, \text{MJ}\)
\item
  \(200 \, \text{kPa} \times 0.5 \, \text{m}^2 = 100,000 \, \text{N}\)
\item
  \(0.8 \, \text{g/cm}^3 \times 250 \, \text{cm}^3 = 200 \, \text{g}\)
\end{enumerate}

\section{Condenser Vacuum}\label{condenser-vacuum}

Condenser vacuum gauge reads 715 mmHg when barometer stands at 757 mmHg.
State the absolute pressure in kN/m² and bar.

\subsection{Given Data}\label{given-data}

\[
P_{atm} = 757~\text{mmHg}, \quad P_{vac} = 715~\text{mmHg}
\]

\subsection{Absolute Pressure in mmHg}\label{absolute-pressure-in-mmhg}

\[
P_{abs} = P_{atm} - P_{vac} = 757 - 715 = 42~\text{mmHg}
\]

\subsection{Convert mmHg → kN/m²}\label{convert-mmhg-knmuxb2}

\[
P = \rho g h = 13{,}600 \cdot 9.81 \cdot 0.001 = 133.416~\text{Pa} = 0.133416~\text{kN/m}^2
\]

\[
P_{abs} = 42 \cdot 0.133416 = 5.6034~\text{kN/m²}
\]

\subsection{Convert kN/m² → bar}\label{convert-knmuxb2-bar}

\[
P_{abs} = \frac{5.6034}{100} = 0.056~\text{bar}
\]

\subsection{Final Answers}\label{final-answers}

\[
\boxed{P_{abs} = 42~\text{mmHg} = 5.6034~\text{kN/m²} = 0.056~\text{bar}}
\]

\subsection{Code}\label{code}

\begin{Shaded}
\begin{Highlighting}[]
\NormalTok{P\_atm\_mmHg }\OperatorTok{=} \DecValTok{757}
\NormalTok{P\_vac\_mmHg }\OperatorTok{=} \DecValTok{715}
\NormalTok{MMHG\_TO\_KN\_M2 }\OperatorTok{=} \FloatTok{0.133416}
\NormalTok{KNM2\_TO\_BAR }\OperatorTok{=} \DecValTok{1} \OperatorTok{/} \DecValTok{100}
\NormalTok{P\_abs\_mmHg }\OperatorTok{=}\NormalTok{ P\_atm\_mmHg }\OperatorTok{{-}}\NormalTok{ P\_vac\_mmHg}
\NormalTok{P\_abs\_kNm2 }\OperatorTok{=}\NormalTok{ P\_abs\_mmHg }\OperatorTok{*}\NormalTok{ MMHG\_TO\_KN\_M2}
\NormalTok{P\_abs\_bar }\OperatorTok{=}\NormalTok{ P\_abs\_kNm2 }\OperatorTok{*}\NormalTok{ KNM2\_TO\_BAR}
\BuiltInTok{print}\NormalTok{(}\SpecialStringTok{f"Absolute Pressure = }\SpecialCharTok{\{}\NormalTok{P\_abs\_mmHg}\SpecialCharTok{:.2f\}}\SpecialStringTok{ mmHg"}\NormalTok{)}
\BuiltInTok{print}\NormalTok{(}\SpecialStringTok{f"Absolute Pressure = }\SpecialCharTok{\{}\NormalTok{P\_abs\_kNm2}\SpecialCharTok{:.3f\}}\SpecialStringTok{ kN/m²"}\NormalTok{)}
\BuiltInTok{print}\NormalTok{(}\SpecialStringTok{f"Absolute Pressure = }\SpecialCharTok{\{}\NormalTok{P\_abs\_bar}\SpecialCharTok{:.4f\}}\SpecialStringTok{ bar"}\NormalTok{)}
\end{Highlighting}
\end{Shaded}

\section{Oil Flow in Tubes}\label{oil-flow-in-tubes}

Oil flows full bore at a velocity of \(V = 2~\text{m/s}\) through 16
tubes of diameter \(d = 30~\text{mm}\). Density of oil:
\(\rho = 0.85~\text{g/mL}\). Find \textbf{volume flow rate} (L/s) and
\textbf{mass flow rate} (kg/min).

\subsection{Cross-sectional area of one
tube}\label{cross-sectional-area-of-one-tube}

\[
A = \pi \frac{d^2}{4} = \pi \frac{0.03^2}{4} \approx 7.0686 \times 10^{-4}~\text{m}^2
\]

\subsection{Total area and volume flow
rate}\label{total-area-and-volume-flow-rate}

\[
A_\text{total} = 16 \cdot 7.0686 \times 10^{-4} \approx 0.01131~\text{m}^2
\]

\[
\dot{v} = A_\text{total} \cdot V \approx 0.02262~\text{m}^3/\text{s} \approx 22.62~\text{L/s}
\]

\subsection{Mass flow rate}\label{mass-flow-rate}

\[
\dot{m} = \rho \cdot \dot{v} = 850 \cdot 0.02262 \approx 19.227~\text{kg/s} \approx 1153.6~\text{kg/min}
\]

\subsection{Final Answers}\label{final-answers-1}

\[
\text{Volume flow rate: } \dot{v} \approx 22.6~\text{L/s}
\] \[
\text{Mass flow rate: } \dot{m} \approx 1154~\text{kg/min}
\]

\subsection{Code}\label{code-1}

\begin{Shaded}
\begin{Highlighting}[]
\ImportTok{import}\NormalTok{ math}
\NormalTok{v }\OperatorTok{=} \FloatTok{2.0}
\NormalTok{N }\OperatorTok{=} \DecValTok{16}
\NormalTok{d }\OperatorTok{=} \FloatTok{0.03}
\NormalTok{rho }\OperatorTok{=} \FloatTok{0.85} \OperatorTok{*} \DecValTok{1000}
\NormalTok{A }\OperatorTok{=}\NormalTok{ math.pi }\OperatorTok{*}\NormalTok{ d}\OperatorTok{**}\DecValTok{2} \OperatorTok{/} \DecValTok{4}
\NormalTok{A\_total }\OperatorTok{=}\NormalTok{ N }\OperatorTok{*}\NormalTok{ A}
\NormalTok{v\_dot\_m3\_s }\OperatorTok{=}\NormalTok{ A\_total }\OperatorTok{*}\NormalTok{ v}
\NormalTok{v\_dot\_L\_s }\OperatorTok{=}\NormalTok{ v\_dot\_m3\_s }\OperatorTok{*} \DecValTok{1000}
\NormalTok{m\_dot\_kg\_s }\OperatorTok{=}\NormalTok{ rho }\OperatorTok{*}\NormalTok{ v\_dot\_m3\_s}
\NormalTok{m\_dot\_kg\_min }\OperatorTok{=}\NormalTok{ m\_dot\_kg\_s }\OperatorTok{*} \DecValTok{60}
\BuiltInTok{print}\NormalTok{(}\SpecialStringTok{f"Volume flow rate: }\SpecialCharTok{\{}\NormalTok{v\_dot\_L\_s}\SpecialCharTok{:.2f\}}\SpecialStringTok{ L/s"}\NormalTok{)}
\BuiltInTok{print}\NormalTok{(}\SpecialStringTok{f"Mass flow rate: }\SpecialCharTok{\{}\NormalTok{m\_dot\_kg\_min}\SpecialCharTok{:.2f\}}\SpecialStringTok{ kg/min"}\NormalTok{)}
\end{Highlighting}
\end{Shaded}

\section{Gauge Pressure}\label{gauge-pressure}

An oil of specific gravity (relative density) \(\text{SG} = 0.8\) is
contained in a vessel to a depth of \(h = 2 \text{ m}\). Find the
\textbf{gauge pressure} at this depth in kPa.

\subsection{Gauge Pressure}\label{gauge-pressure-1}

\[
P_g = \rho g h
\]

where

\(\rho = \text{density of fluid (kg/m³)}\)

g= acceleration due to gravity (9.81 m/s²)

h = depth (m)

\subsection{Compute the density of oil using specific
gravity}\label{compute-the-density-of-oil-using-specific-gravity}

Specific gravity is defined as

\[
\text{SG} = \frac{\rho_{\text{oil}}}{\rho_{\text{water}}}
\]

where \(\rho_{\text{water}} = 1000 \text{ kg/m³}\). Thus,

\[
\rho_{\text{oil}} = \text{SG} \times \rho_{\text{water}} = 0.8 \times 1000 = 800\ \text{kg/m³}
\]

\subsection{Compute the gauge
pressure}\label{compute-the-gauge-pressure}

\[
P_g = \rho g h = 800 \times 9.81 \times 2
\]

\[
P_g = 15696\ \text{Pa} \approx 15.7\ \text{kPa}
\]

\subsection{Answer}\label{answer}

The \textbf{gauge pressure} at a depth of 2 m in the oil is:

\[
\boxed{15.7\ \text{kPa}}
\]

\subsection{Code}\label{code-2}

\begin{Shaded}
\begin{Highlighting}[]
\CommentTok{\# Gauge Pressure Calculation for Oil}

\CommentTok{\# Given data}
\NormalTok{specific\_gravity }\OperatorTok{=} \FloatTok{0.8}  \CommentTok{\# SG of oil}
\NormalTok{depth\_m }\OperatorTok{=} \FloatTok{2.0}           \CommentTok{\# depth in meters}
\NormalTok{g }\OperatorTok{=} \FloatTok{9.81}                \CommentTok{\# acceleration due to gravity in m/s²}
\NormalTok{rho\_water }\OperatorTok{=} \DecValTok{1000}        \CommentTok{\# density of water in kg/m³}

\CommentTok{\# Compute density of oil}
\NormalTok{rho\_oil }\OperatorTok{=}\NormalTok{ specific\_gravity }\OperatorTok{*}\NormalTok{ rho\_water}

\CommentTok{\# Compute gauge pressure (Pa)}
\NormalTok{P\_g\_Pa }\OperatorTok{=}\NormalTok{ rho\_oil }\OperatorTok{*}\NormalTok{ g }\OperatorTok{*}\NormalTok{ depth\_m}

\CommentTok{\# Convert to kPa}
\NormalTok{P\_g\_kPa }\OperatorTok{=}\NormalTok{ P\_g\_Pa }\OperatorTok{/} \DecValTok{1000}

\CommentTok{\# Print results}
\BuiltInTok{print}\NormalTok{(}\SpecialStringTok{f"Density of oil: }\SpecialCharTok{\{}\NormalTok{rho\_oil}\SpecialCharTok{:.1f\}}\SpecialStringTok{ kg/m³"}\NormalTok{)}
\BuiltInTok{print}\NormalTok{(}\SpecialStringTok{f"Gauge pressure at }\SpecialCharTok{\{}\NormalTok{depth\_m}\SpecialCharTok{\}}\SpecialStringTok{ m depth: }\SpecialCharTok{\{}\NormalTok{P\_g\_Pa}\SpecialCharTok{:.1f\}}\SpecialStringTok{ Pa (}\SpecialCharTok{\{}\NormalTok{P\_g\_kPa}\SpecialCharTok{:.2f\}}\SpecialStringTok{ kPa)"}\NormalTok{)}
\end{Highlighting}
\end{Shaded}

\section{Absolute Pressure from Manometer
Reading}\label{absolute-pressure-from-manometer-reading}

A water manometer shows a pressure in a vessel of \(400\ \text{mm}\)
\textbf{below atmospheric pressure}. The atmospheric pressure is
measured as \(763\ \text{mmHg}\). Determine the \textbf{absolute
pressure} in the vessel in kPa.

\subsection{Relationship between absolute and gauge
pressure}\label{relationship-between-absolute-and-gauge-pressure}

\[
P_\text{abs} = P_\text{atm} + P_\text{gauge}
\]

Since the manometer shows a pressure \textbf{below atmospheric}, the
gauge pressure is negative:

\[
P_\text{gauge} = - \rho_\text{water} g h
\]

\subsection{Convert atmospheric pressure to Pa
using}\label{convert-atmospheric-pressure-to-pa-using}

\[
P = \rho g h = 13{,}600 \cdot 9.81 \cdot 0.001 = 133.416~\text{Pa}
\]

So

\[
P_\text{atm} = 763 \times 133.416 \approx 101,801\ \text{Pa} \approx 101.8\ \text{kPa}
\]

\subsection{Compute gauge pressure}\label{compute-gauge-pressure}

Water column height:

\[
h = 400\ \text{mm} = 0.4\ \text{m}
\]

Density of water: \(rho_\text{water} = 1000\ \text{kg/m³}\),
\(g = 9.81\ \text{m/s²}\)

\[
P_\text{gauge} = - \rho g h = - 1000 \times 9.81 \times 0.4
\]

\[
P_\text{gauge} = -3924\ \text{Pa} \approx -3.92\ \text{kPa}
\]

\subsection{Compute absolute pressure}\label{compute-absolute-pressure}

\[
P_\text{abs} = P_\text{atm} + P_\text{gauge} \approx 101.8 - 3.92 \approx 97.9\ \text{kPa}
\]

\subsection{Answer}\label{answer-1}

The \textbf{absolute pressure} in the vessel is:

\[
\boxed{97.9\ \text{kPa}}
\]

\subsection{Code}\label{code-3}

\begin{Shaded}
\begin{Highlighting}[]
\CommentTok{\# Absolute Pressure Calculation from Water Manometer}

\CommentTok{\# Given data}
\NormalTok{h\_mm }\OperatorTok{=} \DecValTok{400}               \CommentTok{\# manometer reading in mm (below atmospheric)}
\NormalTok{atm\_mmHg }\OperatorTok{=} \DecValTok{763}           \CommentTok{\# atmospheric pressure in mmHg}
\NormalTok{rho\_water }\OperatorTok{=} \DecValTok{1000}         \CommentTok{\# density of water in kg/m³}
\NormalTok{g }\OperatorTok{=} \FloatTok{9.81}                 \CommentTok{\# gravity in m/s²}
\NormalTok{mmHg\_to\_Pa }\OperatorTok{=} \FloatTok{133.416}     \CommentTok{\# conversion factor}

\CommentTok{\# Convert manometer height to meters}
\NormalTok{h\_m }\OperatorTok{=}\NormalTok{ h\_mm }\OperatorTok{/} \DecValTok{1000}

\CommentTok{\# Convert atmospheric pressure to Pa}
\NormalTok{P\_atm\_Pa }\OperatorTok{=}\NormalTok{ atm\_mmHg }\OperatorTok{*}\NormalTok{ mmHg\_to\_Pa}

\CommentTok{\# Gauge pressure (negative because below atmospheric)}
\NormalTok{P\_gauge\_Pa }\OperatorTok{=} \OperatorTok{{-}}\NormalTok{ rho\_water }\OperatorTok{*}\NormalTok{ g }\OperatorTok{*}\NormalTok{ h\_m}

\CommentTok{\# Absolute pressure}
\NormalTok{P\_abs\_Pa }\OperatorTok{=}\NormalTok{ P\_atm\_Pa }\OperatorTok{+}\NormalTok{ P\_gauge\_Pa}

\CommentTok{\# Convert to kPa}
\NormalTok{P\_abs\_kPa }\OperatorTok{=}\NormalTok{ P\_abs\_Pa }\OperatorTok{/} \DecValTok{1000}

\CommentTok{\# Print results}
\BuiltInTok{print}\NormalTok{(}\SpecialStringTok{f"Atmospheric pressure: }\SpecialCharTok{\{}\NormalTok{P\_atm\_Pa}\SpecialCharTok{:.1f\}}\SpecialStringTok{ Pa (}\SpecialCharTok{\{}\NormalTok{P\_atm\_Pa}\OperatorTok{/}\DecValTok{1000}\SpecialCharTok{:.1f\}}\SpecialStringTok{ kPa)"}\NormalTok{)}
\BuiltInTok{print}\NormalTok{(}\SpecialStringTok{f"Gauge pressure: }\SpecialCharTok{\{}\NormalTok{P\_gauge\_Pa}\SpecialCharTok{:.1f\}}\SpecialStringTok{ Pa (}\SpecialCharTok{\{}\NormalTok{P\_gauge\_Pa}\OperatorTok{/}\DecValTok{1000}\SpecialCharTok{:.2f\}}\SpecialStringTok{ kPa)"}\NormalTok{)}
\BuiltInTok{print}\NormalTok{(}\SpecialStringTok{f"Absolute pressure in the vessel: }\SpecialCharTok{\{}\NormalTok{P\_abs\_Pa}\SpecialCharTok{:.1f\}}\SpecialStringTok{ Pa (}\SpecialCharTok{\{}\NormalTok{P\_abs\_kPa}\SpecialCharTok{:.2f\}}\SpecialStringTok{ kPa)"}\NormalTok{)}
\end{Highlighting}
\end{Shaded}

\bookmarksetup{startatroot}

\chapter{Heat and Work}\label{heat_work}

\section{Heat Required to Heat Steel}\label{heat-required-to-heat-steel}

A steel block of mass \(m = 5\ \text{kg}\) and specific heat capacity
\(c = 480\ \text{J/kg·K}\) is heated from \(T_1 = 15^\circ \text{C}\) to
\(T_2 = 100^\circ \text{C}\). Determine the \textbf{heat required}.

\subsection{The formula for heat}\label{the-formula-for-heat}

\begin{equation}\phantomsection\label{eq-heat}{
Q = m c \Delta T
}\end{equation}

where

\hfill\break
\(\Delta T = T_2 - T_1\) is the temperature change.

\subsection{Compute the temperature
change}\label{compute-the-temperature-change}

\[
\Delta T = T_2 - T_1 = 100 - 15 = 85\ \text{K}
\]

\subsection{Compute the heat required}\label{compute-the-heat-required}

\[
Q = m c \Delta T = 5 \times 480 \times 85
\]

\[
Q = 204,000\ \text{J}
\]

\subsection{Answer}\label{answer-2}

The \textbf{heat required} to raise the temperature of the steel is:

\[
\boxed{204\ \text{kJ}}
\]

\subsection{Code}\label{code-4}

\begin{Shaded}
\begin{Highlighting}[]
\CommentTok{\# Heat Required to Heat Steel}

\CommentTok{\# Given data}
\NormalTok{mass }\OperatorTok{=} \DecValTok{5}            \CommentTok{\# kg}
\NormalTok{specific\_heat }\OperatorTok{=} \DecValTok{480} \CommentTok{\# J/kg·K}
\NormalTok{T\_initial }\OperatorTok{=} \DecValTok{15}      \CommentTok{\# °C}
\NormalTok{T\_final }\OperatorTok{=} \DecValTok{100}       \CommentTok{\# °C}

\CommentTok{\# Temperature change}
\NormalTok{delta\_T }\OperatorTok{=}\NormalTok{ T\_final }\OperatorTok{{-}}\NormalTok{ T\_initial}

\CommentTok{\# Heat required (in J)}
\NormalTok{Q\_J }\OperatorTok{=}\NormalTok{ mass }\OperatorTok{*}\NormalTok{ specific\_heat }\OperatorTok{*}\NormalTok{ delta\_T}

\CommentTok{\# Convert to kJ}
\NormalTok{Q\_kJ }\OperatorTok{=}\NormalTok{ Q\_J }\OperatorTok{/} \DecValTok{1000}

\CommentTok{\# Print results}
\BuiltInTok{print}\NormalTok{(}\SpecialStringTok{f"Temperature change: }\SpecialCharTok{\{}\NormalTok{delta\_T}\SpecialCharTok{\}}\SpecialStringTok{ K"}\NormalTok{)}
\BuiltInTok{print}\NormalTok{(}\SpecialStringTok{f"Heat required: }\SpecialCharTok{\{}\NormalTok{Q\_J}\SpecialCharTok{:.0f\}}\SpecialStringTok{ J (}\SpecialCharTok{\{}\NormalTok{Q\_kJ}\SpecialCharTok{:.0f\}}\SpecialStringTok{ kJ)"}\NormalTok{)}
\end{Highlighting}
\end{Shaded}

\begin{Shaded}
\begin{Highlighting}[]
\CommentTok{\# Interactive Heat Calculation}

\CommentTok{\# Get user input}
\NormalTok{mass }\OperatorTok{=} \BuiltInTok{float}\NormalTok{(}\BuiltInTok{input}\NormalTok{(}\StringTok{"Enter the mass of the object (kg): "}\NormalTok{))}
\NormalTok{specific\_heat }\OperatorTok{=} \BuiltInTok{float}\NormalTok{(}\BuiltInTok{input}\NormalTok{(}\StringTok{"Enter the specific heat capacity (J/kg·K): "}\NormalTok{))}
\NormalTok{T\_initial }\OperatorTok{=} \BuiltInTok{float}\NormalTok{(}\BuiltInTok{input}\NormalTok{(}\StringTok{"Enter the initial temperature (°C): "}\NormalTok{))}
\NormalTok{T\_final }\OperatorTok{=} \BuiltInTok{float}\NormalTok{(}\BuiltInTok{input}\NormalTok{(}\StringTok{"Enter the final temperature (°C): "}\NormalTok{))}

\CommentTok{\# Calculate temperature change}
\NormalTok{delta\_T }\OperatorTok{=}\NormalTok{ T\_final }\OperatorTok{{-}}\NormalTok{ T\_initial}

\CommentTok{\# Calculate heat required}
\NormalTok{Q\_J }\OperatorTok{=}\NormalTok{ mass }\OperatorTok{*}\NormalTok{ specific\_heat }\OperatorTok{*}\NormalTok{ delta\_T}
\NormalTok{Q\_kJ }\OperatorTok{=}\NormalTok{ Q\_J }\OperatorTok{/} \DecValTok{1000}

\CommentTok{\# Display results}
\BuiltInTok{print}\NormalTok{(}\StringTok{"}\CharTok{\textbackslash{}n}\StringTok{Calculation Results:"}\NormalTok{)}
\BuiltInTok{print}\NormalTok{(}\SpecialStringTok{f"Temperature change: }\SpecialCharTok{\{}\NormalTok{delta\_T}\SpecialCharTok{:.2f\}}\SpecialStringTok{ K"}\NormalTok{)}
\BuiltInTok{print}\NormalTok{(}\SpecialStringTok{f"Heat required: }\SpecialCharTok{\{}\NormalTok{Q\_J}\SpecialCharTok{:.2f\}}\SpecialStringTok{ J (}\SpecialCharTok{\{}\NormalTok{Q\_kJ}\SpecialCharTok{:.2f\}}\SpecialStringTok{ kJ)"}\NormalTok{)}
\end{Highlighting}
\end{Shaded}

\section{Finding Specific Heat
Capacity}\label{finding-specific-heat-capacity}

A liquid of mass \(m = 4\ \text{kg}\) is heated from
\(T_1 = 15^\circ \text{C}\) to \(T_2 = 100^\circ \text{C}\). The heat
supplied is \(Q = 714\ \text{kJ}\). Determine the \textbf{specific heat
capacity} \(c\) of the liquid.

\subsection{Recall the formula for
heat}\label{recall-the-formula-for-heat}

\[
Q = m c \Delta T
\]

where \(\Delta T = T_2 - T_1\).

\subsection{Convert heat to joules}\label{convert-heat-to-joules}

\[
Q = 714\ \text{kJ} = 714 \times 1000 = 714,000\ \text{J}
\]

\subsection{Compute temperature
change}\label{compute-temperature-change}

\[
\Delta T = T_2 - T_1 = 100 - 15 = 85\ \text{K}
\]

\subsection{Solve for specific heat
capacity}\label{solve-for-specific-heat-capacity}

\[
c = \frac{Q}{m \Delta T} = \frac{714,000}{4 \times 85}
\]

\[
c = \frac{714,000}{340} \approx 2100\ \text{J/kg·K}
\]

\subsection{Answer}\label{answer-3}

The \textbf{specific heat capacity} of the liquid is:

\[
\boxed{c \approx 2100\ \text{J/kg·K}}
\]

\subsection{Code}\label{code-5}

\begin{Shaded}
\begin{Highlighting}[]
\CommentTok{\# Specific Heat Capacity Calculation}

\CommentTok{\# Given data}
\NormalTok{mass }\OperatorTok{=} \DecValTok{4}          \CommentTok{\# kg}
\NormalTok{T\_initial }\OperatorTok{=} \DecValTok{15}    \CommentTok{\# °C}
\NormalTok{T\_final }\OperatorTok{=} \DecValTok{100}     \CommentTok{\# °C}
\NormalTok{Q\_kJ }\OperatorTok{=} \DecValTok{714}        \CommentTok{\# heat supplied in kJ}

\CommentTok{\# Convert heat to joules}
\NormalTok{Q\_J }\OperatorTok{=}\NormalTok{ Q\_kJ }\OperatorTok{*} \DecValTok{1000}

\CommentTok{\# Temperature change}
\NormalTok{delta\_T }\OperatorTok{=}\NormalTok{ T\_final }\OperatorTok{{-}}\NormalTok{ T\_initial}

\CommentTok{\# Calculate specific heat capacity}
\NormalTok{c }\OperatorTok{=}\NormalTok{ Q\_J }\OperatorTok{/}\NormalTok{ (mass }\OperatorTok{*}\NormalTok{ delta\_T)}

\CommentTok{\# Print results}
\BuiltInTok{print}\NormalTok{(}\SpecialStringTok{f"Temperature change: }\SpecialCharTok{\{}\NormalTok{delta\_T}\SpecialCharTok{\}}\SpecialStringTok{ K"}\NormalTok{)}
\BuiltInTok{print}\NormalTok{(}\SpecialStringTok{f"Specific heat capacity: }\SpecialCharTok{\{}\NormalTok{c}\SpecialCharTok{:.0f\}}\SpecialStringTok{ J/kg·K"}\NormalTok{)}
\end{Highlighting}
\end{Shaded}

\begin{Shaded}
\begin{Highlighting}[]
\CommentTok{\# Interactive Specific Heat Capacity Calculator}

\CommentTok{\# Get user input}
\NormalTok{mass }\OperatorTok{=} \BuiltInTok{float}\NormalTok{(}\BuiltInTok{input}\NormalTok{(}\StringTok{"Enter the mass of the liquid (kg): "}\NormalTok{))}
\NormalTok{T\_initial }\OperatorTok{=} \BuiltInTok{float}\NormalTok{(}\BuiltInTok{input}\NormalTok{(}\StringTok{"Enter the initial temperature (°C): "}\NormalTok{))}
\NormalTok{T\_final }\OperatorTok{=} \BuiltInTok{float}\NormalTok{(}\BuiltInTok{input}\NormalTok{(}\StringTok{"Enter the final temperature (°C): "}\NormalTok{))}
\NormalTok{Q\_kJ }\OperatorTok{=} \BuiltInTok{float}\NormalTok{(}\BuiltInTok{input}\NormalTok{(}\StringTok{"Enter the heat supplied (kJ): "}\NormalTok{))}

\CommentTok{\# Convert heat to joules}
\NormalTok{Q\_J }\OperatorTok{=}\NormalTok{ Q\_kJ }\OperatorTok{*} \DecValTok{1000}

\CommentTok{\# Temperature change}
\NormalTok{delta\_T }\OperatorTok{=}\NormalTok{ T\_final }\OperatorTok{{-}}\NormalTok{ T\_initial}

\CommentTok{\# Calculate specific heat capacity}
\NormalTok{c }\OperatorTok{=}\NormalTok{ Q\_J }\OperatorTok{/}\NormalTok{ (mass }\OperatorTok{*}\NormalTok{ delta\_T)}

\CommentTok{\# Display results}
\BuiltInTok{print}\NormalTok{(}\StringTok{"}\CharTok{\textbackslash{}n}\StringTok{Calculation Results:"}\NormalTok{)}
\BuiltInTok{print}\NormalTok{(}\SpecialStringTok{f"Temperature change: }\SpecialCharTok{\{}\NormalTok{delta\_T}\SpecialCharTok{:.2f\}}\SpecialStringTok{ K"}\NormalTok{)}
\BuiltInTok{print}\NormalTok{(}\SpecialStringTok{f"Specific heat capacity: }\SpecialCharTok{\{}\NormalTok{c}\SpecialCharTok{:.2f\}}\SpecialStringTok{ J/kg·K"}\NormalTok{)}
\end{Highlighting}
\end{Shaded}

\section{Work Done by Fluid in
Cylinder}\label{work-done-by-fluid-in-cylinder}

A fluid in a cylinder is at pressure \(P = 700\ \text{kPa}\). It is
\textbf{expanded at constant pressure} from a volume of
\(V_1 = 0.28\ \text{m³}\) to \(V_2 = 1.68\ \text{m³}\). Determine the
\textbf{work done}.

\subsection{The formula for work done at constant
pressure}\label{the-formula-for-work-done-at-constant-pressure}

\begin{equation}\phantomsection\label{eq-workcp}{
W = P \Delta V
}\end{equation}

where

\[
\Delta V = V_2 - V_1
\]

\subsection{Compute the change in
volume}\label{compute-the-change-in-volume}

\[
\Delta V = 1.68 - 0.28 = 1.40\ \text{m³}
\]

\subsection{Convert pressure to Pa}\label{convert-pressure-to-pa}

\[
P = 700\ \text{kPa} = 700 \times 10^3\ \text{Pa} = 700,000\ \text{Pa}
\]

\subsection{Compute the work done}\label{compute-the-work-done}

\[
W = P \Delta V = 700,000 \times 1.40
\]

\[
W = 980,000\ \text{J} \approx 980\ \text{kJ}
\]

\subsection{Answer}\label{answer-4}

The \textbf{work done} by the fluid during expansion is:

\[
\boxed{980\ \text{kJ}}
\]

\subsection{Code}\label{code-6}

\begin{Shaded}
\begin{Highlighting}[]
\CommentTok{\# Work Done by Fluid at Constant Pressure}

\CommentTok{\# Given data}
\NormalTok{P\_kPa }\OperatorTok{=} \DecValTok{700}          \CommentTok{\# pressure in kPa}
\NormalTok{V1 }\OperatorTok{=} \FloatTok{0.28}            \CommentTok{\# initial volume in m³}
\NormalTok{V2 }\OperatorTok{=} \FloatTok{1.68}            \CommentTok{\# final volume in m³}

\CommentTok{\# Convert pressure to Pa}
\NormalTok{P\_Pa }\OperatorTok{=}\NormalTok{ P\_kPa }\OperatorTok{*} \DecValTok{1000}

\CommentTok{\# Compute change in volume}
\NormalTok{delta\_V }\OperatorTok{=}\NormalTok{ V2 }\OperatorTok{{-}}\NormalTok{ V1}

\CommentTok{\# Compute work done (J)}
\NormalTok{W\_J }\OperatorTok{=}\NormalTok{ P\_Pa }\OperatorTok{*}\NormalTok{ delta\_V}

\CommentTok{\# Convert to kJ}
\NormalTok{W\_kJ }\OperatorTok{=}\NormalTok{ W\_J }\OperatorTok{/} \DecValTok{1000}

\CommentTok{\# Print results}
\BuiltInTok{print}\NormalTok{(}\SpecialStringTok{f"Pressure: }\SpecialCharTok{\{}\NormalTok{P\_Pa}\SpecialCharTok{\}}\SpecialStringTok{ Pa"}\NormalTok{)}
\BuiltInTok{print}\NormalTok{(}\SpecialStringTok{f"Change in volume: }\SpecialCharTok{\{}\NormalTok{delta\_V}\SpecialCharTok{:.2f\}}\SpecialStringTok{ m³"}\NormalTok{)}
\BuiltInTok{print}\NormalTok{(}\SpecialStringTok{f"Work done: }\SpecialCharTok{\{}\NormalTok{W\_J}\SpecialCharTok{:.0f\}}\SpecialStringTok{ J (}\SpecialCharTok{\{}\NormalTok{W\_kJ}\SpecialCharTok{:.0f\}}\SpecialStringTok{ kJ)"}\NormalTok{)}
\end{Highlighting}
\end{Shaded}

\begin{Shaded}
\begin{Highlighting}[]
\CommentTok{\# Interactive Work Done Calculator}

\CommentTok{\# User input}
\NormalTok{P\_kPa }\OperatorTok{=} \BuiltInTok{float}\NormalTok{(}\BuiltInTok{input}\NormalTok{(}\StringTok{"Enter the constant pressure (kPa): "}\NormalTok{))}
\NormalTok{V1 }\OperatorTok{=} \BuiltInTok{float}\NormalTok{(}\BuiltInTok{input}\NormalTok{(}\StringTok{"Enter the initial volume (m³): "}\NormalTok{))}
\NormalTok{V2 }\OperatorTok{=} \BuiltInTok{float}\NormalTok{(}\BuiltInTok{input}\NormalTok{(}\StringTok{"Enter the final volume (m³): "}\NormalTok{))}

\CommentTok{\# Convert pressure to Pa}
\NormalTok{P\_Pa }\OperatorTok{=}\NormalTok{ P\_kPa }\OperatorTok{*} \DecValTok{1000}

\CommentTok{\# Compute change in volume}
\NormalTok{delta\_V }\OperatorTok{=}\NormalTok{ V2 }\OperatorTok{{-}}\NormalTok{ V1}

\CommentTok{\# Compute work done (J)}
\NormalTok{W\_J }\OperatorTok{=}\NormalTok{ P\_Pa }\OperatorTok{*}\NormalTok{ delta\_V}

\CommentTok{\# Convert to kJ}
\NormalTok{W\_kJ }\OperatorTok{=}\NormalTok{ W\_J }\OperatorTok{/} \DecValTok{1000}

\CommentTok{\# Display results}
\BuiltInTok{print}\NormalTok{(}\StringTok{"}\CharTok{\textbackslash{}n}\StringTok{Calculation Results:"}\NormalTok{)}
\BuiltInTok{print}\NormalTok{(}\SpecialStringTok{f"Pressure: }\SpecialCharTok{\{}\NormalTok{P\_Pa}\SpecialCharTok{:.0f\}}\SpecialStringTok{ Pa (}\SpecialCharTok{\{}\NormalTok{P\_kPa}\SpecialCharTok{:.0f\}}\SpecialStringTok{ kPa)"}\NormalTok{)}
\BuiltInTok{print}\NormalTok{(}\SpecialStringTok{f"Change in volume: }\SpecialCharTok{\{}\NormalTok{delta\_V}\SpecialCharTok{:.2f\}}\SpecialStringTok{ m³"}\NormalTok{)}
\BuiltInTok{print}\NormalTok{(}\SpecialStringTok{f"Work done: }\SpecialCharTok{\{}\NormalTok{W\_J}\SpecialCharTok{:.0f\}}\SpecialStringTok{ J (}\SpecialCharTok{\{}\NormalTok{W\_kJ}\SpecialCharTok{:.2f\}}\SpecialStringTok{ kJ)"}\NormalTok{)}
\end{Highlighting}
\end{Shaded}

\bookmarksetup{startatroot}

\chapter{Thermal Expansion}\label{thermal_expansion}

Thermal expansion is the tendency of materials to change their shape,
area, and volume in response to a change in temperature. When most
substances are heated, their particles move more vigorously and tend to
occupy more space, leading to an increase in dimensions. Conversely,
when substances are cooled, they generally contract. This phenomenon
occurs in solids, liquids, and gases, although the degree and nature of
expansion vary depending on the material's state and properties.

\section{Linear Expansion}\label{linear-expansion}

This occurs along a specific dimension or direction, primarily in long,
narrow objects (like rods or beams). When the temperature of a solid
object increases, its length expands by an amount proportional to its
original length and the temperature change. The equation for linear
expansion is:

\begin{equation}\phantomsection\label{eq-linearexp}{
\Delta L = \alpha L_0 \Delta T
}\end{equation} where:

\begin{itemize}
\item
  \(\Delta L\) is the change in length,
\item
  \(\alpha\) is the coefficient of linear expansion (unique to each
  material),
\item
  \(L_0\) is the original length, and
\item
  \(\Delta T\) is the temperature change.
\end{itemize}

\section{Superficial Expansion}\label{superficial-expansion}

Applicable to two-dimensional surfaces, such as sheets or plates. Here,
both length and width expand, leading to an increase in surface area.
The formula for area expansion is:
\begin{equation}\phantomsection\label{eq-superficial-exp}{
\Delta A = 2 \alpha A_0 \Delta T
}\end{equation} where:

\begin{itemize}
\tightlist
\item
  \(\Delta A\) is the change in area,
\item
  \(A_0\) is the initial area, and
\item
  \(\Delta T\) is the temperature change.
\end{itemize}

\section{Volumetric Expansion}\label{volumetric-expansion}

Relevant for three-dimensional objects (like solids, liquids, and
gases). The volume of an object expands with temperature, especially in
fluids where this effect is more pronounced. The formula is:
\begin{equation}\phantomsection\label{eq-volumetric-exp}{
\Delta V = \beta V_0 \Delta T
}\end{equation}

where:

\begin{itemize}
\tightlist
\item
  \(\Delta V\) is the change in volume,
\item
  \(\beta\) is the coefficient of volumetric expansion, which is
  approximately three times the linear expansion coefficient for
  isotropic solids,
\item
  \(V_0\) is the initial volume, and
\item
  \(\Delta T\) is the temperature change.
\end{itemize}

\section{Thermal Expansion of a Steel
Pipeline}\label{thermal-expansion-of-a-steel-pipeline}

A steel section of pipeline is \(75\ \mathrm{m}\) long when out of
service at an ambient temperature of \(20^\circ\mathrm{C}\). In service,
it transports steam at \(203^\circ\mathrm{C}\). Assuming the pipe is
free to expand, find its length at the operating temperature.\\
(The coefficient of linear expansion for steel is
\(\alpha = 12\times10^{-6}\ /\!^\circ\mathrm{C}\).)

Given:

\[
\begin{aligned}
L_0 &= 75\ \mathrm{m}, \\
T_1 &= 20^\circ\mathrm{C}, \\
T_2 &= 203^\circ\mathrm{C}, \\
\alpha &= 12\times10^{-6}\ /\!^\circ\mathrm{C}.
\end{aligned}
\]

The temperature rise is:

\[
\Delta T = T_2 - T_1 = 203 - 20 = 183^\circ\mathrm{C}.
\]

For linear expansion:

\[
L = L_0 (1 + \alpha \Delta T).
\]

Substitute the values:

\[
\begin{aligned}
L &= 75 \left(1 + 12\times10^{-6} \times 183\right) \\
  &= 75 \left(1 + 0.002196\right) \\
  &= 75.1647\ \mathrm{m}.
\end{aligned}
\]

\textbf{Final Answer}

\[
\boxed{L = 75.165\ \mathrm{m}}
\]

\textbf{Note:} The pipe expands by:

\[
\Delta L = L - L_0 = 0.165\ \mathrm{m} = 165\ \mathrm{mm}.
\]

\subsection{Code}\label{code-7}

\begin{Shaded}
\begin{Highlighting}[]
\CommentTok{\# Linear Expansion of a Steel Pipeline}
\CommentTok{\# {-}{-}{-}{-}{-}{-}{-}{-}{-}{-}{-}{-}{-}{-}{-}{-}{-}{-}{-}{-}{-}{-}{-}{-}{-}{-}{-}{-}{-}{-}{-}{-}{-}{-}{-}{-}}
\CommentTok{\# Given:}
\CommentTok{\#   L0 = 75 m   (initial length)}
\CommentTok{\#   T1 = 20 °C  (ambient temperature)}
\CommentTok{\#   T2 = 203 °C (operating temperature)}
\CommentTok{\#   α  = 12 × 10⁻⁶ /°C (coefficient of linear expansion for steel)}
\CommentTok{\#}
\CommentTok{\# Find:}
\CommentTok{\#   Final length L at 203 °C.}

\CommentTok{\# Given values}
\NormalTok{L0 }\OperatorTok{=} \FloatTok{75.0}                 \CommentTok{\# m}
\NormalTok{T1 }\OperatorTok{=} \FloatTok{20.0}                 \CommentTok{\# °C}
\NormalTok{T2 }\OperatorTok{=} \FloatTok{203.0}                \CommentTok{\# °C}
\NormalTok{alpha }\OperatorTok{=} \FloatTok{12e{-}6}             \CommentTok{\# /°C}

\CommentTok{\# Temperature change}
\NormalTok{delta\_T }\OperatorTok{=}\NormalTok{ T2 }\OperatorTok{{-}}\NormalTok{ T1}

\CommentTok{\# Final length using linear expansion formula}
\NormalTok{L }\OperatorTok{=}\NormalTok{ L0 }\OperatorTok{*}\NormalTok{ (}\DecValTok{1} \OperatorTok{+}\NormalTok{ alpha }\OperatorTok{*}\NormalTok{ delta\_T)}

\CommentTok{\# Expansion amount}
\NormalTok{expansion }\OperatorTok{=}\NormalTok{ L }\OperatorTok{{-}}\NormalTok{ L0}

\CommentTok{\# Display results}
\BuiltInTok{print}\NormalTok{(}\SpecialStringTok{f"Initial Length (m): }\SpecialCharTok{\{}\NormalTok{L0}\SpecialCharTok{:.2f\}}\SpecialStringTok{"}\NormalTok{)}
\BuiltInTok{print}\NormalTok{(}\SpecialStringTok{f"Final Length (m):   }\SpecialCharTok{\{}\NormalTok{L}\SpecialCharTok{:.3f\}}\SpecialStringTok{"}\NormalTok{)}
\BuiltInTok{print}\NormalTok{(}\SpecialStringTok{f"Expansion (m):      }\SpecialCharTok{\{}\NormalTok{expansion}\SpecialCharTok{:.3f\}}\SpecialStringTok{"}\NormalTok{)}
\end{Highlighting}
\end{Shaded}

\section{Thermal Expansion of a Brass
Cube}\label{thermal-expansion-of-a-brass-cube}

If a solid brass cube measures \textbf{50 mm × 50 mm × 50 mm} at
\textbf{10°C}, what volume will it occupy when heated to \textbf{78°C}?

Coefficient of linear expansion for brass,
\(\alpha = 18.4 \times 10^{-6}\,/^\circ \mathrm{C}\).

\textbf{Step 1: Given Data}

\[
\begin{aligned}
L_0 &= 50\ \mathrm{mm} = 0.05\ \mathrm{m}, \\
T_1 &= 10^\circ \mathrm{C}, \\
T_2 &= 78^\circ \mathrm{C}, \\
\Delta T &= T_2 - T_1 = 68^\circ \mathrm{C}, \\
\alpha &= 18.4 \times 10^{-6}\,/^\circ \mathrm{C}.
\end{aligned}
\]

\textbf{Step 2: Coefficient of Volumetric Expansion}

\[
\beta = 3\alpha = 3(18.4 \times 10^{-6}) = 55.2 \times 10^{-6}\,/^\circ \mathrm{C}.
\]

\textbf{Step 3: Initial and Final Volumes}

Initial volume:

\[
V_0 = L_0^3 = (0.05)^3 = 0.000125\ \mathrm{m^3}.
\]

Expanded volume:

\[
V = V_0(1 + \beta \Delta T)
\]

\[
V = 0.000125[1 + (55.2 \times 10^{-6})(68)].
\]

\[
V = 0.000125(1 + 0.0037536) = 0.00012547\ \mathrm{m^3}.
\]

\textbf{Step 4: Final Answer}

\[
\boxed{V = 0.00012547\ \mathrm{m^3}}
\]

\textbf{Note:} The multiplier from m³ to mm³ is 1 × 10⁹. The brass cube
expands from \textbf{125000 mm³} to \textbf{125469.20 mm³} when heated
from \textbf{10°C} to \textbf{78°C}.

\subsection{Code}\label{code-8}

\begin{Shaded}
\begin{Highlighting}[]
\CommentTok{\# Thermal Expansion of a Brass Cube}
\CommentTok{\# {-}{-}{-}{-}{-}{-}{-}{-}{-}{-}{-}{-}{-}{-}{-}{-}{-}{-}{-}{-}{-}{-}{-}{-}{-}{-}{-}{-}{-}{-}{-}{-}}
\CommentTok{\# Given:}
\CommentTok{\#   L0 = 50 mm  (initial side length)}
\CommentTok{\#   T1 = 10 °C  (initial temperature)}
\CommentTok{\#   T2 = 78 °C  (final temperature)}
\CommentTok{\#   α  = 18.4 × 10⁻⁶ /°C  (linear expansion coefficient for brass)}
\CommentTok{\#}
\CommentTok{\# Find:}
\CommentTok{\#   Final volume V at 78 °C (in mm³).}

\CommentTok{\# Given values}
\NormalTok{L0 }\OperatorTok{=} \FloatTok{50e{-}3}                \CommentTok{\# convert mm to m}
\NormalTok{T1 }\OperatorTok{=} \DecValTok{10}
\NormalTok{T2 }\OperatorTok{=} \DecValTok{78}
\NormalTok{alpha }\OperatorTok{=} \FloatTok{18.4e{-}6}           \CommentTok{\# /°C}

\CommentTok{\# Temperature change}
\NormalTok{delta\_T }\OperatorTok{=}\NormalTok{ T2 }\OperatorTok{{-}}\NormalTok{ T1}

\CommentTok{\# Volumetric expansion coefficient}
\NormalTok{beta }\OperatorTok{=} \DecValTok{3} \OperatorTok{*}\NormalTok{ alpha}

\CommentTok{\# Initial and final volumes (in m³)}
\NormalTok{V0 }\OperatorTok{=}\NormalTok{ L0 }\OperatorTok{**} \DecValTok{3}
\NormalTok{V }\OperatorTok{=}\NormalTok{ V0 }\OperatorTok{*}\NormalTok{ (}\DecValTok{1} \OperatorTok{+}\NormalTok{ beta }\OperatorTok{*}\NormalTok{ delta\_T)}

\CommentTok{\# Convert to mm³ for output (1 m³ = 1e9 mm³)}
\NormalTok{V0\_mm3 }\OperatorTok{=}\NormalTok{ V0 }\OperatorTok{*} \FloatTok{1e9}
\NormalTok{V\_mm3 }\OperatorTok{=}\NormalTok{ V }\OperatorTok{*} \FloatTok{1e9}
\NormalTok{delta\_V }\OperatorTok{=}\NormalTok{ V\_mm3 }\OperatorTok{{-}}\NormalTok{ V0\_mm3}

\CommentTok{\# Display results}
\BuiltInTok{print}\NormalTok{(}\SpecialStringTok{f"Initial Volume (mm³): }\SpecialCharTok{\{}\NormalTok{V0\_mm3}\SpecialCharTok{:.2f\}}\SpecialStringTok{"}\NormalTok{)}
\BuiltInTok{print}\NormalTok{(}\SpecialStringTok{f"Final Volume (mm³):   }\SpecialCharTok{\{}\NormalTok{V\_mm3}\SpecialCharTok{:.2f\}}\SpecialStringTok{"}\NormalTok{)}
\BuiltInTok{print}\NormalTok{(}\SpecialStringTok{f"Increase in Volume (mm³): }\SpecialCharTok{\{}\NormalTok{delta\_V}\SpecialCharTok{:.2f\}}\SpecialStringTok{"}\NormalTok{)}
\end{Highlighting}
\end{Shaded}

\bookmarksetup{startatroot}

\chapter{Heat Transfer}\label{heat_transfer}

\section{Conduction}\label{conduction}

Conduction transfers heat through \textbf{direct contact}, as
faster-vibrating molecules pass energy to slower ones. In solids, heat
moves from molecule to molecule, and between objects in contact, it
flows from the hotter region to the cooler one until thermal equilibrium
is reached.

An example of conduction\index{Conduction} is an iron bar with one end
placed in a flame. The other end soon becomes hot as heat is conducted
along the bar from molecule to molecule through the metal.

The rate of heat transfer by conduction through a solid is given by:

\begin{equation}\phantomsection\label{eq-conduction}{
Q = \frac{k \, A \, t \, \Delta T}{s}
}\end{equation}

where:

\begin{itemize}
\item
  \(Q\) = heat transferred (J)
\item
  \(k\) = thermal conductivity of the material (\(\text{W/m·K}\))
\item
  \(A\) = cross-sectional area through which heat flows (\(\text{m}^2\))
\item
  \(t\) = time of heat transfer (s)
\item
  \(\Delta T\) = temperature difference across the material (K or °C)
\item
  \(s\) = thickness or length of the material through which heat is
  conducted (m)
\end{itemize}

This equation shows that heat conduction increases with greater thermal
conductivity, larger surface area, longer time, and larger temperature
difference but decreases as the material's thickness increases.

\section{Temperature Difference Across a Heat Exchanger
Endplate}\label{temperature-difference-across-a-heat-exchanger-endplate}

The shell diameter of a large heat exchanger is 1.8 m. The flat endplate
(cover) at one end is made of carbon steel, 45 mm thick, and is
uninsulated. If the maximum allowable heat loss through the cover, to
avoid insulation is 150 MJ/h, determine the temperature difference
permitted across the endplate. (For steel, k = 55 W/m°C)

\textbf{Given}

\begin{itemize}
\tightlist
\item
  Shell diameter: (D = 1.8~\(\mathrm{m}\))\\
\item
  Endplate thickness: (s = 45~\(\mathrm{mm}\) = 0.045~\(\mathrm{m}\))\\
\item
  Material: carbon steel, thermal conductivity: (k =
  55~\(\mathrm{W/m\cdot K}\))\\
\item
  Maximum allowable heat loss:
  (\(Q_\mathrm{max} = 150\)~\(\mathrm{MJ/h}\))\\
\item
  Endplate is \textbf{uninsulated}
\end{itemize}

Calculate the \textbf{temperature difference} \(\Delta T\) across the
endplate to limit heat loss.

\begin{enumerate}
\def\labelenumi{\arabic{enumi}.}
\tightlist
\item
  \textbf{Area of the circular endplate}
\end{enumerate}

\[
A = \pi \left( \frac{D}{2} \right)^2 = \pi \left( \frac{1.8}{2} \right)^2
\]

\[
A = \pi \times 0.9^2 = 2.5447\ \mathrm{m^2}
\]

\begin{enumerate}
\def\labelenumi{\arabic{enumi}.}
\setcounter{enumi}{1}
\tightlist
\item
  \textbf{Convert heat loss to Watts}
\end{enumerate}

\[
Q_\mathrm{max} = 150\ \mathrm{MJ/h} = 150 \times 10^6\ \mathrm{J/h}
\]

\[
1\ \mathrm{h} = 3600\ \mathrm{s} \quad \Rightarrow \quad Q = \frac{150 \times 10^6}{3600} = 41666.7\ \mathrm{W}
\]

\begin{enumerate}
\def\labelenumi{\arabic{enumi}.}
\setcounter{enumi}{2}
\tightlist
\item
  \textbf{Use law of conduction for a flat plate}
\end{enumerate}

\[
Q = \frac{k A \Delta T}{s} \quad \Rightarrow \quad \Delta T = \frac{Q s}{k A}
\]

Substitute values:

\[
\Delta T = \frac{41666.7 \times 0.045}{55 \times 2.5447}
\]

\[
\Delta T = \frac{1875.0}{139.96} = 13.39\ \mathrm{K}
\]

\textbf{Result}

\[
\boxed{\Delta T = 13.4^\circ \mathrm{C}}
\]

\textbf{Notes:}

\begin{itemize}
\item
  The maximum allowable \textbf{temperature difference across the
  endplate} is 13.4°C to keep heat loss below 150 MJ/h.
\item
  If the temperature difference exceeds this, insulation would be
  required.
\end{itemize}

\subsection{Code}\label{code-9}

\begin{Shaded}
\begin{Highlighting}[]
\CommentTok{\# Given data}
\NormalTok{D }\OperatorTok{=} \FloatTok{1.8}              \CommentTok{\# diameter of endplate, m}
\NormalTok{L }\OperatorTok{=} \FloatTok{0.045}            \CommentTok{\# thickness, m}
\NormalTok{k }\OperatorTok{=} \DecValTok{55}               \CommentTok{\# thermal conductivity, W/m.K}
\NormalTok{Q\_MJ\_per\_h }\OperatorTok{=} \DecValTok{150}     \CommentTok{\# maximum heat loss, MJ/h}

\CommentTok{\# Convert heat loss to W}
\NormalTok{Q\_W }\OperatorTok{=}\NormalTok{ Q\_MJ\_per\_h }\OperatorTok{*} \FloatTok{1e6} \OperatorTok{/} \DecValTok{3600}  \CommentTok{\# W}

\CommentTok{\# Area of the circular plate}
\ImportTok{import}\NormalTok{ math}
\NormalTok{A }\OperatorTok{=}\NormalTok{ math.pi }\OperatorTok{*}\NormalTok{ (D}\OperatorTok{/}\DecValTok{2}\NormalTok{)}\OperatorTok{**}\DecValTok{2}

\CommentTok{\# Temperature difference}
\NormalTok{delta\_T }\OperatorTok{=}\NormalTok{ Q\_W }\OperatorTok{*}\NormalTok{ L }\OperatorTok{/}\NormalTok{ (k }\OperatorTok{*}\NormalTok{ A)}

\CommentTok{\# Display results}
\BuiltInTok{print}\NormalTok{(}\SpecialStringTok{f"Endplate area: }\SpecialCharTok{\{}\NormalTok{A}\SpecialCharTok{:.4f\}}\SpecialStringTok{ m²"}\NormalTok{)}
\BuiltInTok{print}\NormalTok{(}\SpecialStringTok{f"Heat loss: }\SpecialCharTok{\{}\NormalTok{Q\_W}\SpecialCharTok{:.2f\}}\SpecialStringTok{ W"}\NormalTok{)}
\BuiltInTok{print}\NormalTok{(}\SpecialStringTok{f"Temperature difference across the endplate: }\SpecialCharTok{\{}\NormalTok{delta\_T}\SpecialCharTok{:.2f\}}\SpecialStringTok{ °C"}\NormalTok{)}
\end{Highlighting}
\end{Shaded}

\section{Convection Heat Transfer}\label{convection-heat-transfer}

Convection transfers heat through the movement of fluids (liquids or
gases). When a fluid is heated, it expands, becomes less dense, and
rises, while cooler, denser fluid moves in to replace it, creating a
convection current that distributes heat.

Natural convection occurs without mechanical aid, whereas forced
convection involves devices such as pumps or fans. When fluid movement
is driven by a pump or fan, heat is transferred by \textbf{forced
convection}. Examples include:

\begin{itemize}
\tightlist
\item
  A pump circulating hot water through a building's heating system,
\item
  A fan forcing air through an automobile radiator, or
\item
  A forced draft fan pushing hot gases through a boiler.
\end{itemize}

The \textbf{total heat transferred} between a solid surface and a moving
fluid over a specified time is given by:

\begin{equation}\phantomsection\label{eq-convection}{
Q = h_A \, A \, t \, (T_s - T_f)
}\end{equation}

where:

\begin{itemize}
\item
  \(Q\) = total heat transferred (J)
\item
  \(h_A\) = surface (convective) heat transfer coefficient
  (\(\text{W/m}^2\text{·K}\))
\item
  \(A\) = surface area of heat transfer (\(\text{m}^2\))
\item
  \(t\) = time during which heat transfer occurs (s), with
  \(t = 1 \, \text{s}\) for unit time
\item
  \(T_s\) = surface temperature of the solid (K or °C)
\item
  \(T_f\) = temperature of the surrounding fluid (K or °C)
\end{itemize}

This equation describes how heat energy is transferred between a surface
and a fluid over a unit time of 1 second, depending on the
\textbf{temperature difference}, the \textbf{surface area}, the
\textbf{time}, and the fluid's heat-carrying effectiveness, represented
by the \textbf{convective heat transfer coefficient} (\(h_A\))

\section{Hot Metal Convection Heat
Transfer}\label{hot-metal-convection-heat-transfer}

A hot metal plate measuring \textbf{1.2 m × 0.8 m} is exposed to air at
\textbf{25°C}. The surface temperature of the plate is maintained at
\textbf{85°C}. If the \textbf{convective heat transfer coefficient}
(surface heat transfer coefficient) between the plate and air is
\(h_A = 18 \, \text{W/m}^2\text{·°C}\), calculate the \textbf{rate of
heat loss by convection} from the entire plate surface.

\textbf{Given:}

\begin{itemize}
\item
  Plate dimensions: (\(1.2\  \text{m} \times 0.8\  \text{m}\))
\item
  Plate surface area: \(A = 1.2 \times 0.8 = 0.96 \, \text{m}^2\)
\item
  Surface temperature: \(T_s = 85^\circ \text{C}\)
\item
  Air temperature: \(T_f = 25^\circ \text{C}\)
\item
  Convection coefficient: \(h_A = 18 \, \text{W/m}^2\text{·°C}\)
\end{itemize}

\textbf{Solution}

The rate of convective heat loss is given by:

\[
Q = h_A \, A \, (T_s - T_f)
\]

Substituting the values:

\[
Q = 18 \times 0.96 \times (85 - 25)
\]

\[
Q = 18 \times 0.96 \times 60
\]

\[
Q = 1036.8 \, \text{W}
\]

\textbf{Answer}

\[
\boxed{Q = 1036.8 \, \text{W}}
\]

\subsection{Code}\label{code-10}

\begin{Shaded}
\begin{Highlighting}[]
\CommentTok{\# Given data}
\NormalTok{L }\OperatorTok{=} \FloatTok{1.2}        \CommentTok{\# length of the plate (m)}
\NormalTok{W }\OperatorTok{=} \FloatTok{0.8}        \CommentTok{\# width of the plate (m)}
\NormalTok{T\_surface }\OperatorTok{=} \DecValTok{85} \CommentTok{\# surface temperature (°C)}
\NormalTok{T\_air }\OperatorTok{=} \DecValTok{25}     \CommentTok{\# air temperature (°C)}
\NormalTok{hA }\OperatorTok{=} \DecValTok{18}        \CommentTok{\# convective heat transfer coefficient (W/m²·°C)}

\CommentTok{\# Calculate area of the plate}
\NormalTok{A }\OperatorTok{=}\NormalTok{ L }\OperatorTok{*}\NormalTok{ W}

\CommentTok{\# Calculate heat loss rate}
\NormalTok{Q }\OperatorTok{=}\NormalTok{ hA }\OperatorTok{*}\NormalTok{ A }\OperatorTok{*}\NormalTok{ (T\_surface }\OperatorTok{{-}}\NormalTok{ T\_air)}

\CommentTok{\# Display results}
\BuiltInTok{print}\NormalTok{(}\SpecialStringTok{f"Plate area: }\SpecialCharTok{\{}\NormalTok{A}\SpecialCharTok{:.2f\}}\SpecialStringTok{ m²"}\NormalTok{)}
\BuiltInTok{print}\NormalTok{(}\SpecialStringTok{f"Temperature difference: }\SpecialCharTok{\{}\NormalTok{T\_surface }\OperatorTok{{-}}\NormalTok{ T\_air}\SpecialCharTok{:.1f\}}\SpecialStringTok{ °C"}\NormalTok{)}
\BuiltInTok{print}\NormalTok{(}\SpecialStringTok{f"Convective heat loss: }\SpecialCharTok{\{}\NormalTok{Q}\SpecialCharTok{:.2f\}}\SpecialStringTok{ W"}\NormalTok{)}
\end{Highlighting}
\end{Shaded}

\section{Radiation}\label{radiation}

Radiation transfers heat through electromagnetic waves that travel in
straight lines and can pass through a vacuum. When these waves strike a
surface, they may be absorbed (increasing temperature), reflected, or
transmitted. Dark, rough surfaces absorb more radiation, while shiny,
smooth ones reflect it.

The heat transfer by radiation is given by:

\begin{equation}\phantomsection\label{eq-radiation}{
Q = \sigma \, \varepsilon \, A \, t \, (T_1^4 - T_2^4)
}\end{equation}

where:

\begin{itemize}
\item
  \(Q\) is the \textbf{heat energy transferred by radiation} (in
  kilojoules, kJ).
\item
  \(\sigma\) is the \textbf{Stefan--Boltzmann constant}, equal to
  \(5.6703 \times 10^{-11}\ \text{kW·m}^{-2}\text{·K}^{-4}\).
\item
  \(\varepsilon\) is the \textbf{emissivity} of the surface
  (dimensionless, between 0 and 1), indicating how efficiently the
  surface emits or absorbs radiation compared to a perfect blackbody.
\item
  \(A\) is the \textbf{surface area} of the radiating body (in square
  meters, m²).
\item
  \(t\) is the \textbf{time} during which heat transfer occurs (in
  seconds, s).
\item
  \(T_1\) is the \textbf{absolute temperature of the radiating surface}
  (in kelvin, K).
\item
  \(T_2\) is the \textbf{absolute temperature of the surroundings} (in
  kelvin, K).
\end{itemize}

This equation expresses the \textbf{net radiant heat transfer} between
two bodies at different temperatures, taking into account the emissivity
of the surface, its area, and the duration of heat exchange.

Examples include heat from the sun reaching Earth and radiant heat in a
boiler furnace.

In a steam boiler, radiation occurs in the furnace. Any heating surfaces
that are directly exposed to the furnace will receive heat directly by
radiation from the flame. These include the waterwalls and some
generating tubes of a watertube boiler, radiant superheater tubes
(located at the outlet of the furnace), and the furnace walls of a
firetube boiler.

\section{Radiant Heat from a Flat Circular
Plate}\label{radiant-heat-from-a-flat-circular-plate}

A flat circular plate is \textbf{400 mm} in diameter. Calculate the
theoretical quantity of heat radiated per hour when its temperature is
\textbf{227 °C} and the temperature of its surrounds is \textbf{27 °C}.

\textbf{Given}:

\begin{itemize}
\item
  Stefan--Boltzmann constant:
  \(\sigma = 5.6703\times10^{-11}\ \mathrm{kW\,m^{-2}\,K^{-4}}\)
\item
  Emissivity: \(\varepsilon = 1\) (ideal black body)
\end{itemize}

\begin{enumerate}
\def\labelenumi{\arabic{enumi}.}
\tightlist
\item
  \textbf{Convert temperatures to Kelvin}
\end{enumerate}

\[
T_1 = 227 + 273 = 500\ \mathrm{K}, \quad
T_2 = 27 + 273 = 300\ \mathrm{K}
\]

\begin{enumerate}
\def\labelenumi{\arabic{enumi}.}
\setcounter{enumi}{1}
\tightlist
\item
  \textbf{Calculate the area of the circular plate}
\end{enumerate}

\[
A = \pi \left( \frac{D}{2} \right)^2 = \pi \left( \frac{0.4}{2} \right)^2 \approx 0.125664\ \mathrm{m^2}
\]

\begin{enumerate}
\def\labelenumi{\arabic{enumi}.}
\setcounter{enumi}{2}
\tightlist
\item
  \textbf{Apply the formula}
\end{enumerate}

\[
Q = \sigma \times \varepsilon \times A \times t \times \left( T_1^4 - T_2^4 \right)
\]

Substituting the values:

\[
Q = 5.6703 \times 10^{-11} \times 1 \times 0.125664 \times 3600 \times \left(500^4 - 300^4\right)
\] \[
Q = 1395.46\ \mathrm{kWh}
\]

\textbf{Result}

\[
\text{Plate area: } 0.125664\ \mathrm{m^2}
\] \[
\text{Emissivity: } \varepsilon = 1
\] \[
\boxed{\text{Heat radiated (per hour): } 1395.46\ \mathrm{kWh}}
\]

\textbf{Note:} This represents the \textbf{theoretical maximum
radiation} for a perfect black body. Real materials would radiate less
depending on their emissivity \(\varepsilon < 1\).

\subsection{Code}\label{code-11}

\begin{Shaded}
\begin{Highlighting}[]
\ImportTok{import}\NormalTok{ math}

\CommentTok{\# Given data}
\NormalTok{sigma }\OperatorTok{=} \FloatTok{5.6703e{-}11}   \CommentTok{\# kW/m\^{}2/K\^{}4}
\NormalTok{epsilon }\OperatorTok{=} \FloatTok{1.0}       \CommentTok{\# emissivity (black body)}
\NormalTok{T1 }\OperatorTok{=} \DecValTok{227} \OperatorTok{+} \DecValTok{273}  \CommentTok{\# K (plate temperature)}
\NormalTok{T2 }\OperatorTok{=} \DecValTok{27} \OperatorTok{+} \DecValTok{273}   \CommentTok{\# K (surroundings)}
\NormalTok{d }\OperatorTok{=} \DecValTok{400} \OperatorTok{/} \DecValTok{1000}     \CommentTok{\# m (diameter)}
\NormalTok{A }\OperatorTok{=}\NormalTok{ math.pi }\OperatorTok{*}\NormalTok{ (d }\OperatorTok{/} \DecValTok{2}\NormalTok{)}\OperatorTok{**}\DecValTok{2}  \CommentTok{\# area in m²}

\CommentTok{\# Heat radiated per second (kW)}
\NormalTok{Q\_dot }\OperatorTok{=}\NormalTok{ sigma }\OperatorTok{*}\NormalTok{ epsilon }\OperatorTok{*}\NormalTok{ A }\OperatorTok{*}\NormalTok{ (T1}\OperatorTok{**}\DecValTok{4} \OperatorTok{{-}}\NormalTok{ T2}\OperatorTok{**}\DecValTok{4}\NormalTok{)}

\CommentTok{\# Heat radiated per hour (kWh)}
\NormalTok{Q\_hour }\OperatorTok{=}\NormalTok{ Q\_dot }\OperatorTok{*} \DecValTok{3600}

\CommentTok{\# Display results}
\BuiltInTok{print}\NormalTok{(}\SpecialStringTok{f"Plate area: }\SpecialCharTok{\{}\NormalTok{A}\SpecialCharTok{:.6f\}}\SpecialStringTok{ m²"}\NormalTok{)}
\BuiltInTok{print}\NormalTok{(}\SpecialStringTok{f"Emissivity (theoretical): }\SpecialCharTok{\{}\NormalTok{epsilon}\SpecialCharTok{\}}\SpecialStringTok{"}\NormalTok{)}
\BuiltInTok{print}\NormalTok{(}\SpecialStringTok{f"Heat radiated per second: }\SpecialCharTok{\{}\NormalTok{Q\_dot}\SpecialCharTok{:.4f\}}\SpecialStringTok{ kW"}\NormalTok{)}
\BuiltInTok{print}\NormalTok{(}\SpecialStringTok{f"Heat radiated per hour: }\SpecialCharTok{\{}\NormalTok{Q\_hour}\SpecialCharTok{:.2f\}}\SpecialStringTok{ kWh"}\NormalTok{)}
\end{Highlighting}
\end{Shaded}

\section{Combined Mode}\label{combined-mode}

A composite wall is made up of three layers:

\begin{itemize}
\tightlist
\item
  \textbf{Brickwork:} thickness (L\_1=150~\mathrm{mm}), thermal
  conductivity (k\_1=1.20~\mathrm{W/m\cdot K}).\\
\item
  \textbf{Fibreglass:} thickness (L\_2=50~\mathrm{mm}), thermal
  conductivity (k\_2=0.04~\mathrm{W/m\cdot K}).\\
\item
  \textbf{Insulating board:} thickness (L\_3=40~\mathrm{mm}), thermal
  conductivity (k\_3=0.05~\mathrm{W/m\cdot K}).
\end{itemize}

Surface convective coefficients:\\
(h\_\{\text{in}\}=5.0~\mathrm{W/m^2\cdot K}) (inside),
(h\_\{\text{out}\}=10.0~\mathrm{W/m^2\cdot K}) (outside).

Internal ambient temperature (T\_\{\text{in}\}=22\^{}\circ\mathrm{C}),
external ambient temperature (T\_\{\text{out}\}=0\^{}\circ\mathrm{C}).

Wall dimensions: height (4~\mathrm{m}), length (8~\mathrm{m}) → area
(A=32~\mathrm{m^2}).

Calculate:

\begin{enumerate}
\def\labelenumi{\arabic{enumi}.}
\tightlist
\item
  The overall heat-transfer coefficient (U) (W/m²·K).\\
\item
  Heat lost per hour through the whole wall.\\
\item
  Internal and external \textbf{surface} temperatures of the wall.
\end{enumerate}

\textbf{Thermal resistances (per unit area)}

Convection (inside): {[}
R\_\{\text{conv,in}\}=\frac{1}{h_{\text{in}}}=\frac{1}{5.0}=0.2000~\mathrm{m^2K/W}.
{]}

Brickwork conduction: {[}
R\_\{\text{brick}\}=\frac{L_1}{k_1}=\frac{0.150}{1.20}=0.1250~\mathrm{m^2K/W}.
{]}

Fibreglass conduction: {[}
R\_\{\text{fib}\}=\frac{L_2}{k_2}=\frac{0.050}{0.04}=1.2500~\mathrm{m^2K/W}.
{]}

Insulating board conduction: {[}
R\_\{\text{ins}\}=\frac{L_3}{k_3}=\frac{0.040}{0.05}=0.8000~\mathrm{m^2K/W}.
{]}

Convection (outside): {[}
R\_\{\text{conv,out}\}=\frac{1}{h_{\text{out}}}=\frac{1}{10.0}=0.1000~\mathrm{m^2K/W}.
{]}

Total thermal resistance per unit area: {[}

\begin{aligned}
R_{\text{total}} &= R_{\text{conv,in}} + R_{\text{brick}} + R_{\text{fib}} + R_{\text{ins}} + R_{\text{conv,out}} \\
&= 0.2000 + 0.1250 + 1.2500 + 0.8000 + 0.1000 \\
&= 2.4750\ \mathrm{m^2K/W}.
\end{aligned}

{]}

Overall heat-transfer coefficient: {[} U = \frac{1}{R_{\text{total}}} =
\frac{1}{2.4750} \approx 0.4040~\mathrm{W/(m^2\cdot K)}. {]}

\textbf{Heat lost (whole wall)}

Temperature difference: {[} \Delta T = T\_\{\text{in}\} -
T\_\{\text{out}\} = 22 - 0 = 22~\mathrm{K}. {]}

Heat-loss rate (power): {[} \dot{Q} = U,A,\Delta T = 0.4040 \times 32
\times 22 \approx 284.42~\mathrm{W}. {]}

Energy lost in one hour:

{[} Q\_\{\text{hour}\} = \dot{Q}\times 3600 \approx 284.42 \times 3600
\approx 1.024\times10\^{}\{6\}~\mathrm{J}. {]}

Also in kWh: {[} Q\_\{\text{hour}\} \approx 0.2844~\mathrm{kWh}. {]}

\textbf{Surface temperatures}

Heat flux (per unit area): {[} q'\,' = \frac{\dot{Q}}{A}
\approx \frac{284.42}{32} \approx 8.888~\mathrm{W/m^2}. {]}

Internal surface temperature (wall inner face): {[} T\_\{\text{s,in}\} =
T\_\{\text{in}\} - q'\,',R\_\{\text{conv,in}\} = 22 - 8.888\times 0.2000
\approx 22 - 1.7776 \approx 20.22\^{}\circ\mathrm{C}. {]}

External surface temperature (wall outer face): {[} T\_\{\text{s,out}\}
= T\_\{\text{out}\} + q'\,',R\_\{\text{conv,out}\} = 0 +
8.888\times 0.1000 \approx 0.89\^{}\circ\mathrm{C}. {]}

\textbf{Final answers (rounded)}

\begin{enumerate}
\def\labelenumi{\arabic{enumi}.}
\tightlist
\item
  ( \displaystyle U \approx 0.404~\mathrm{W/(m^2\cdot K)}.)\\
\item
  Heat loss rate ( \dot{Q} \approx 284.4~\mathrm{W}) →
  (Q\_\{\text{hour}\}\approx 1.02\times10\^{}\{6\}~\mathrm{J}) (≈ 0.284
  kWh).\\
\item
  Surface temperatures:
  (T\_\{\text{s,in}\}\approx 20.22\circ\mathrm{C},\quad T\_\{\text{s,out}\}\approx 0.89\circ\mathrm{C}.)
\end{enumerate}

\cleardoublepage
\phantomsection
\addcontentsline{toc}{part}{Appendices}
\appendix

\chapter{SI System Common Mistakes}\label{si-system-common-mistakes}

Using the SI system correctly is crucial for clear communication in
science and engineering. Below are common mistakes in using the SI
system, examples of incorrect usage, and how to correct them.

\begin{longtable}[]{@{}
  >{\raggedright\arraybackslash}p{(\linewidth - 6\tabcolsep) * \real{0.2500}}
  >{\raggedright\arraybackslash}p{(\linewidth - 6\tabcolsep) * \real{0.2500}}
  >{\raggedright\arraybackslash}p{(\linewidth - 6\tabcolsep) * \real{0.2500}}
  >{\raggedright\arraybackslash}p{(\linewidth - 6\tabcolsep) * \real{0.2500}}@{}}
\caption{SI system rules and common mistakes}\tabularnewline
\toprule\noalign{}
\begin{minipage}[b]{\linewidth}\raggedright
\textbf{Concept}
\end{minipage} & \begin{minipage}[b]{\linewidth}\raggedright
\textbf{Mistake}
\end{minipage} & \begin{minipage}[b]{\linewidth}\raggedright
\textbf{Correct Usage}
\end{minipage} & \begin{minipage}[b]{\linewidth}\raggedright
\textbf{Notes}
\end{minipage} \\
\midrule\noalign{}
\endfirsthead
\toprule\noalign{}
\begin{minipage}[b]{\linewidth}\raggedright
\textbf{Concept}
\end{minipage} & \begin{minipage}[b]{\linewidth}\raggedright
\textbf{Mistake}
\end{minipage} & \begin{minipage}[b]{\linewidth}\raggedright
\textbf{Correct Usage}
\end{minipage} & \begin{minipage}[b]{\linewidth}\raggedright
\textbf{Notes}
\end{minipage} \\
\midrule\noalign{}
\endhead
\bottomrule\noalign{}
\endlastfoot
\textbf{Use of SI Unit Symbols} & \texttt{m./s} & \texttt{m/s} & Use the
correct format without additional punctuation. \\
\textbf{Spacing Between Value \& Unit} & \texttt{10kg} & \texttt{10\ kg}
& Always leave a space between the number and the unit symbol. \\
\textbf{Incorrect Unit Symbols} & \texttt{sec}, \texttt{hrs},
\texttt{°K} & \texttt{s}, \texttt{h}, \texttt{K} & Use the proper SI
symbols; symbols are case-sensitive. \\
\textbf{Abbreviations for Units} & \texttt{5\ kilograms\ (kgs)} &
\texttt{5\ kilograms\ (kg)} & Avoid informal abbreviations like ``kgs'';
adhere to standard symbols. \\
\textbf{Multiple Units in Expressions} & \texttt{5\ m/s/s},
\texttt{5\ kg/meter²} & \texttt{5\ m/s²}, \texttt{5\ kg/m²} & Use
compact, standardized formats for derived units. \\
\textbf{Incorrect Use of Prefixes} & \texttt{0.0001\ km} &
\texttt{100\ mm} & Choose prefixes to keep numbers in the range (0.1
\leq x \textless{} 1000). \\
\textbf{Misplaced Unit Symbols} & \texttt{5/s}, \texttt{kg10} &
\texttt{5\ s⁻¹}, \texttt{10\ kg} & Symbols must follow numerical values,
not precede them. \\
\textbf{Degrees Celsius vs.~Kelvin} & \texttt{300°K} & \texttt{300\ K} &
Kelvin is written without ``degree'' \\
\textbf{Singular vs.~Plural Units} & \texttt{5\ kgs}, \texttt{1\ meters}
& \texttt{5\ kg}, \texttt{1\ meter} & Symbols do not pluralize; full
unit names follow grammar rules. \\
\textbf{Capitalization of Symbols} & \texttt{Kg}, \texttt{S},
\texttt{Km}, \texttt{MA} & \texttt{kg}, \texttt{s}, \texttt{km},
\texttt{mA} & Symbols are case-sensitive; use uppercase only where
specified (e.g., \texttt{N}, \texttt{Pa}). \\
\textbf{Capitalization of Unit Names} & \texttt{Newton},
\texttt{Pascal}, \texttt{Watt} & \texttt{newton}, \texttt{pascal},
\texttt{watt} & Unit names are lowercase, even if derived from a
person's name, unless starting a sentence. \\
\textbf{Prefix Capitalization} & \texttt{MilliMeter}, \texttt{MegaWatt}
& \texttt{millimeter}, \texttt{megawatt} & Prefixes are lowercase for
(10\^{}\{-1\}) to (10\^{}\{-9\}), uppercase for (10\^{}6) and larger
(except \texttt{k} for kilo). \\
\textbf{Formatting in Reports} & \texttt{5}, \texttt{Temperature:\ 300}
& \texttt{5\ kg}, \texttt{Temperature:\ 300\ K} & Always specify units
explicitly. \\
\end{longtable}

\chapter{Greek Letters}\label{greek-letters}

The following tables present the names of Greek letters and selected
symbols commonly used in engineering courses, ensuring precise reference
and avoiding reliance on informal descriptors such as ``squiggle.''

\begin{longtable}[]{@{}ccl@{}}
\caption{Greek letters.}\tabularnewline
\toprule\noalign{}
Lower Case & Upper Case & Name \\
\midrule\noalign{}
\endfirsthead
\toprule\noalign{}
Lower Case & Upper Case & Name \\
\midrule\noalign{}
\endhead
\bottomrule\noalign{}
\endlastfoot
\(\alpha\) & A & alpha \\
\(\beta\) & B & beta \\
\(\gamma\) & \(\Gamma\) & gamma \\
\(\delta\) & \(\Delta\) & delta \\
\(\epsilon\) & E & epsilon \\
\(\zeta\) & Z & zeta \\
\(\eta\) & E & eta \\
\(\theta\) & \(\Theta\) & theta \\
\(\iota\) & I & iota \\
\(\kappa\) & K & kappa \\
\(\lambda\) & \(\Lambda\) & lambda \\
\(\mu\) & M & mu \\
\(\nu\) & N & nu \\
\(\xi\) & \(\Xi\) & xi \\
\(\omicron\) & O & omicron \\
\(\pi\) & \(\Pi\) & pi \\
\(\rho\) & P & rho \\
\(\sigma\) & \(\Sigma\) & sigma \\
\(\tau\) & T & tau \\
\(\upsilon\) & \(\Upsilon\) & upsilon \\
\(\phi\) & \(\Phi\) & phi \\
\(\chi\) & X & chi \\
\(\psi\) & \(\Psi\) & psi \\
\(\omega\) & \(\Omega\) & omega \\
\end{longtable}

\begin{longtable}[]{@{}
  >{\centering\arraybackslash}p{(\linewidth - 6\tabcolsep) * \real{0.2361}}
  >{\raggedright\arraybackslash}p{(\linewidth - 6\tabcolsep) * \real{0.2361}}
  >{\raggedright\arraybackslash}p{(\linewidth - 6\tabcolsep) * \real{0.2361}}
  >{\raggedright\arraybackslash}p{(\linewidth - 6\tabcolsep) * \real{0.2917}}@{}}
\caption{Commonly used symbols in engineering courses.}\tabularnewline
\toprule\noalign{}
\begin{minipage}[b]{\linewidth}\centering
Symbol
\end{minipage} & \begin{minipage}[b]{\linewidth}\raggedright
Name
\end{minipage} & \begin{minipage}[b]{\linewidth}\raggedright
Use
\end{minipage} & \begin{minipage}[b]{\linewidth}\raggedright
Course
\end{minipage} \\
\midrule\noalign{}
\endfirsthead
\toprule\noalign{}
\begin{minipage}[b]{\linewidth}\centering
Symbol
\end{minipage} & \begin{minipage}[b]{\linewidth}\raggedright
Name
\end{minipage} & \begin{minipage}[b]{\linewidth}\raggedright
Use
\end{minipage} & \begin{minipage}[b]{\linewidth}\raggedright
Course
\end{minipage} \\
\midrule\noalign{}
\endhead
\bottomrule\noalign{}
\endlastfoot
\(\Delta\) & Delta & Change & Thermodynamics \\
\(\Delta\) & Delta & Displacement & Naval Architecture \\
\(\nabla\) & Nabla & Volume & Naval Architecture \\
\(\Sigma\) & Sigma & Sum & Thermodynamics, Naval Architecture, Applied
Mechanics \\
\(\sigma\) & Sigma & Stress & Thermodynamics, Applied Mechanics \\
\(\epsilon\) & Epsilon & Modulus of elasticity & Thermodynamics, Applied
Mechanics \\
\(\eta\) & Eta & Efficiency & Thermodynamics \\
\(\mu\) & Mu & Friction & Thermodynamics, Applied Mechanics \\
\(\omega\) & Omega & Angular velocity & Thermodynamics, Applied
Mechanics \\
\(\rho\) & Rho & Density & Thermodynamics, Naval Architecture \\
\(\tau\) & Tau & Torque & Thermodynamics, Applied Mechanics \\
\end{longtable}


\backmatter

\printindex


\end{document}
